\documentclass[pra,12pt]{revtex4}
\usepackage{amsmath}
\usepackage{amssymb}
\usepackage{graphicx}
\usepackage{color}
\usepackage[pdfborder={0 0 0},colorlinks=true,linkcolor=blue]{hyperref}

\def\ket#1{\left|#1\right\rangle}
\def\bra#1{\left\langle#1\right|}
\def\braket#1{\left\langle#1\right\rangle}

\setlength{\parindent}{0pt}

\renewcommand{\baselinestretch}{1.0}
\setlength{\parskip}{0.07in}

\begin{document}

\begin{center}
{\large \textbf{Appendix A: Partial Wave Analysis}}
\end{center}

In this Appendix, we describe the method of \textbf{partial wave
  analysis}, which can be used to solve a specific but important class
of 3D scattering problems: those featuring a \textit{spherically
  symmetric} scattering potential, $V(r)$, which depends only on the
radial distance $r = \sqrt{x^2 + y^2 + z^2}$ and not on direction.
This typically describes a situation where a point particle or
spherically-symmetric object sits at the coordinate origin,
$\mathbf{r} = 0$, and is bombarded by incident particles.

\section{Incoming and outgoing spherical waves}

We begin by considering ``exterior'' solutions to the Schr\"odinger
wave equation, which apply at distances sufficiently far from the
scatterer so that $V(r) \rightarrow 0$.  In that case, the
Schr\"odinger wave equation can be cast into the convenient form
$$\Big(\nabla^2 + k^2\Big) \psi(\mathbf{r}) = 0,\;\;\;\mathrm{where}\;\;k = \sqrt{2mE/\hbar^2},$$
which is called the \textbf{Helmholtz equation}.  In spherical
coordinates $(r,\theta,\phi)$, it can be written explicitly as
$$\frac{1}{r^2}\frac{\partial}{\partial r}\left(r^2\frac{\partial \psi}{\partial r}\right) + \frac{1}{r^2\sin\theta}\frac{\partial}{\partial\theta}\left(\sin\theta\frac{\partial\psi}{\partial\theta}\right)+\frac{1}{r^2\sin^2\theta}\frac{\partial^2\psi}{\partial\phi^2} + k^2\psi(r,\theta,\phi) = 0.$$

There is a standard procedure for solving this partial differential
equation.  We will merely summarize the key steps here; the
mathematical details can be looked up elsewhere.  The first step is to
perform a separation of variables, and look for solutions of the
form
$$\psi(r,\theta,\phi) = A(r) Y_{\ell m}(\theta,\phi),$$
where $A(r)$ is a function to be determined and $Y_{\ell m}(\theta,\phi)$
is a special function known as a
\href{https://en.wikipedia.org/wiki/Spherical_harmonics}{spherical
  harmonic}.  The spherical harmonic functions are designed precisely
for the purpose of representing the angular dependence of waves with
definite angular momenta.  It can be shown that the indices $\ell$ and
$m$ must be integers satisfying $\ell \ge 0$ and $-\ell\le m \le \ell$, so that
the function is periodic in $\phi$ and regular at the poles of the
sphere.  In the context of quantum mechanics, $\ell$ and $m$ are the
quantum numbers representing the total angular momentum and the
$z$-component of the angular momentum, respectively.

Substituting into the Helmholtz equation yields the ordinary
differential equation
$$\frac{d}{dr}\left(r^2\frac{dA}{dr}\right) + \Big[k^2r^2 - \ell(\ell+1)\Big] A(r) = 0, \;\;\;\ell \in \mathbb{Z}_0^+,$$
which is a variant of the Bessel equation.  Note that there is no $m$
dependence; this is a manifestation of the spherical isotropy of the
problem.

The above equation has two linearly independent real solutions,
$j_\ell(kr)$ and $y_\ell(kr)$, which are called \textbf{spherical Bessel
  functions}.  Most numerical packages provide functions to calculate
these; in Scientific Python, for example, you can use the
\href{https://docs.scipy.org/doc/scipy/reference/generated/scipy.special.spherical_jn.html}{\texttt{scipy.special.spherical\_jn}}
and
\href{https://docs.scipy.org/doc/scipy/reference/generated/scipy.special.spherical_yn.html}{\texttt{scipy.special.spherical\_yn}}
functions.  Various spherical Bessel functions are plotted in the
figure below.  It is noteworthy that the $y_\ell(kr)$ functions
(spherical Bessel functions of the second type) are divergent at
$kr\rightarrow 0$; however, this does not bother us since we're
currently interested in exterior region solutions, defined away from
the coordinate origin.


When the inputs are large, the spherical Bessel functions have the
asymptotic forms
$$\begin{aligned}j_\ell(kr) &\overset{kr\rightarrow\infty}{\longrightarrow} \frac{\sin(kr-\ell\pi/2)}{kr} \\ y_\ell(kr) &\overset{kr\rightarrow\infty}{\longrightarrow} \frac{\cos(kr-\ell\pi/2)}{kr}.\end{aligned}$$



\end{document}
