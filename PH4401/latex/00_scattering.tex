\documentclass[pra,11pt]{revtex4}
\usepackage{amsmath}
\usepackage{amssymb}
\usepackage{graphicx}
\usepackage{color}
\def\ket#1{\left|#1\right\rangle}
\def\bra#1{\left\langle#1\right|}
\def\braket#1{\left\langle#1\right\rangle}

\setlength{\parindent}{0pt}

\renewcommand{\baselinestretch}{1.0}
\setlength{\parskip}{0.07in}

\begin{document}

\section{Scattering Processes in Quantum Mechanics}

\subsection{Scattering experiments on quantum particles}

Quantum particles exhibit \textbf{wave-particle duality}, a
fundamental phenomenon that can be summarized by a thought experiment
called the \textbf{quantum double-slit experiment}.  As shown in the
figure, a source emits electrons with energy $E$, which move towards a
screen with a pair of slits.  A detector is positioned on the other
side of the screen.  By moving the detector around, we can measure the
rate at which electrons are detected at different positions.

[Fig.]

According to quantum theory, the findings of the experiment are as
follows: (i) the detector shows electrons arriving in discrete
units---one at a time, like classical particles; (ii) when the
detector is moved around to measure the \textit{statistical}
distribution of the detection events, the result is an interference
pattern that looks like a classical wave diffracted by the slits.  The
wavelength $\lambda$ is related to the electron energy $E$ by
$$\lambda = \frac{2\pi}{k}, \;\;\; E = \frac{\hbar^2k^2}{2m},$$
where $\hbar = h/2\pi$ is Dirac's constant, and $m$ is the electron
mass.

Wave-particle duality arises from quantum theory's distinction between
a particle's state and the outcome of a measurement on it.  The state
is described by a wavefunction, $\psi(\mathbf{r})$, which can undergo
diffraction like a classical wave.  Measurement outcomes, however,
depend \textit{probabilistically} on the wavefunction (e.g., the
probability of locating a particle in a volume $dV$ around
$\mathbf{r}$ is $|\psi(\mathbf{r})|^2 \,dV$).

We will study a generalization of the quantum double-slit experiment,
called a \textbf{scattering experiment}.  The idea is to take an
object called a \textbf{scatterer}, shoot quantum particles at it, and
measure the resulting particle distribution.  Just as we can use the
double-slit interference pattern to deduce the spacing between the
slits, a scattering experiment can likewise be used to deduce various
facts about the scatterer.  Scattering experiments, as a class,
constitute a large proportion of the methods used to probe the quantum
world---ranging from electron- and photon-based laboratory experiments
for measuring the properties of materials, to huge accelerator
experiments for probing high-energy phenomena like the Higgs boson.

We will focus on a relatively simple scenario, consisting of a single
non-relativistic quantum particle and a classical scatterer.  Consider
a continuous and unbounded $d$-dimensional space, describable by
coordinates $\mathbf{r}$.  Somewhere around the origin, $\mathbf{r} =
0$, is a finite-sized scatterer.  An incoming quantum particle, with
energy $E$, interacts with the scatterer through the Hamiltonian
$$\hat{H} = \hat{H}_0 + V(\hat{\mathbf{r}}), \;\;\; \hat{H}_0 = \frac{\hat{\mathbf{p}}^2}{2m}.$$
Here, $\hat{H}_0$ is the kinetic part of the Hamiltonian, $m$ is the
particle mass, $\hat{\mathbf{r}}$ and $\hat{\mathbf{p}}$ are the
position and momentum operators, and $V$ is a \textbf{scattering
  potential} describing how the scatterer affects the quantum
particle.  We assume that the scatterer is finite, so that
$V(\mathbf{r}) \rightarrow 0$ as $|\mathbf{r}| \rightarrow \infty$.

[Fig.]

We want to prepare an incoming particle state with energy $E$, and see
how the particle is scattered by the potential.  However, converting
these words into a well-defined mathematical problem is a bit tricky!
We will give the formulation first, before discussing its meaning:
\begin{enumerate}
\item 
The particle is described by a state $|\psi\rangle$ obeying the
time-independent Schr\"odinger equation:
$$\hat{H} |\psi\rangle = E |\psi\rangle,$$
where $E$ is the same as the incoming particle energy.  

\item
This state can be decomposed into two terms,
$$|\psi\rangle \,=\, |\psi_i\rangle \,+\, |\psi_s\rangle,$$
where $|\psi_i\rangle$ is called the \textbf{incident state} and
$|\psi_s\rangle$ is called the \textbf{scattered state}.  

\item
The incident state is an eigenstate of $\hat{H}_0$ with energy $E$:
$$\hat{H}_0 |\psi_i\rangle = E |\psi_i\rangle.$$

\item
Lastly, we require the scattered state to be an ``outgoing'' state.
This is the trickiest bit, and we will describe how to deal with this
requirement later.
\end{enumerate}
The first of the above conditions says that the scattering process is
elastic; since the scatterer takes the form of a potential
$V(\mathbf{r})$, its interaction with the particle is conservative
(i.e., the total energy $E$ is fixed).  The second condition, when
viewed in the position basis, says that the particle's wavefunction
$\psi(\mathbf{r}) = \langle \mathbf{r} |\psi\rangle$ can be taken as a
superposition of an incoming wave and a scattered wave.  The third
condition tells us that far away from the scatterer
($|\mathbf{r}|\rightarrow \infty$), the wavelength of the incident
wave is fixed, consistent with the specified energy $E$.  Finally, the
scattered part of the wavefunction must correspond to a wave moving
outwards from the scatterer.

Given $|\psi_i\rangle$ (and hence $E$) along with $V(\mathbf{r})$, we
are to solve for $|\psi_s\rangle$.  Note that \textit{this is not an
  eigenproblem}!  Usually, when dealing with the time-independent
Schr\"odinger equation, we solve for the energy eigenvalues and
eigenstates.  But here, $E$ is not an output of the calculation, but
part of the input---it describes the energy of the incoming particles
in the experiment.

\subsection{Recap: position and momentum states}

Before proceeding, let us review the properties of quantum particles
in free space.

In a $d$-dimensional space, a coordinate vector $\mathbf{r}$ is a real
vector of $d$ components.  A quantum particle can be described by the
position basis---a set of quantum states $\{|\mathbf{r}\rangle\}$, one
for each possible $\mathbf{r}$.  (The $\mathbf{r}$'s form a continuum,
and so, like the real numbers, this set of states is uncountably
infinite.)  For a particle trapped in a finite region (e.g., a
particle in a box), we only consider $\mathbf{r}$ in that region;
otherwise, there is no restriction on $\mathbf{r}$.  We assume that
the appropriate set of $\{|\mathbf{r}\rangle\}$ spans the quantum
state space; thus, the identity operator can be resolved as
$$\hat{I} = \int d^dr \, |\mathbf{r}\rangle \,\langle\mathbf{r}|,$$
where the integral is taken over the allowed $\mathbf{r}$'s (a finite
region for a trapped particle, or infinite space otherwise).  It
follows that the position eigenstates are delta-function normalized:
$$\langle \mathbf{r} | \mathbf{r}' \rangle = \delta^d(\mathbf{r}-\mathbf{r}').$$
Note that $\delta^d(\cdots)$ denotes the $d$-dimensional delta
function.  For example, in 2D,
$$\langle x,y \,|\, x',y' \rangle = \delta(x-x') \, \delta(y-y').$$
The position operator $\hat{\mathbf{r}}$ is the operator that has
$\{|\mathbf{r}\rangle\}$ as its eigenstates:
$$\hat{\mathbf{r}} |\mathbf{r}\rangle \,=\, \mathbf{r}\, |\mathbf{r}\rangle.$$

Momentum eigenstates can be constructed from position eigenstates via
Fourier transforms.  First, suppose that the allowed region of space
is a box of finite length $L$ on each side, and periodic boundary
conditions in every direction.  Define the set of wave-vectors
$\mathbf{k}$ corresponding to plane waves that satisfy the periodic
boundary conditions at the boundaries of the box:
$$\Big\{\mathbf{k}  \; \Big| \; k_j = 2\pi m/L, \,m\in\mathbb{Z}, \;\text{for each} \; j = 1, \dots,d\, \Big\}.$$
Note that so long as $L$ is finite, the $\mathbf{k}$ vectors form a
discrete set.  Next, define
$$|\mathbf{k}\rangle = \frac{1}{L^{d/2}} \, \int d^dr \; e^{i\mathbf{k}\cdot\mathbf{r}} |\mathbf{r}\rangle,$$
where the integral is taken over the box.  These states
can be shown to satisfy the following:
$$\langle\mathbf{k}|\mathbf{k}'\rangle = \delta_{\mathbf{k},\mathbf{k}'}, \quad \langle\mathbf{r}|\mathbf{k}'\rangle = \frac{1}{L^{d/2}} e^{i\mathbf{k}\cdot\mathbf{r}}, \quad I = \sum_{\mathbf{k}} |\mathbf{k}\rangle\,\langle\mathbf{k}|.$$
The momentum operator is defined as an operator that has
$\{|\mathbf{k}\rangle\}$ as its eigenstates:
$$\hat{\mathbf{p}} |\mathbf{k}\rangle \,=\, \hbar \mathbf{k}\, |\mathbf{k}\rangle.$$
For finite $L$, momentum takes on quantized values; the momentum in
each direction is an integer multiple of $\Delta p = 2\pi\hbar/L$.

Now, take the limit of an infinite box.  The set of $\mathbf{k}$
vectors gets denser with increasing $L$, and in the $L \rightarrow
\infty$ limit they merge into a continuum.  In approaching this limit,
it is conventional to alter the normalization of the momentum states
as follows:
$$|\mathbf{k}\rangle \rightarrow \left(\frac{L}{2\pi}\right)^{d/2} |\mathbf{k}\rangle.$$
Then, in the $L\rightarrow\infty$ limit, the momentum states will
satisfy
$$\boxed{\begin{aligned} |\mathbf{k}\rangle &= \frac{1}{(2\pi)^{d/2}} \, \int d^dr \; e^{i\mathbf{k}\cdot\mathbf{r}} |\mathbf{r}\rangle, \\ |\mathbf{r}\rangle &= \frac{1}{(2\pi)^{d/2}} \, \int d^dk \; e^{-i\mathbf{k}\cdot\mathbf{r}} |\mathbf{k}\rangle, \\\langle\mathbf{k}|\mathbf{k}'\rangle = \delta^d(\mathbf{k}-\mathbf{k}'),& \quad \langle\mathbf{r}|\mathbf{k}\rangle = \frac{1}{(2\pi)^{d/2}} e^{i\mathbf{k}\cdot\mathbf{r}}, \quad I = \int d^dk \;|\mathbf{k}\rangle\,\langle\mathbf{k}|,\end{aligned}}$$
with the integrals now taken over infinite spaces.  In deriving
the above, it is helpful to use the formula
$$\int_{-\infty}^\infty dx\; \exp(ikx) \;=\; 2\pi\delta(k).$$

For an arbitrary quantum state $|\psi\rangle$, a wavefunction is
defined as the projection onto the position basis: $\psi(\mathbf{r}) =
\langle \mathbf{r}|\psi\rangle$.  Using the above equations, we can
show that
$$\begin{aligned}\langle \mathbf{r}|\hat{\mathbf{p}}|\psi\rangle &=  \int d^dk \; \langle\mathbf{r}|\mathbf{k}\rangle \; \hbar\mathbf{k} \; \langle\mathbf{k}|\psi\rangle \\ &=  \int \frac{d^dk}{(2\pi)^{d/2}}\; \hbar\mathbf{k} \;e^{i\mathbf{k}\cdot\mathbf{r}} \langle\mathbf{k}|\psi\rangle \\ &=  -i\hbar\nabla \int \frac{d^dk}{(2\pi)^{d/2}}\; \;e^{i\mathbf{k}\cdot\mathbf{r}} \langle\mathbf{k}|\psi\rangle \\ &= -i\hbar \nabla\psi(\mathbf{r}).\end{aligned}$$
This result can also be used to prove Heisenberg's commutation relation
$[\hat{r}_i, \hat{p}_j] = i\hbar\delta_{ij}$.

[Discuss normalization]

\subsection{Scattering from a 1D delta-function potential}

We are now ready to solve our first scattering problem.  Consider a
one-dimensional space with spatial coordinate denoted by $x$, and a
scattering potential that consists of a ``spike'' at $x = 0$:
$$V(x) = \frac{\hbar^2\gamma}{2m} \,\delta(x).$$
The form of the prefactor $\hbar^2\gamma/2m$ is chosen for later
convenience; the parameter $\gamma$, which has units of $[1/x]$,
controls the strength of the scattering potential.

You might be disturbed by the idea of a delta function potential, as
it is singular.  It helps to think of the delta function as a limiting
case of a family of non-singular functions (e.g., increasingly tall
and narrow gaussians centered at $x=0$).  When the Schr\"odinger wave
equation has a non-singular potential, any solution $\psi(x)$ must be
continuous and have well-defined first and second derivatives.  In the
delta function limit, the second derivative of $\psi(x)$ at $x=0$
blows up; moreover, the first derivative is discontinuous but finite,
while $\psi(x)$ itself remains continuous.  To verify this, integrate
the Schr\"odinger wave equation over an infinitesimal range around $x
= 0$:
$$\begin{aligned}\lim_{\varepsilon\rightarrow 0^+} \int_{-\varepsilon}^{+\varepsilon} dx\; \left[-\frac{\hbar^2}{2m} \frac{d^2}{dx^2} + \frac{\hbar^2\gamma}{2m} \delta(x)\right] \psi(x) &= \lim_{\varepsilon\rightarrow 0^+} \int_{-\varepsilon}^{+\varepsilon} dx\; E \psi(x) \\ = \lim_{\varepsilon\rightarrow 0^+} \left\{-\frac{\hbar^2}{2m} \left[\frac{d\psi}{dx}\right]_{-\varepsilon}^{+\varepsilon} \right\} + \frac{\hbar^2\gamma}{2m} \psi(0) &= 0\\ \Rightarrow \;\; \lim_{\varepsilon\rightarrow 0^+} \left\{\; \left.\frac{d\psi}{dx}\right|_{x = +\varepsilon} - \left.\frac{d\psi}{dx}\right|_{x = -\varepsilon}\; \right\}  &=  \gamma \,\psi(0).\end{aligned}$$

To proceed, we consider a particle incident from the left, with
energy $E$.  This can be described by an incident state proportional
to a momentum eigenstate $|k_i\rangle$, where $k_i > 0$ and $E =
\hbar^2k_i^2/2m$.  We said ``proportional'', not ``equal'', as it
is conventional to normalize the incident state as follows:
$$|\psi_i\rangle = \sqrt{2\pi}\Psi_i |k_i\rangle \;\;\; \Leftrightarrow\;\;\; \psi_i(x) = \langle x|\psi\rangle = \Psi_i \, e^{ik_i x}.$$
The constant $\Psi_i$ is called the ``incident amplitude''; its
physical meaning will be discussed later.  Plug this into the
Schr\"odinger wave equation:
$$\left[-\frac{\hbar^2}{2m} \frac{d^2}{dx^2} + \frac{\hbar^2\gamma}{2m}\delta(x)\right] \left(\Psi_i \, e^{ikx} + \psi_s(x) \right) = E \left(\Psi_i \, e^{ikx} + \psi_s(x) \right)$$
Taking $E = \hbar^2k_i^2/2m$, and doing a bit of algebra, simplifies this to
$$\left[ \frac{d^2}{dx^2} + k_i^2\right] \psi_s(x) =  \gamma \delta(x) \left(\Psi_i \, e^{ikx} + \psi_s(x) \right).$$
This has the form of an inhomogenous ordinary differential equation (ODE)
for $\psi_s(x)$, with the potential term on the right acting
as a ``driving term''.  To find the solution, consider the two regions $x <
0$ and $x > 0$.  Since $\delta(x) \rightarrow 0$ for $x \ne 0$, the
equation in each half-space reduces to
$$\left[\frac{d^2}{dx^2} + k_i^2\right] \psi_s(x) = 0.$$
This second-order ODE is called the Helmholtz equation.  Its general
solution can be written as
$$\psi_s(x) = \Psi_i \left(f_1 e^{ik_i x} + f_2 e^{-ik_i x}\right),$$
where $f_1$ and $f_2$ are arbitrary complex constants.  These
constants can have different values in the two regions $x < 0$ and $x
> 0$.

We want $\psi_s(x)$ to describe an \textbf{outgoing wave}, moving away
from the scatterer towards infinity.  So, it should be purely
left-moving for $x < 0$, and purely right-moving for $x > 0$.  To
achieve this, we let $f_1 = 0$ for $x < 0$, and $f_2 = 0$ for $x > 0$,
so that $\psi_s(x)$ has the form
$$\psi_s(x) = \Psi_i \times \begin{cases}f_- \,e^{-ik_ix}, & x < 0 \\ f_+ \,e^{ik_ix}, & x > 0.\end{cases}$$
The complex numbers $f_-$ and $f_+$ are called \textbf{scattering
  amplitudes}.  They describe the magnitude and phase of the
wavefunction scattered backwards into the $x<0$ region, and scattered
forward into the $x > 0$ region, respectively.

To calculate the scattering amplitudes, we first recall from earlier
that the total wavefunction $\psi(x)$ must be continuous everywhere,
including at $x = 0$.  Since $\psi_i(x)$ is continuous, $\psi_s(x)$
must be as well, which means that $f_- = f_+$.

Moreover, we previously found that the wavefunction's first
derivative is discontinuous at $x=0$:
$$\lim_{\varepsilon\rightarrow 0^+} \left\{\; \left.\frac{d\psi}{dx}\right|_{x = +\varepsilon} - \left.\frac{d\psi}{dx}\right|_{x = -\varepsilon}\; \right\}  =  \gamma\, \psi(0).$$
Plugging in our expression for $\psi(x)$ gives
$$\begin{aligned}\Psi_i\left[ik_i(1+f_\pm) - ik_i(1-f_\pm)\right]  &=  \Psi_i(1+f_\pm) \gamma \\ \Rightarrow \;\;\; f_+ = f_- = -\frac{\gamma}{\gamma - 2ik_i}.\end{aligned}$$

As we shall see, both the magnitude and phase of the scattering
amplitude contain important information.  For now, we focus on the
former.  The quantity $|f_\pm|^2$ describes the overall strength of
the scattering process:
$$|f_\pm|^2 = \left[1 + \frac{8mE}{(\hbar\gamma)^2}\right]^{-1}.$$
Hence, for a fixed potential strength $\gamma$, the scattering
decreases with $E$---in other words, high-energy particles are less
easily scattered than low-energy particles.  For a given $E$, the
scattering increases with $|\gamma|$, approaching the limit $|f|^2
\rightarrow 1$ as $|\gamma|\rightarrow \infty$.  Note that an
attractive potential ($\gamma < 0$) and a repulsive potential ($\gamma
> 0$) are equally effective at scattering the particle.

\subsection{Scattering in 2D and 3D}

We are now ready to describe scattering experiments in spatial
dimension $d \ge 2$.  Compared to the example from the previous
section, the $d \ge 2$ cases have an important distinguishing feature:
for $d = 1$, the quantum particle can only scatter forward or
backward, but for $d \ge 2$ it can scatter in a variety of directions
sideways.  Thus, the concept of an ``outgoing'' wavefunction
$\psi_s(\mathbf{r})$ must be formulated with care.

Far away from the scatterer, where $V(\mathbf{r})\rightarrow 0$, we
know that the scattered wavefunction satisfies the free-space
Schr\"odinger wave equation:
$$-\frac{\hbar^2}{2m} \nabla^2 \psi_s(\mathbf{r}) = E \psi_s(\mathbf{r}).$$
Here, $\nabla^2$ denotes the $d$-dimensional Laplacian.  Let $E =
\hbar^2 K^2 / 2m$, where $K \in \mathbf{R}^+$ is the wavenumber in
free space.  Then the above equation can be written as
$$\left[\nabla^2 + K^2\right] \psi_s(\mathbf{r}) = 0,$$
which is the Helmholtz equation in $d$-dimensional space.

One set of elementary solutions to the Helmholtz equation are the
plane waves, $\exp(i\mathbf{k}\cdot\mathbf{r})$, where $|\mathbf{k}| =
K$.  But plane waves don't capture the concept of an ``outgoing'' wave
that we need.  So, instead, we use a set of solutions expressed using
curvilinear coordinates.

In 2D, we can adopt polar coordinates $(r,\phi)$, and look for
separable solutions to the Helmholtz equation of the form
$A(r)B(\phi)$.  Skipping the math details, the end-result is that the
general solution to the 2D Helnmholtz equation can be written as
$$\psi(\mathbf{r})=\sum_{\pm}\sum_{m=-\infty}^\infty c_m^\pm\,H_m^\pm(Kr)\,e^{im\phi}.$$
Each $c_m^\pm$ is a complex coefficient, and $H_m^\pm$ is a special
function called a Hankel function of the first kind ($+$) or second
kind ($-$).  Each term in the sum describes a wave component that has a
definite angular momentum, and is moving either toward or away from the
origin.  The integer index $m$ specifies the angular momentum, while
$\pm$ denotes whether it is outgoing ($+$) or incoming ($-$); this is
because, far from the origin, the Hankel functions reduce to
$$H_m^\pm(Kr) \overset{r\rightarrow\infty}{\longrightarrow} \sqrt{\frac{2}{\pi Kr}} \, \exp\left[\pm i\left(kr - \frac{(m+\frac{1}{2})\pi}{2}\right)\right] \;\sim\; r^{-1/2} e^{ikr}.$$

In 3D, we use spherical coordinates $(r,\theta,\phi)$.  Again skipping
the math details, the general solution to the Helmholtz equation can
be written as
$$\psi(\mathbf{r})=\sum_{\pm}\sum_{l=0}^\infty\sum_{m=-l}^lc_{lm}^\pm\,h_l^\pm(Kr)\,Y_{lm}(\theta,\phi).$$
The $c$'s are again complex coefficients, each $h_l^\pm$ is a
spherical Hankel functions, and each $Y_{lm}$ is a spherical harmonic.
The $l$ and $m$ indices specify the angular momentum of each wave
component, while $\pm$ denotes if it is outgoing ($+$) or incoming
($-$); far from the origin,
$$h_l^\pm(Kr) \overset{r\rightarrow\infty}{\longrightarrow} \frac{1}{Kr}\,\exp\left[\pm i\left(kr-\frac{(l+1)\pi}{2}\right)\right] \;\sim\; r^{-1} e^{ikr}.$$

From this discussion, it is evident that we can make
$\psi_s(\mathbf{r})$ a purely ``outgoing'' wave by discarding the
incoming ($-$) wave components, keeping only the $+$ terms of the
general solution:
$$\psi_s(\mathbf{r}) = \begin{cases} \displaystyle\sum_{m} c_m^+\,H_m^\pm(Kr)\,e^{im\phi}, &d=2\\ \displaystyle\sum_{lm} c_{lm}^+\,h_l^+(Kr)\,Y_{lm}(\theta,\phi),&d=3.\end{cases}$$
For large $r$, the outgoing wavefunction's $r$-dependence can be
written as follows:
$$\psi_s(\mathbf{r}) \; \overset{r\rightarrow\infty}{\sim} \; r^{\frac{1-d}{2}} e^{ikr}.$$

For $d > 1$, the magnitude of the wavefunction decreases with distance
from the origin.  This is to be expected, because with increasing $r$
an outgoing wave spreads out over a wider area.  To verify this
intuition, consider the probability current density
$\mathbf{J} = (\hbar/m) \mathrm{Im}\left[\psi_s^*\nabla\psi_s\right]$;
its $r$-component is
$$\begin{aligned}J_r \; &\overset{r\rightarrow\infty}{\sim} \; \mathrm{Im}\left[r^{\frac{1-d}{2}} e^{-ikr} \frac{\partial}{\partial r}\left(r^{\frac{1-d}{2}} e^{ikr}\right)\right] \\ &\;\;=\;\;\;\mathrm{Im}\left[\frac{1-d}{2}\, r^{-d} + ik r^{1-d}\right]\\ &\;\;=\;\;\; kr^{1-d}.\end{aligned}$$
In $d$ dimensions, the ``area'' of a ``spherical'' wave-front scales
as $r^{d-1}$, so the probability flux goes as $J_r \,r^{d-1} \sim k$,
independent of $r$.  Thus, the outgoing solution describes a constant
probability flux, flowing outward from the origin.  Note also that the
results for $d=1$ are consistent with our findings from the previous
section.  In 1D, waves do not spread out with distance since there is
no transverse dimension; accordingly, $\psi_s$ and $J_r$ are
independent of the distance traveled.

\subsection{The scattering amplitude and scattering cross section}

The results of the previous section show us how to systematically
characterize the outcome of a scattering experiment.  Let the incident
wavefunction be a plane wave
$$\psi_i(\mathbf{r}) = \Psi_i \, e^{i\mathbf{k}_i\cdot\mathbf{r}},$$
where $\Psi_i \in \mathbb{C}$ is the \textbf{incident wave amplitude},
and $\mathbf{k}_i$ is the incident momentum, which is tied to the
particle energy $E = \hbar^2|\mathbf{k}_i|^2/2m$.  We adopt
coordinates $(r,\Omega)$, where $r$ is the distance from the origin.
For 1D, $\Omega \in \pm$, specifying the choice of ``forward'' or
``backward'' scattering; for 2D polar coordinates, $\Omega = \phi$;
and for 3D spherical coordinates, $\Omega = (\theta,\phi)$.  Far from
the origin, the scattered wavefunction has the form
$$\psi_s(\mathbf{r})\;  \overset{r\rightarrow\infty}{\longrightarrow}\; \Psi_i \, r^{\frac{1-d}{2}} \, e^{i|\mathbf{k}_i|r} \, f(\Omega).$$
The complex function $f(\Omega)$, called the \textbf{scattering
  amplitude}, is the fundamental object of interest in scattering
experiments.  It describes how the particle gets scattered in various
directions.  It is dependent on the inputs to the scattering problem,
including $\mathbf{k}_i$ and the scattering potential.  From it, we
can also define two other important quantities of interest:
$$\boxed{\begin{aligned}\frac{d\sigma}{d\Omega} &= \big|f(\Omega)\big|^2 \qquad\quad\quad \text{(the \textbf{differential scattering cross section})} \\ \sigma &= \int d\Omega\; \big|f(\Omega)\big|^2 \quad\; \text{(the \textbf{total scattering cross section}).}
  \end{aligned}}$$
In the second equation, $\int d\Omega$ denotes the integral(s) over
all the angle coordinates (or, in 1D, a sum over the two directions
$\Omega \in \pm$).

The term ``cross section'' comes from an analogy with the scattering
of classical particles.  To understand this, consider the probablity
current density associated with the scattered wavefunction:
$$\mathbf{J}_s = \frac{\hbar}{m} \mathrm{Im}\big[\psi_s^*\nabla\psi_s\big].$$
Let us focus only on the $r$-component of the current density, in the
$r\rightarrow\infty$ limit:
$$\begin{aligned}J_{s,r}\; &\;\;=\;\;\;\, \frac{\hbar}{m}\, \mathrm{Im}\left[\psi_s^* \frac{\partial}{\partial r}\psi_s\right] \\ &\overset{r\rightarrow\infty}{\longrightarrow}\;\, \frac{\hbar}{m}\; |\Psi_i|^2 \; |f(\Omega)|^2 \mathrm{Im}\left[\left(r^{\frac{1-d}{2}} \, e^{i|\mathbf{k}_i|r}\right)^* \frac{\partial}{\partial r} \left(r^{\frac{1-d}{2}} \, e^{i|\mathbf{k}_i|r}\right)\right] \\ & \;\;=\;\;\;\, \frac{\hbar \,|\mathbf{k}_i|}{m}\; |\Psi_i|^2 \; |f(\Omega)|^2 \,r^{1-d}.\end{aligned}$$
The total flux of outgoing probability is obtained by integrating
$J_{s,r}$ over a constant-$r$ surface:
$$I_s \;=\; \int d\Omega\; r^{d-1} \, J_{s,r} \;=\; \frac{\hbar|\mathbf{k}_i|}{m} \;|\Psi_i|^2 \; \int d\Omega\; \big|f(\Omega)\big|^2.$$
We can assign a neat physical interpretation to each term in this
result.  The first factor, $\hbar|\mathbf{k}_i|/m$, is the
particle's speed (i.e., the group velocity of the de Broglie wave).
The second factor, $|\Psi_i|^2$, is the probability density of the incident
wave, which has units of $[x^{-d}]$ (i.e., inverse $d$-dimensional
``volume'').  The product of these two factors represents the
\textbf{incident flux},
$$J_{i} = \frac{\hbar|\mathbf{k}_i|}{m} \;|\Psi_i|^2.$$
This has units of $[x^{1-d}t^{-1}]$ (i.e., rate per unit ``area'').
Now, imagine a stream of classical particles with incident flux $J_i$,
and a ``hard-body'' scatterer that only interacts with those particles
striking it directly.  The scattering rate will be $J_{i}$ times the
cross-sectional area of the scatterer exposed to the particle stream.
If we denote this area by $\sigma$, the scattering rate will be
$$I_s = J_i \, \sigma.$$
Comparing the quantum scattering expression to this, we identify $\int
d\Omega\;|f|^2$ as playing a role analogous to the hard-body's
cross-sectional area.  We hence call this the ``total scattering cross
section''.  Moreover, this quantity is expressed as an integral over
angles; the integrand $|f|^2$ is called the ``differential scattering
cross section'', and it represents the rate, per unit of solid
angle, at which particles are scattered in a given direction.


\end{document}


%% For decades after the discovery of quantum mechanics, the quantum
%% double-slit experiment was just a ``thought experiment'', meant to
%% illustrate the features of quantum mechanics that had been uncovered
%% by other, more complicated experiments.  Nowadays, the most convenient
%% way to do the experiment is with light, using single-photon sources
%% and single-photon detectors.  Quantum interference has also been
%% demonstrated experimentally using electrons, neutrons, and even
%% large-scale particles such as buckyballs.
