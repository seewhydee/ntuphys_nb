\documentclass[pra,12pt]{revtex4}
\usepackage{amsmath}
\usepackage{amssymb}
\usepackage{graphicx}
\usepackage{color}
\usepackage[pdfborder={0 0 0},colorlinks=true,linkcolor=blue,urlcolor=blue]{hyperref}

\def\ket#1{\left|#1\right\rangle}
\def\bra#1{\left\langle#1\right|}
\def\braket#1{\left\langle#1\right\rangle}

\setlength{\parindent}{0pt}

\renewcommand{\baselinestretch}{1.0}
\setlength{\parskip}{0.07in}

\begin{document}

\begin{center}
{\large \textbf{Appendix D: Numerical Tensor Products}}
\end{center}

As described in Chapter 2, multi-particle quantum states are described
mathematically using tensor products.  This appendix describes these
tensor product states, and the operators acting upon them, are handled
in numerical linear algebra software.  Our examples will be given in
terms of \href{https://scipy.org/}{Scientific Python}, but the
underlying concepts carry over to other programming languages like GNU
Octave or Matlab.

Tensor products of vectors (states) and matrices (operators) are
implemented using the
\href{https://docs.scipy.org/doc/numpy-1.13.0/reference/generated/numpy.kron.html}{\texttt{numpy.kron}}
function.  (GNU Octave and Matlab also has a \texttt{kron} function
which does almost exactly the same thing.)


\end{document}
