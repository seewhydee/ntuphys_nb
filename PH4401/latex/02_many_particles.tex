\documentclass[pra,11pt]{revtex4}
\usepackage{amsmath}
\usepackage{amssymb}
\usepackage{graphicx}
\usepackage{color}
\usepackage{mathrsfs}

\def\ket#1{\left|#1\right\rangle}
\def\bra#1{\left\langle#1\right|}
\def\braket#1{\left\langle#1\right\rangle}

\setlength{\parindent}{0pt}

\renewcommand{\baselinestretch}{1.0}
\setlength{\parskip}{0.07in}

\begin{document}

\section{Entangled quantum states}

So far, we have studied quantum mechanical systems consisting of
single particles.  The next important step is to look at systems
containing more than one particle.  As we shall see, the postulates of
quantum mechanics have interesting implications which only show up in
multi-particle systems, such as the phenomenon of quantum
entanglement.

In the following discussion, you may assume that we are dealing with
systems of ``distinguishable'' particles.  A simple example of such a
system is a two-particle system containing different types of
particles, like a proton and a neutron.  There is a separate set of
considerations to take into account whenever we're dealing with
systems of ``identical'' particles, such as a system composed of two
electrons, where the electrons cannot be distinguished from each other
even in principle.  We will spend the next chapter dealing with those
special complications.  If you're not sure what this paragraph is
talking about, don't worry; just read on.

We begin by considering a system of two particles, labeled $A$ and
$B$.  If each individual particle is treated as a quantum system, then
according to the usual postulates of quantum mechanics, its state is
described by a vector in some complex Hilbert space.  Let
$\mathscr{H}_A$ and $\mathscr{H}_B$ denote the respective Hilbert
spaces for these two particles.  When the two particles are considered
as a single quantum system, that system has its own Hilbert space,
denoted by
$$\mathscr{H} = \mathscr{H}_A\otimes \mathscr{H}_B.$$
The symbol $\otimes$ refers to a mathematical operation known as the
\textbf{tensor product}, which is a way of combining two Hilbert
spaces to form a larger Hilbert space.

The mathematical meaning of the tensor product is as follows.  Suppose
$\mathscr{H}_A$ can be spanned by an orthonormal basis
$\{|\mu_1\rangle, |\mu_2\rangle, |\mu_3\rangle, \dots\}$, and
$\mathscr{H}_B$ can be spanned by $\{|\nu_1\rangle, |\nu_2\rangle,
|\nu_3\rangle, \dots\}$.  Then the tensor product space $\mathscr{H}_A
\otimes \mathscr{H}_B$ can be defined in terms of the basis set
$$\Big\{\;\,|\mu_i\rangle\otimes|\nu_j\rangle \;\;  \Big| \;\; \textrm{all}\;|\mu_i\rangle,\; |\nu_j\rangle \;\,\Big\}.$$
Intuitively, this means that if you know particle $A$ is in state
$|\mu_i\rangle$, and particle $B$ is in state $|\nu_j\rangle$, that
specifies a state of the combined system.  A complete basis set for
the combined system can be constructed in this way.  Note that if
$\mathscr{H}_A$ has dimension $d_A$ and $\mathscr{H}_B$ has dimension
$d_B$, then $\mathscr{H}_A \otimes \mathscr{H}_B$ has dimension $d_A
d_B$.

Any two-particle state can then be written as
$$|\psi\rangle = \sum_{i} \sum_{j} \, c_{ij}\; |\mu_i\rangle \otimes |\nu_j\rangle.$$
The inner product for tensor product basis states is defined in terms
of the inner product of the individual spaces:
$$\Big(|\mu_i\rangle \otimes |\nu_j\rangle\;,\; |\mu_p\rangle \otimes |\nu_q\rangle \Big) \;\equiv\; \big(\langle\mu_i| \otimes \langle\nu_j| \big) \big(|\mu_p\rangle \otimes |\nu_q\rangle\big) \;\equiv\; \langle\mu_i|\mu_p\rangle \, \langle\nu_j|\nu_q\rangle.$$
In other words, if you need to multiply a tensor product bra with a
tensor product ket, the procedure is to (i) do the bra-ket product for
the first slot, (ii) do the bra-ket product for the second slot, and
(iii) multiply the results together.

As an example, suppose that both $\mathscr{H}_A$ and $\mathscr{H}_B$
are two-dimensional Hilbert spaces describing spin-$\frac{1}{2}$
degrees of freedom, such that each can be spanned by an orthonormal basis
$\{\,|\!\uparrow\rangle, \,|\downarrow\rangle \, \}$.  Then the tensor
product space $\mathscr{H}$ is spanned by
$$\Big\{\;|\uparrow\rangle\otimes|\uparrow\rangle\,,\; |\uparrow\rangle\otimes|\downarrow\rangle\,,\; |\downarrow\rangle\otimes|\uparrow\rangle\,,\; |\downarrow\rangle\otimes|\downarrow\rangle \;\Big\}.$$
The dimension of $\mathscr{H}$ is $2 \times 2 = 4$.  

At this point, we must make an extremely crucial observation.  As
mentioned, if we specify that particle $A$ is in state $|\mu_i\rangle$
and particle $B$ is in state $|\nu_j\rangle$, that specifies a state
of the combined system.  But the reverse is not true!  \textit{Some
  states of the combined system cannot be expressed by simply stating
  the individual particle states.}  For example, take the previous
example of two spaces describing spin-$\frac{1}{2}$ degrees of
freedom, and consider the two-particle state
$$|\psi\rangle = \frac{1}{\sqrt{2}} \Big(|\uparrow\rangle\otimes|\downarrow\rangle \,-\, |\downarrow\rangle\otimes|\uparrow\rangle\Big).$$

\section{The Einstein-Podolsky-Rosen ``paradox''}


\section{Bell's inequalities}

\section{Entanglement as a resource: quantum cryptography}

\section{The Many-Worlds Interpretation}

\section{Exercises}

%% \begin{enumerate}
%% \item Phase shift under scattering resonance

%% \item Classical driven oscillator analogy  
%% \end{enumerate}




\section{Further Reading}

%% \begin{itemize}
%% \item Bransden \& Joachain, \S4.4, 9.2--9.3, 13.4
%% \item Sakurai, \S5.6, 7.7--7.8

%% \end{itemize}


\end{document}


%% For decades after the discovery of quantum mechanics, the quantum
%% double-slit experiment was just a ``thought experiment'', meant to
%% illustrate the features of quantum mechanics that had been uncovered
%% by other, more complicated experiments.  Nowadays, the most convenient
%% way to do the experiment is with light, using single-photon sources
%% and single-photon detectors.  Quantum interference has also been
%% demonstrated experimentally using electrons, neutrons, and even
%% large-scale particles such as buckyballs.
