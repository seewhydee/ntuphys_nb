\documentclass[pra,12pt]{revtex4}
\usepackage{amsmath}
\usepackage{amssymb}
\usepackage{graphicx}
\usepackage{color}
\usepackage{mathrsfs}
\usepackage[pdfborder={0 0 0},colorlinks=true,linkcolor=blue]{hyperref}

\def\ket#1{\left|#1\right\rangle}
\def\bra#1{\left\langle#1\right|}
\def\braket#1{\left\langle#1\right\rangle}

\setlength{\parindent}{0pt}

\renewcommand{\baselinestretch}{1.0}
\setlength{\parskip}{0.07in}

\begin{document}

\section{Particle indistinguishability and exchange symmetry}

We saw, in the previous chapter, how the principles of quantum
mechanics apply to systems of multiple particles.  That discussion
omitted an important feature of multi-particle systems, namely the
fact that particles of the same type are fundamentally
indistinguishable from each other.  As it turns out,
indistinguishability is a strong constraint on the form of the
multi-particle quantum states.  Looking into this will ultimately lead
us toward a fundamental re-interpretation of what ``particles'' are.

Suppose we have two particles of the same type, such as two electrons.
It is a fact of Nature that all electrons have identical physical
properties: same mass, same charge, same total spin, etc.  One
consequence of this is that the single-particle Hilbert spaces of the
two electrons must be mathematically identical.  Let us denote this
space by $\mathscr{H}^{(1)}$.  For the combined two-electron system,
the Hilbert space is a tensor product of two single-electron Hilbert
spaces, denoted by
$$\mathscr{H}^{(2)} = \mathscr{H}^{(1)} \otimes \mathscr{H}^{(1)}.$$
Since the electrons have identical properties, any Hamiltonian for the
two-electron system must act on the electrons in a symmetrical way.
An example of such a Hamiltonian is
$$\hat{H} = \frac{1}{2m_e} \Big(|\hat{\mathbf{p}}_1|^2 + |\hat{\mathbf{p}}_2|^2\Big) + \frac{e^2}{4\pi\varepsilon_0|\hat{\mathbf{r}}_1 - \hat{\mathbf{r}}_2|},$$
which contains non-relativistic kinetic energies for the two
electrons, as well as the potential energy for the Coulomb interaction
between the electrons.  The operators $\hat{\mathbf{p}}_1$ and
$\hat{\mathbf{r}}_1$ act on electron 1, while $\hat{\mathbf{p}}_2$ and
$\hat{\mathbf{r}}_2$ act on electron 2.

This kind of two-electron Hamiltonian is evidently invariant under an
interchange of the operators acting on the two electrons (i.e.,
$\hat{\mathbf{p}}_1 \leftrightarrow \hat{\mathbf{p}}_2$ and
$\hat{\mathbf{r}}_1 \leftrightarrow \hat{\mathbf{r}}_2$).  This can be
regarded as a symmetry of the system, known as \textbf{exchange
  symmetry}.  In other contexts, we have seen how symmetries of
quantum systems can be represented by unitary (i.e., norm-conserving)
operators that commute with the Hamiltonian.  Exchange symmetry can
also be represented by an operator $\hat{P}$, defined as follows: let
$\{|\mu_i\rangle\}$ be a basis set for the single-electron Hilbert
space $\mathscr{H}^{(1)}$; then
$$P \Big (\sum_{ij} \psi_{ij} |\mu_i\rangle|\mu_j\rangle \Big)
\;\equiv\;  \sum_{ij} \psi_{ij} |\mu_j\rangle|\mu_i\rangle = \sum_{ij} \psi_{ji} |\mu_i\rangle|\mu_j\rangle.$$
Since two consecutive exchanges clearly leave the state unchanged,
$$\hat{P}^2 = \hat{I},$$
where $\hat{I}$ is the identity operator.  It can be shown that:
\begin{enumerate}
\item $\hat{P}$ is linear, unitary (as required for a symmetry
  operator), and Hermitian; see \hyperref[ex:1]{Exercise 1}.
  
\item The effect of $\hat{P}$ does not depend on the specific choice
  of basis; see \hyperref[ex:1]{Exercise 1}.

\item $\hat{P}$ commutes with the above Hamiltonian $\hat{H}$, or
  indeed any Hamiltonian where the single-particle operators appear in
  a symmetrical manner; see \hyperref[ex:2]{Exercise 2}.
\end{enumerate}

According to Noether's theorem, any symmetry implies a conservation
law.  In the case of exchange symmetry, since $\hat{P}$ is itself
Hermitian, we can take the conserved quantity to be its eigenvalue.
We call this eigenvalue, $p$, the \textbf{exchange parity}.  Given
that $\hat{P}^2 = \hat{I}$, there are just two possibilities:
$$\hat{P} |\psi\rangle = p|\psi\rangle \;\;\;\Rightarrow\;\;\; p = \begin{cases}+1 & \textrm{(``symmetric\;state'')}, \;\;\textrm{or} \\ -1 & \textrm{(``antisymmetric\;state'').}\end{cases}$$
Since $\hat{P}$ commutes with $\hat{H}$, if the system starts out in
one eigenstate of $\hat{P}$ with parity $p$, then it always retains
the same definite parity $p$ under time evolution via Schr\"odinger's
equation.

The concept of exchange parity generalizes to systems of more than two
particles.  In a system of $N$ particles, we can define a set of
exchange operators $\hat{P}_{ij}$, where $i,j\in\{1,2,\dots,N\}$ and
$i\ne j$, such that $\hat{P}_{ij}$ exchanges particle $i$ and particle
$j$.  If the particles are identical, the Hamiltonian must commute
with \textit{all} the exchange operators---i.e., the parities ($\pm
1$) are all separately conserved.

We now invoke the following postulates:
\begin{enumerate}
\item A multi-particle state of identical particles is an eigenstate
  of every exchange operator $\hat{P}_{ij}$.

\item For each $\hat{P}_{ij}$, the exchange parity $p_{ij}$ has the
  same value: i.e., all $+1$ or all $-1$.

\item The exchange parity depends \textit{uniquely} on the type of
  particle.
\end{enumerate}
These facts are \textit{not} a logical result of our discussion thus
far!  Instead, they are to be regarded as empirical facts of Nature.
Let us assume that they are true, and explore the consequences.
Later, in Section~\ref{sec:qft}, we will discuss why it may be
``natural'' for these facts to hold.

Particles that have symmetric states ($p_{ij} = +1$) are called
\textbf{bosons}.  The elementary particles that ``carry'' the
fundamental forces are all bosons: these are photons (the elementary
particles of light, which ``carry'' the electromagnetic force), gluons
(which carry the ``strong nuclear force'' that holds protons and
neutrons together), and $W$ and $Z$ bosons (which carry the ``weak
nuclear force'' that is responsible for beta decay).  Certain
composite particles, such as helium-4 nuclei (alpha particles), are
also bosons.

Particles that have antisymmetric states ($p_{ij} = -1$) are called
\textbf{fermions}.  The elementary particles of ``matter'' are all
fermions: these are electrons, muons, tauons, the various types of
quarks, and the various types of neutrinos, along with their
anti-particles (positrons, etc.).  Protons and neutrons are also
fermions, although these are not elementary particles, being each
composed of three quarks.  Certain other composite particles, such as
helium-3 nuclei, are also fermions.

We will discuss later, in Section~\ref{sec:spinstats}, what determines
if a given particle type is a fermion or a boson.  Before that, let us
undertake an examination of the special features possessed by
symmetric (``bosonic'') and antisymmetric (``fermionic'')
multi-particle states.

\section{Bosons}

A state of $N$ bosons must be symmetric under every possible exchange
operator:
$$\hat{P}_{ij}\; |\psi\rangle = |\psi\rangle \;\;\; \forall\, i, j \in\{1,\dots,N\},\;\; i\ne j.$$

There is a standard way to construct multi-particle states obeying
this symmetry condition, based on the ``occupancy'' of the
single-particle states.  To illustrate the procedure, consider a
two-boson system ($N = 2$).  Suppose both bosons occupy the same
single-particle state, $|\phi_a\rangle \in \mathscr{H}^{(1)}$; then
the two-boson state is simply
$$|\phi_a,\phi_a\rangle = |\phi_a\rangle  |\phi_a\rangle.$$
It is apparent that this satisfies $\hat{P}_{12}
|\phi_a,\phi_a\rangle = |\phi_a,\phi_a\rangle$, as required.

Now suppose the two bosons occupy \textit{different} single-particle
states, $|\phi_a\rangle$ and $|\phi_b\rangle$.  It would be wrong to
write the two-boson state as $|\phi_a\rangle |\phi_b\rangle$, because
the particles would not be symmetric under exchange.  Instead, we
construct the multi-particle state
$$|\phi_a,\phi_b\rangle = \frac{1}{\sqrt{2}} \Big( |\phi_a\rangle  |\phi_b\rangle + |\phi_b\rangle  |\phi_a\rangle\Big).$$
This has the appropriate exchange symmetry:
$$\begin{aligned}\hat{P}_{12}|\phi_a,\phi_b\rangle &= \frac{1}{\sqrt{2}} \Big( |\phi_b\rangle  |\phi_a\rangle + |\phi_a\rangle  |\phi_b\rangle\Big) \\ &= |\phi_a,\phi_b\rangle.\end{aligned}$$

This construction is generalizable to arbitrary $N$.  Denote the $N$
occupied single-particle states by
$$|\phi_1\rangle, \, |\phi_2\rangle, \, |\phi_3\rangle, \, \dots, |\phi_N\rangle.$$
Here, we label the single-particle states using numerical subscripts,
but the states may in actuality not be unique; for example, it might
be that $|\phi_1\rangle = |\phi_2\rangle = |\phi_a\rangle$, meaning
there is a single-particle state $|\phi_a\rangle$ occupied by two
particles.  In any case, the $N$-boson state can be written as
$$|\phi_1,\phi_2,\dots,\phi_N\rangle = \mathcal{N} \sum_p \Big(|\phi_{p(1)}\rangle  |\phi_{p(2)}\rangle  |\phi_{p(3)}\rangle  \cdots  |\phi_{p(N)}\rangle\Big).$$
Here, the prefactor $\mathcal{N}$ is a normalization constant that
we'll discuss later.  The sum is taken over each of the $N!$ distinct
permutations acting on the sequence $\{1,2,\dots,N\}$.  For each
permutation $p$, we let $p(j)$ denote the integer that $j$ is permuted
into.

To see that this $N$-particle state has the exchange symmetry we are
looking for, apply an arbitrary exchange operator $\hat{P}_{ij}$ to it:
$$\begin{aligned}\hat{P}_{ij}|\phi_1,\phi_2,\dots,\phi_N\rangle &= \mathcal{N} \sum_p \hat{P}_{ij} \Big(\cdots  |\phi_{p(i)}\rangle  \cdots  |\phi_{p(j)}\rangle\cdots\Big) \\&= \mathcal{N} \sum_p \Big(\cdots  |\phi_{p(j)}\rangle  \cdots  |\phi_{p(i)}\rangle\cdots\Big).\end{aligned}$$
In each term of the sum, the states $i$ and $j$ are interchanged.  But
since the sum runs through all distinct permutations of the states, it
is evident that the result is the same with or without the exchange,
sowe end up with $|\phi_1,\phi_2,\dots,\phi_N\rangle$.  The
mutli-particle state is therefore symmetric under every exchange
operation.

As another example, consider a three-boson system with two particles
in state $|\phi_a\rangle$, and one particle in state $|\phi_b\rangle$.
To express the three-particle state, define $\{|\phi_1\rangle,
|\phi_2\rangle, |\phi_3\rangle\}$ such that $|\phi_1\rangle =
|\phi_2\rangle = |\phi_a\rangle$ and $|\phi_3\rangle =
|\phi_b\rangle$.  Then
$$\begin{aligned}|\phi_1,\phi_2,\phi_3\rangle &= \mathcal{N} \Big( \;\;
|\phi_1\rangle|\phi_2\rangle|\phi_3\rangle +
|\phi_2\rangle|\phi_3\rangle|\phi_1\rangle +
|\phi_3\rangle|\phi_1\rangle|\phi_2\rangle \\ &\qquad\;\, +
|\phi_1\rangle|\phi_3\rangle|\phi_2\rangle +
|\phi_3\rangle|\phi_2\rangle|\phi_1\rangle +
|\phi_2\rangle|\phi_1\rangle|\phi_3\rangle\Big) \\
&= 2\mathcal{N} \Big(
|\phi_a\rangle|\phi_a\rangle|\phi_b\rangle +
|\phi_a\rangle|\phi_b\rangle|\phi_a\rangle +
|\phi_b\rangle|\phi_a\rangle|\phi_a\rangle\Big).
\end{aligned}$$

\textcolor{red}{[Discuss normalization]}

\section{Fermions}

A state of $N$ fermions must be antisymmetric under every possible
exchange operator:
$$\hat{P}_{ij}\; |\psi\rangle = -|\psi\rangle \;\;\; \forall\, i,j\in\{1,\dots,N\}, \; i\ne j.$$
Similar to the boson case, we can explicitly construct multi-fermion
states based on the occupancy of single-particle states.  Let us again
consider the $N=2$ case first.  If the fermions occupy states
$|\phi_a\rangle$ and $|\phi_b\rangle$, the appropriate two-particle
state is
$$|\phi_a,\phi_b\rangle = \frac{1}{\sqrt{2}} \Big(|\phi_a\rangle|\phi_b\rangle - |\phi_b\rangle|\phi_a\rangle\Big).$$
We can easily check that
$$\hat{P}_{12} |\phi_a,\phi_b\rangle = \frac{1}{\sqrt{2}} \Big(|\phi_b\rangle|\phi_a\rangle - |\phi_a\rangle|\phi_b\rangle\Big) = - |\phi_a,\phi_b\rangle.$$
This construction breaks down if we try letting $|\phi_a\rangle$ and
$|\phi_b\rangle$ be the same state: in that case, the two terms would
cancel each other out to give the zero vector, which is not a valid
quantum state (as states must be described by vectors of unit norm).
This leads to \textbf{Pauli's exclusion principle}, which states that
fermions cannot occupy the same quantum state.  Another way of stating
Pauli's principle is that any single-particle state is either
unoccupied, or occupied by exactly one fermion.

For general $N$, let the occupied single-particle states be
$\{|\phi_1\rangle, |\phi_2\rangle,\dots,|\phi_N\rangle\}$.  Then the
appropriate $N$-fermion state is
$$|\phi_1,\dots,\phi_N\rangle = \frac{1}{\sqrt{N!}} \sum_p s(p)\, |\phi_{p(1)}\rangle |\phi_{p(2)}\rangle \cdots |\phi_{p(N)}\rangle.$$
The $1/\sqrt{N!}$ prefactor is a normalization constant (which you can
verify for yourself).  Similar to the boson case, the expression
involves a sum over every permutation $p$ of the sequence
$\{1,2,\dots,N\}$.  However, each term in the sum now has a
coefficient $s(p)$, denoting the \textbf{parity} of permutation $p$.
The permutation's parity is defined as $+1$ if $p$ involves an even
number of exchanges starting from the sequence $\{1,2,\dots,N\}$, and
$-1$ if $p$ involves an odd number of exchanges.

A couple of examples will make these definitions clearer.  For $N=2$,
the sequence $\{1,2\}$ has $2! = 2$ permutations:
$$\begin{aligned}p_1 : \{1,2\} &\rightarrow \{1,2\}, \;\;\;s(p_1) = +1 \\ p_2 : \{1,2\} &\rightarrow \{2,1\}, \;\;\;s(p_2) = -1.\end{aligned}$$
For $N=3$, the sequence $\{1,2,3\}$ has $3!=6$ permutations:
$$\begin{aligned}
  p_1 : \{1,2,3\} &\rightarrow \{1,2,3\}, \;\;\;s(p_1) = +1 \\
  p_2 : \{1,2,3\} &\rightarrow \{2,1,3\}, \;\;\;s(p_2) = -1 \\
  p_3 : \{1,2,3\} &\rightarrow \{2,3,1\}, \;\;\;s(p_3) = +1 \\
  p_4 : \{1,2,3\} &\rightarrow \{3,2,1\}, \;\;\;s(p_4) = -1 \\
  p_5 : \{1,2,3\} &\rightarrow \{3,1,2\}, \;\;\;s(p_5) = +1 \\
  p_6 : \{1,2,3\} &\rightarrow \{1,3,2\}, \;\;\;s(p_6) = -1.\end{aligned}$$
The permutations can be generated by exchanging pairs of sequence
elements; each time we perform such an exchange, the sign of $s(p)$
is reversed.

We can now see why the above $N$-particle state describes fermions.
Let us apply $\hat{P}_{ij}$ to it:
$$\begin{aligned}\hat{P}_{ij}|\phi_1,\dots,\phi_N\rangle &= \frac{1}{\sqrt{N!}} \sum_p s(p)\, \hat{P}_{ij} \big[\cdots |\phi_{p(i)}\rangle \cdots |\phi_{p(j)}\rangle \cdots\big] \\&= \frac{1}{\sqrt{N!}} \sum_p s(p)\, \big[\cdots |\phi_{p(j)}\rangle \cdots |\phi_{p(i)}\rangle \cdots\big].\end{aligned}$$
In each term of the sum, the single-particle state for $p(i)$ and
$p(j)$ have exchanged places.  The resulting term must be an exact
match for one of the terms in the original sum in
$|\phi_1,\dots,\phi_N\rangle$ (since that sum runs over all possible
permutations), \textit{except} for one thing: the coefficient $s(p)$
must have an \textit{opposite} sign (since the two permutations are
related by a single exchange).  It then follows that
$\hat{P}_{ij}|\phi_1,\dots,\phi_N\rangle = -
|\phi_1,\dots,\phi_N\rangle$, as desired for fermions.

You can also check that if the occupied single-particle states are
non-unique, this construction breaks down since the terms in the sum
over $p$ cancel out pairwise.  This is the manifestation of the Pauli
exclusion principle for $N$ fermions: no two fermions are allowed to
occupy the same single-particle quantum state.



\section{Second quantization}

\section{Measurements and state collapse}

\section{Quantum field theory}
\label{sec:qft}

\section{The spin-statistics connection}
\label{sec:spinstats}

\section*{Exercises}

\begin{enumerate}
\item Consider a system of two identical particles.  Each
  single-particle Hilbert space $\mathscr{H}^{(1)}$ is spanned by a
  basis $\{|\mu_i\}$.  The exchange operator is defined on
  $\mathscr{H}^{(2)} = \mathscr{H}^{(1)} \otimes \mathscr{H}^{(1)}$ by
$$P \Big (\sum_{ij} \psi_{ij} |\mu_i\rangle|\mu_j\rangle \Big)
  \;\equiv\;  \sum_{ij} \psi_{ij} |\mu_j\rangle|\mu_i\rangle.$$
  Prove that $\hat{P}$ is linear, unitary, and Hermitian.  Moreover,
  prove that the operation is basis-independent: i.e., given any other
  basis $\{\nu_j\}$ that spans $\mathscr{H}^{(1)}$, it is likewise
  true that
$$P \Big (\sum_{ij} \varphi_{ij} |\nu_i\rangle|\nu_j\rangle \Big)
  \;=\;  \sum_{ij} \varphi_{ij} |\nu_j\rangle|\nu_i\rangle.$$
  \label{ex:1}

\item
  Prove that the exchange operator commutes with the Hamiltonian
$$\hat{H} = - \frac{\hbar^2}{2m_e} \Big(\nabla_1^2 + \nabla^2_2\Big) + \frac{e^2}{4\pi\varepsilon_0|\mathbf{r}_1 - \mathbf{r}_2|}.$$ \label{ex:2}
  
\end{enumerate}

\section*{Further Reading}

%% \begin{itemize}
%% \item
%% \end{itemize}

\end{document}


%% For decades after the discovery of quantum mechanics, the quantum
%% double-slit experiment was just a ``thought experiment'', meant to
%% illustrate the features of quantum mechanics that had been uncovered
%% by other, more complicated experiments.  Nowadays, the most convenient
%% way to do the experiment is with light, using single-photon sources
%% and single-photon detectors.  Quantum interference has also been
%% demonstrated experimentally using electrons, neutrons, and even
%% large-scale particles such as buckyballs.
