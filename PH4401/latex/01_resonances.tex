\documentclass[pra,11pt]{revtex4}
\usepackage{amsmath}
\usepackage{amssymb}
\usepackage{graphicx}
\usepackage{color}
\def\ket#1{\left|#1\right\rangle}
\def\bra#1{\left\langle#1\right|}
\def\braket#1{\left\langle#1\right\rangle}

\setlength{\parindent}{0pt}

\renewcommand{\baselinestretch}{1.0}
\setlength{\parskip}{0.07in}

\begin{document}

\section{Bound states and extended states}

The quantum states of particles in infinite space come in two distinct
types: \textbf{extended states} (which, as the name suggests, extend
over the entire space) and \textbf{bound states} (which are localized
to a particular region of space).  Both types can co-exist in a single
system.

A simple model containing bound and extended states is the
\textbf{1D finite potential well}.  Consider the Hamiltonian
$$\hat{H} = \frac{\hat{p}^2}{2m} - V_0 \,\Theta(a -|\hat{x}|),$$
where $\hat{x}$ and $\hat{p}$ are the 1D position and momentum
operators, $m$ is the particle mass, $V_0$ and $a$ are positive real
parameters, and $\Theta$ denotes the Heaviside step function (which
gives 1 if the input is positive, and 0 otherwise).  The potential
function describes a square well of depth $V_0$ and width $2a$, whose
value is $-V_0$ for $|x| < a$ and $0$ for $|x|>a$.

\textcolor{red}{[Fig.]}

We are interested in solutions to the time-independent Schr\"odinger
equation for this Hamiltonian.  But we will not discuss the solution
procedure in detail here (if interested, refer to Appendix A).
Instead, we'll zoom in on certain key features.

When solving the Schr\"odinger equation, we must choose boundary
conditions at infinity.  The choice depends on whether we're looking
for a bound or extended state.  A bound state's wavefunction
diminishes exponentially as $x \rightarrow \pm\infty$: in either
external region (i.e., $x < -a$ or $x > a$), it satisfies
$$-\frac{\hbar^2}{2m}\,\frac{d^2\psi}{dx^2} = E \psi(x),$$
subject to the boundary conditions
$$\psi(x) \overset{x\rightarrow\pm\infty}{\sim} e^{\mp\kappa x}, \;\;\;\mathrm{Re}(\kappa) > 0.$$
Therefore, the solutions in the external regions take the form
$$\psi(x) = c_\pm\, e^{\mp\kappa x}, \;\;\mathrm{where}\;\, -\frac{\hbar^2\kappa^2}{2m} = E.$$
Since $E$ is real, it follows that $\kappa$ is real, and hence $E <
0$.  Furthermore, the variational principle implies that $E \ge -V_0$,
so we conclude that bound state energies are restricted to the range
$-V_0 \le E < 0$.  Moreover, it can be shown that the bound state
energies form a \textit{discrete} set within this range; the energy
spacing decreases with increasing $a$, but is larger than zero so long
as $a$ is finite (similar to the particle-in-a-box problem).  Thus,
the total number of bound states is finite.

Each bound state wavefunction takes the form
$$\psi(x) = \begin{cases} c_-\, e^{\kappa x}, & \;\;\;x < -a\\ (\mathrm{something}) , & -a < x < a\\ c_+\, e^{-\kappa x} , & \;\;\,x > a.\end{cases}$$
Since $|\psi(x)|^2$ vanishes exponentially as $x \rightarrow \pm
\infty$, such a wavefunction can always be normalized:
$$\int_{-\infty}^\infty |\psi(x)|^2\, dx\; =\; 1.$$

An extended state, by contrast, does not vanish exponentially at
infinity; it takes the form
$$\psi(x) = \begin{cases} \alpha_-\, e^{ik x} + \beta_-\, e^{-ik x}, & \;\;\;x < -a\\ (\mathrm{something}) , & -a < x < a\\ \alpha_+\, e^{ik x} + \beta_+\, e^{-ik x} , & \;\;\,x > a.\end{cases}$$
Within each external region, $\psi(x)$ consists of a superposition of
left-moving and right-moving plane waves of wavenumber $k \in
\mathbb{R}$.  In order to satisfy Schr\"odinger's equation, $k$
satisfies
$$\frac{\hbar^2k^2}{2m} = E,$$
which means that extended states occur only for energies $E \ge 0$.
In fact, the solutions form a continuum, meaning that there are
extended states for \textit{every} $E \ge 0$.  Since $|\psi(x)|^2$
does not diminish exponentially at infinity, $\int_{-\infty}^\infty
|\psi(x)|^2\, dx$ cannot converge to a finite value, except in the
trivial case where the integrand scales to zero.  Hence, the
wavefunction of an extended state has no finite normalization.  (Note:
we formally exclude the case of wavefunctions that blow up at
infinity, rather than going to a constant magnitude.  To understand
why, and for a more rigorous discussion of how extended states are
normalized, see Appendix B.)

The above results are summarized in the following figure:

\textcolor{red}{[Fig.]}

For the energy range ``inside'' the potential well, $-V_0 < E < 0$,
there is a discrete and finite set of bound states, whose
wavefunctions vanish exponentially outside the well and are
normalizable.  For $E \ge 0$, there is a continuum of extended states,
which have wavefunctions that are spread out over the entire space and
do not have any finite normalization.

\section{Quasi-bound states}

For the finite potential well studied in the previous section, the
bound states and extended states are extremely different from each
other.  But under certain circumstances, an extended state can take on
characteristics that make it very similar to a bound state.  Such
states are called \textbf{quasi-bound states}, and as we shall see,
they play an incredibly important role in various types of scattering
experiments.

An example of a potential that supports quasi-bound states is shown
below.  It consists of a ``wall'' of potential $V_1$, in which we
embed a shallow well of potential $V_0$.  In the external region, the
potential goes to zero.  Since $V_0, V_1 > 0$, there do not exist any
bound states.

\textcolor{red}{[Fig.]}

Although there are no bound states, the situation in the energy range
$V_0 < E < V_1$ seems peculiar.






\appendix
\section{Transfer matrix method for solving the Schr\"odinger equation}


\section{Normalization of bound states and extended states}

In this Appendix, we take a closer look at where bound and extended
states come from, and how they are normalized.  Imagine that instead
of taking $x \in (-\infty,\infty)$, we enclose the system in a very
large but finite box, so that $x \in [-L,L]$ where $L \gg a$.  We
impose periodic boundary conditions on the box walls, $\psi(-L) \equiv
\psi(L)$, and solve the Schr\"odinger equation for wavefunctions
normalizable within the box, such that $\int_{-L}^{L} |\psi(x)|^2 dx =
1$.  For any finite $L$, there is an infinite but discrete set of
solutions, with energies in the range $-V_0 <E < \infty$.

We now examine how the solutions vary upon increasing $L$.

\textcolor{red}{[Fig.]}





\section{Further Reading}

\begin{itemize}
\item Bransden \& Joachain, \S4.4, 9.2--9.3, 13.4
\item Sakurai, \S5.6, 7.7--7.8

\end{itemize}


\end{document}


%% For decades after the discovery of quantum mechanics, the quantum
%% double-slit experiment was just a ``thought experiment'', meant to
%% illustrate the features of quantum mechanics that had been uncovered
%% by other, more complicated experiments.  Nowadays, the most convenient
%% way to do the experiment is with light, using single-photon sources
%% and single-photon detectors.  Quantum interference has also been
%% demonstrated experimentally using electrons, neutrons, and even
%% large-scale particles such as buckyballs.
