\documentclass[10pt,a4paper]{article}
\usepackage{amsmath}
\usepackage{amssymb}
\usepackage{graphicx}
\usepackage{color}
\usepackage{fancyhdr}
\usepackage{fancyvrb}
\usepackage[margin=3.5cm]{geometry}
\usepackage{framed}
\usepackage{enumerate}
\usepackage{textcomp}
\def\ket#1{\left|#1\right\rangle}
\def\bra#1{\left\langle#1\right|}
\def\braket#1{\left\langle#1\right\rangle}

\definecolor{linkcol}{rgb}{0.0, 0.0, 0.5}
\usepackage[colorlinks=true,urlcolor=linkcol,citecolor=black,linkcolor=linkcol]{hyperref}

\renewcommand\thesection{5.\arabic{section}}
\renewcommand\thesubsection{\thesection.\arabic{subsection}}

\fancyhf{}
\lhead{\tiny Y.~D.~Chong (2016)}
\rhead{\scriptsize MH2801: Complex Methods for the Sciences}
\lfoot{}
\rfoot{\thepage}
\pagestyle{fancy}

\makeatletter
\def\PY@reset{\let\PY@it=\relax \let\PY@bf=\relax%
    \let\PY@ul=\relax \let\PY@tc=\relax%
    \let\PY@bc=\relax \let\PY@ff=\relax}
\def\PY@tok#1{\csname PY@tok@#1\endcsname}
\def\PY@toks#1+{\ifx\relax#1\empty\else%
    \PY@tok{#1}\expandafter\PY@toks\fi}
\def\PY@do#1{\PY@bc{\PY@tc{\PY@ul
\def\PYZdl{\char`\$}
\def\PYZhy{\char`\-}
\def\PYZsq{\char`\'}
\def\PYZdq{\char`\"}
\def\PYZti{\char`\~}

\begin{document}
\setcounter{page}{35}
\noindent
\underline{\textbf{\LARGE 5. Complex Waves}}
\vskip 0.1in

Complex numbers are extremely useful for describing the propagation of
waves. This includes electromagnetic waves (radio waves, visible
light, X-rays, etc.), sound waves, quantum mechanical wavefunctions,
and more.  It is therefore very important for physicists to have a
good understanding of the complex description of wave phenomena.

\section{The wave equation}
\label{the-wave-equation}

To describe wave propagation in space and time, we use a partial
differential equation (PDE) known as the \textbf{time-dependent wave
  equation}. For simplicity, we restrict our attention to a single
spatial coordinate, denoted $x$.  The time coordinate is denoted
$t$. We describe a wave using a function $f(x,t)$, called the
\textbf{wavefunction}, which specifies the value of some measurable
physical quantity at each position $x$ and time $t$. For instance, for
a sound wave $f(x,t)$ stands for the pressure of the air at that
position and time.

The time-independent wave equation is:
\begin{equation}
  \frac{\partial^2 f}{\partial x^2} = \frac{1}{c^2} \frac{\partial^2 f}{\partial t^2}.
  \label{wavepde}
\end{equation}
The constant $c$ is called the \textbf{wave speed}, for a reason that
will shortly become clear.  For neatness, we sometimes write the wave
equation by putting everything on one side:
\begin{equation}
  \left(\frac{\partial^2}{\partial x^2} - \frac{1}{c^2} \frac{\partial^2}{\partial t^2}\right) \; f(x,t) = 0.
\end{equation}

\subsection{Real solutions to the wave equation}
\label{real-solutions-to-the-wave-equation}

We first consider real solutions to the wave equation. There exists a
family of solutions known as \textbf{travelling waves}, which have the
form
\begin{equation}
  f(x,t) = f_0 \, \cos\big(kx - \omega t + \phi\big)\;\;\mathrm{where}\;\; \left|\frac{\omega}{k}\right| = c.
  \label{travelsol}
\end{equation}
By direct substitution, you can verify that this satisfies
Eq.~(\ref{wavepde}).  Here, $f_0$ is called the \textbf{amplitude} of
the wave, $\phi$ is the \textbf{phase}, $\omega$ is the
\textbf{frequency}, and $k$ is the \textbf{wavenumber}. (Note: some
authors call $\omega$ the ``angular frequency'', reserving the term
``frequency'' for the quantity $f = \omega/2\pi$. But we'll simply
deal with $\omega$ rather than $f$, and call $\omega$ the frequency.)
By convention, we let $\omega > 0$. The frequency and wavenumber are
inversely related to the \textbf{period} $T = 2\pi/\omega$ and the
\textbf{wavelength} $\lambda = 2\pi/|k|$.

Eq.~(\ref{travelsol}) describes a sinusoidal wave that is moving to
the right (for positive $k$) or to the left (for negative $k$) with
constant speed $c$. To see why, observe that if time advances by
$\delta t$, the cosine is left unchanged if we change $x$ by $\delta x
= \omega \delta t / k$. Thus, during the time interval $\delta t$, the
wave shifts by $\delta x$, so
\begin{equation}
  \mathrm{speed} \; = \frac{|\delta x|}{\delta t} = \frac{\omega}{|k|} = c.
\end{equation}
This is why we refer to $c$ as the wave speed.

The travelling wave solution is valid for \emph{any} $\omega > 0$. For
sound waves, these different frequencies correspond to the human
sensation of pitch. For light waves, the different frequencies
correspond to color. Since $|k| = \omega/c$, higher frequencies
correspond to larger wavenumbers, i.e.~shorter wavelengths.

Since the wave equation is linear, any linear superposition of
traveling waves is also a solution. For example, suppose we have two
waves of equal amplitude and frequency, moving in opposite directions:
\begin{equation}
  f(x,t) = f_0 \, \cos(kx - \omega t + \phi_1) + A \cos(-kx - \omega t + \phi_2),
\end{equation}
for some $k = \omega/c$. Such a superposition is a solution to the
wave equation known as a \textbf{standing wave}:
\begin{equation}
  f(x,t) = 2f_0 \, \cos\big[kx + (\phi_1-\phi_2)/2\big]\, \cos\big[\omega t - (\phi_1+\phi_2)/2\big].
\end{equation}
This is easily proven for the simplest case, $\phi_1 = \phi_2 = 0$.
For arbitrary $\phi_1, \phi_2$, the proof involves simple but tedious
applications of the trignometric addition formulas.

\subsection{Complex solutions to the wave equation}
\label{complex-solutions-to-the-wave-equation}

It is much easier to deal with the wave equation if we look at
\emph{complex} solutions, by allowing the wavefunction $f(x,t)$ to
take on complex values. (Note: only the value of the wavefunction is
complex; we continue to assume that $x$ and $t$ are real.) Since the
wave equation is a linear PDE, the real part of any complex solution
is a valid real solution.

The wave equation has complex solutions that are \textbf{complex
  travelling waves}, which take the form
\begin{equation}
  f(x,t) = A \, e^{i(kx - \omega t)} \quad\mathrm{where}\;\; \left|\frac{\omega}{k}\right| = c.
\end{equation}
Again, you can verify by direct substitution that this satisfies the
PDE. The complex constant $A$ is called the \textbf{complex amplitude}
of the wave. Now, consider what happens if we take the real part of
the complex travelling wave solution:
\begin{align}
  \mathrm{Re}\Big\{A \, e^{i(kx - \omega t)}\Big\}
  &= \mathrm{Re}\Big\{ \left(\big|A\big|\, e^{i\mathrm{arg}[A]}\right) \; e^{i(kx - \omega t)}\Big\} \\
  &= \big|A\big|\; \mathrm{Re}\Big\{ e^{i\mathrm{arg}[A]} \, e^{i(kx - \omega t)}\Big\} \\
  &= \big|A\big|\; \cos\big(kx - \omega t + \mathrm{arg}[A]\big)
\end{align}
Here, we have made use of the polar representation of the complex
amplitude $A$. Evidently, the magnitude $|A|$ serves as the
\emph{real} wave amplitude, while $\mathrm{arg}(A)$ serves as the
phase factor $\phi$. In this sense, the complex solution is more
mathematically succinct: a single complex parameter $A$ combines the
roles of two separate parameters in the real solution.

The complex representation makes wave superpositions much easier to
handle. For example, consider again the superposition of
equal-amplitude wave of frequency $\omega$, with arbitrary phases:
\begin{align}
  f(x,t) &= \big|A\big| \, e^{i(kx - \omega t + \phi_1)}
  + \big|A\big| \, e^{i(-kx - \omega t + \phi_2)} \\
  &= \big|A\big|\, \left(e^{i(kx + \phi_1)}
  + e^{-i(kx - \phi_2)}\right)\, e^{-i\omega t} \\
  &= \big|A\big|\, \left(e^{i[kx + (\phi_1-\phi_2)/2]} + e^{-i[kx + (\phi_1 - \phi_2)/2]}\right)\,
  e^{i(\phi_1 + \phi_2)/2} \,e^{-i\omega t} \\
  &= 2\big|A\big|\, \cos\left[kx + (\phi_1-\phi_2)/2\right] \,
  e^{-i[\omega t -(\phi_1+\phi_2)/2]}
\end{align}
Taking the real part then yields the result that we had previously
obtained using trigonometric formulas.

\section{Waves in 3D space}
\label{waves-in-3d-space}

The wave equation can be easily generalized to three spatial
dimensions.  Instead of $f(x,t)$, we have a wavefunction that depends
on three spatial coordinates, $f(x,y,z,t)$. The second-order
derivative in $x$ is replaced by second-order derivatives in each
spatial direction.  The generalized PDE is:
\begin{equation}
  \left[\frac{\partial^2}{\partial x^2} + \frac{\partial^2}{\partial y^2}
    + \frac{\partial^2}{\partial z^2}
    - \frac{1}{c^2} \frac{\partial^2}{\partial t^2}\right] \; f(x,y,z,t) = 0.
  \label{3dwave}
\end{equation}
This PDE supports complex travelling wave solutions of the form
\begin{equation}
  f(x,y,z,t) = A \, e^{i(\vec{k} \cdot \vec{r} - \omega t)},
\end{equation}
where
\begin{equation}
  \vec{k} = \begin{bmatrix}k_x\\k_y\\k_z\end{bmatrix}, \;\;\;
    \vec{r} = \begin{bmatrix}x\\y\\z\end{bmatrix},
      \;\;\;\frac{\omega}{\left|\vec{k}\right|} = c.
\end{equation}
Again, you can verify that this is a solution by direct
substitution. We call $\vec{k}$ the \textbf{wave-vector}, which
generalizes the wave-number parameter in the solution for 1D
space. The direction of the wave-vector the spatial direction in which
the wave travels.

\section{Harmonic waves}
\label{harmonic-waves}

We are often interested in waves undergoing \textbf{harmonic
  oscillation}, i.e.~varying sinusoidally with a constant frequency
$\omega$. Such waves can be described by wavefunctions of the form
\begin{equation}
  f(x,y,z,t) = \psi(x,y,z) \, e^{-i\omega t}.
\end{equation}
By writing a wavefunction in this form, we are performing a
\textbf{separation of variables}, which means specializing to
solutions consisting of two factors, one depending only on $\vec{r}$
and the other only on $t$. This is a common method for simplifying
PDEs, and is justified by the linearity of the wave equation. If we
can find harmonic solutions for each frequency $\omega$, then we can
linearly combine them to form more general solutions. As we shall see
when discussing Fourier transforms, such superpositions can be used to
construct a general set of solutions to the PDE.

By direct substitution into Eq.~(\ref{3dwave}), we can show that
$\psi(x)$ satisfies the differential equation
\begin{equation}
  \left[\frac{\partial^2}{\partial x^2} + \frac{\partial^2}{\partial y^2}
    + \frac{\partial^2}{\partial z^2}
    + \left(\frac{\omega}{c}\right)^2\right] \, \psi(x) = 0.
\end{equation}
This is related to the original time-dependent wave equation by the
replacement of each $\partial/\partial t$ with $-i\omega$. Thus, it
contains $\omega$ as a numerical parameter.

We will postpone further discussion of the solution of this PDE till
later.

\section{Exercises}

\begin{enumerate}
\item
The complex representation for waves provides a convenient way to
describe damping and amplification. Consider again the complex wave
equation
\begin{equation}
  \frac{\partial^2 f}{\partial x^2}
  = \left(\frac{n}{c}\right)^2 \, \frac{\partial^2 f}{\partial t^2},
\end{equation}
where $n$ is called the \textbf{refractive index}, which can take
complex values.

\begin{itemize}
\item
Show that the propagating-wave solutions are either damped or amplified.
What determines whether they are damped or amplified? In your answer,
consider both left- and right-moving waves.

\item
Try finding a real differential equation which has the same
damped/amplified wave solutions.

\item
Consider the 1D wave equation in a enclosed box of length $L$ and
uniform refractive index $n\in\mathbb{R}$. Let the walls of the box be
at $x = -L/2$ and $x = L/2$, and let the wavefunction go to zero at
these points: $\psi(\pm L/2) = 0$. For this boundary conditions, show
that $\psi(x) = 0$ for all $x$, \emph{except} for certain discrete
values of $\omega$. Find these frequencies, and the corresponding
non-zero solutions $\psi(x)$.
\end{itemize}
\end{enumerate}

\end{document}
