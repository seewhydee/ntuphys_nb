\documentclass[10pt,a4paper]{article}
\usepackage{amsmath}
\usepackage{amssymb}
\usepackage{graphicx}
\usepackage{color}
\usepackage{fancyhdr}
\usepackage{fancyvrb}
\usepackage[margin=3.5cm]{geometry}
\usepackage{framed}
\usepackage{enumerate}
\usepackage{textcomp}
\def\ket#1{\left|#1\right\rangle}
\def\bra#1{\left\langle#1\right|}
\def\braket#1{\left\langle#1\right\rangle}

\definecolor{linkcol}{rgb}{0.0, 0.0, 0.5}
\usepackage[colorlinks=true,urlcolor=linkcol,citecolor=black,linkcolor=linkcol]{hyperref}

\setcounter{section}{4}
\renewcommand\thesection{\arabic{section}}
\renewcommand\thesubsection{\thesection.\arabic{subsection}}

\fancyhf{}
\lhead{\tiny Y.~D.~Chong (2018)}
\rhead{\scriptsize MH2801: Complex Methods for the Sciences}
\lfoot{}
\rfoot{\thepage}
\pagestyle{fancy}

\makeatletter
\def\PY@reset{\let\PY@it=\relax \let\PY@bf=\relax%
    \let\PY@ul=\relax \let\PY@tc=\relax%
    \let\PY@bc=\relax \let\PY@ff=\relax}
\def\PY@tok#1{\csname PY@tok@#1\endcsname}
\def\PY@toks#1+{\ifx\relax#1\empty\else%
    \PY@tok{#1}\expandafter\PY@toks\fi}
\def\PY@do#1{\PY@bc{\PY@tc{\PY@ul
\def\PYZdl{\char`\$}
\def\PYZhy{\char`\-}
\def\PYZsq{\char`\'}
\def\PYZdq{\char`\"}
\def\PYZti{\char`\~}

\begin{document}
\setcounter{page}{38}

\section{Complex Waves}\label{complex-waves}

Complex numbers are extremely useful for describing the propagation of
waves. This includes electromagnetic waves (radio waves, visible light,
X-rays, etc.), sound waves, quantum mechanical wavefunctions, and more.
It is therefore very important for physicists to have a good
understanding of the complex description of wave phenomena.

\subsection{The wave equation}\label{the-wave-equation}

To describe wave propagation in space and time, we use a partial
differential equation (PDE) known as the \textbf{time-dependent wave
  equation}. For simplicity, we restrict our attention to a single
spatial coordinate, denoted $x$.  The time coordinate is denoted
$t$. We describe a wave using a function $f(x,t)$, called the
\textbf{wavefunction}, which specifies the value of some measurable
physical quantity at each position $x$ and time $t$. For instance, for
a sound wave $f(x,t)$ stands for the pressure of the air at that
position and time.

The time-independent wave equation is:
\begin{equation}
\frac{\partial^2 f}{\partial x^2} = \frac{1}{v^2} \frac{\partial^2 f}{\partial t^2}, \;\;\; v \in\mathbb{R}^+.
\end{equation}
The parameter $v$, which we currently assume to be a positive real
constant, is called the \textbf{wave speed}. The reason for this will
shortly become clear.

For neatness, we sometimes write the wave equation by putting everything
on one side:
\begin{equation}
\left(\frac{\partial^2}{\partial x^2} - \frac{1}{v^2} \frac{\partial^2}{\partial t^2}\right) \; f(x,t) = 0.
\end{equation}

\subsubsection{Real solutions to the wave equation}
\label{real-solutions-to-the-wave-equation}

We first consider real solutions to the wave equation. There exists a
family of solutions known as \textbf{travelling waves}, which have the
form
\begin{equation}
f(x,t) = f_0 \, \cos\big(kx - \omega t + \phi\big)\;\;\mathrm{where}\;\; \left|\frac{\omega}{k}\right| = v.
\end{equation}
By direct substitution, you can verify that this satisfies the PDE.
Here, $f_0$ is called the \textbf{amplitude} of the wave, $\phi$ is
the \textbf{phase}, $\omega$ is the \textbf{frequency}, and $k$ is
the \textbf{wavenumber}. (Note: some authors call $\omega$ the
``angular frequency'', reserving the term ``frequency'' for the quantity
$f = \omega/2\pi$. But we'll simply deal in terms of $\omega$ rather
than $f$, and call $\omega$ the frequency.) By convention, we
usually take $\omega > 0$. The frequency and wavenumber are inversely
related to the \textbf{period} $T = 2\pi/\omega$ and the
\textbf{wavelength} $\lambda = 2\pi/|k|$.

This solution describes a sinusoidal wave.  The wave moves to the
right for positive $k$, or to the left for negative $k$. Here's why:
consider a small change in time, $\delta t$. If, together with this
time shift, we change $x$ by $\delta x = \omega \delta t / k$, then
the change in the $kx$ term and the change in the $\omega t$ term in
$\cos(kx - \omega t + \phi)$ would cancel each other, leaving the
value of the cosine unchanged. This means that the wave shifts by
$\delta x$ during the time interval $\delta t$. The wave velocity is
\begin{equation}
\textrm{velocity} \; = \, \frac{\delta x}{\delta t} = \frac{\omega \delta t / k}{\delta t} = \frac{\omega}{k}.
\end{equation}
Note that we have adopted the convention that $\omega > 0$. The above
equation tells us that the sign of the wave-number determines the
propagation direction: positive $k$ means positive wave velocity (a
right-moving wave), and negative $k$ means negative wave velocity (a
left-moving wave). We can also compute the wave speed by taking the
absolute value of the velocity:
\begin{equation}
\textrm{speed}\; = \, \left|\frac{\delta x}{\delta t}\right| = \left|\frac{\omega}{k}\right| = v.
\end{equation}
The travelling wave solution is valid for \emph{any} $\omega > 0$. For
sound waves, these different frequencies correspond to the human
sensation of pitch. For light waves, the different frequencies
correspond to color. Since $|k| = \omega/v$, higher frequencies
correspond to larger wavenumbers, i.e.~shorter wavelengths.

The wave equation is a \textbf{linear} PDE, meaning that any linear
superposition of solutions is also a solution. (You can prove this by
direct substitution.) For instance, suppose we have two travelling wave
solutions, with equal amplitude and frequency, moving in opposite
directions:
\begin{equation}
f(x,t) = f_0 \, \cos(kx - \omega t + \phi_1) + A \cos(-kx - \omega t + \phi_2),
\end{equation}
for some $k = \omega/c$. Such a superposition is a solution to the
wave equation known as a \textbf{standing wave}:
\begin{equation}
f(x,t) = 2f_0 \, \cos\big[kx + (\phi_1-\phi_2)/2\big]\, \cos\big[\omega t - (\phi_1+\phi_2)/2\big].
\end{equation}
This is easily proven for the simplest case, $\phi_1 = \phi_2 = 0$.
For arbitrary $\phi_1, \phi_2$, the proof involves simple but tedious
applications of the trignometric addition formulas.

\subsubsection{Complex solutions to the wave equation}
\label{complex-solutions-to-the-wave-equation}

It is much easier to deal with the wave equation if we look at
\emph{complex} solutions, by allowing the wavefunction $f(x,t)$ to
take on complex values. (Note: only the value of the wavefunction is
complex; we continue to assume that $x$ and $t$ are real.) Since the
wave equation is a linear PDE, the real part of any complex solution
is a valid real solution:
\begin{equation}
\left(\frac{\partial^2}{\partial x^2} - \frac{1}{v^2} \frac{\partial^2}{\partial t^2}\right) f(x,t) = 0 \;\;\;\Rightarrow \;\;\; \left(\frac{\partial^2}{\partial x^2} - \frac{1}{v^2} \frac{\partial^2}{\partial t^2}\right) \mathrm{Re}\left[f(x,t)\right] = 0.
\end{equation}
There is a very nice set of complex solutions to the wave equation,
known as \textbf{complex travelling waves}, which take the form
\begin{equation}
f(x,t) = A \, e^{i(kx - \omega t)} \quad\mathrm{where}\;\; \left|\frac{\omega}{k}\right| = v.
\end{equation}
Again, you can verify by direct substitution that this satisfies the
PDE. The complex constant $A$ is called the \textbf{complex amplitude}
of the wave.

Now, consider what happens if we take the real part of the above
solution:
\begin{align}
  \mathrm{Re}\Big\{A \, e^{i(kx - \omega t)}\Big\} &= \mathrm{Re}\Big\{ \left(\big|A\big|\, e^{i\mathrm{arg}[A]}\right) \; e^{i(kx - \omega t)}\Big\} \\ &= \big|A\big|\; \mathrm{Re}\Big\{ e^{i\mathrm{arg}[A]} \, e^{i(kx - \omega t)}\Big\} \\ &= \big|A\big|\; \cos\big(kx - \omega t + \mathrm{arg}[A]\big)
\end{align}
Here, we have made use of the polar representation of the complex
amplitude $A$. Evidently, the magnitude $|A|$ serves as the
\emph{real} wave amplitude, while $\mathrm{arg}(A)$ serves as the
phase factor $\phi$. In this sense, the complex solution is more
mathematically succinct: a single complex parameter $A$ combines the
roles of two separate parameters in the real solution.

The complex representation makes wave superpositions much easier to
handle. For example, consider again the superposition of two
counter-propagating waves of equal amplitude and frequency, with
arbitrary phases. Using complex travelling waves, we can calculate the
superposition using a few lines of complex algebra:
\begin{align}
  f(x,t) &= \displaystyle \big|A\big| \, e^{i(kx - \omega t + \phi_1)} + \big|A\big| \, e^{i(-kx - \omega t + \phi_2)} \\ &= \displaystyle \big|A\big|\, \left(e^{i(kx + \phi_1)} + e^{-i(kx - \phi_2)}\right)\, e^{-i\omega t} \\ &= \displaystyle \big|A\big|\, \left(e^{i[kx + (\phi_1-\phi_2)/2]} + e^{-i[kx + (\phi_1 - \phi_2)/2]}\right)\, e^{i(\phi_1 + \phi_2)/2} \,e^{-i\omega t} \\ &= \displaystyle 2\big|A\big|\, \cos\left[kx + (\phi_1-\phi_2)/2\right] \,e^{-i[\omega t -(\phi_1+\phi_2)/2]}
\end{align}
Taking the real part then yields the result that we had previously
obtained using trigonometric formulas.

\subsection{Waves in 3D space}
\label{waves-in-3d-space}

The wave equation can be easily generalized to three spatial
dimensions.  We replace $f(x,t)$ with a wavefunction that depends on
three spatial coordinates, $f(x,y,z,t)$. The second-order derivative
in $x$ is replaced by second-order derivatives in each spatial
direction. The generalized PDE is:
\begin{equation}
\left(\frac{\partial^2}{\partial x^2} + \frac{\partial^2}{\partial y^2} + \frac{\partial^2}{\partial z^2} - \frac{1}{v^2} \frac{\partial^2}{\partial t^2}\right) \; f(x,y,z,t) = 0.
\label{wave3d}
\end{equation}
This PDE supports complex travelling wave solutions of the form
\begin{equation}
f(x,y,z,t) = A \, e^{i(\vec{k} \cdot \vec{r} - \omega t)},
\end{equation}
where
\begin{equation}
\vec{k} = \begin{bmatrix}k_x\\k_y\\k_z\end{bmatrix}, \;\;\; \vec{r} = \begin{bmatrix}x\\y\\z\end{bmatrix}, \;\;\;\frac{\omega}{\sqrt{k_x^2 + k_y^2 + k_z^2}} = v.
\end{equation}
Again, you can verify that this is a solution by direct substitution. We
call $\vec{k}$ the \textbf{wave-vector}, which generalizes the
wave-number parameter in the solution for 1D space. The direction of the
wave-vector specifies the spatial direction in which the wave travels.

\subsection{Harmonic waves}
\label{harmonic-waves}

We are often interested in waves undergoing \textbf{harmonic
  oscillation}, i.e.~varying sinusoidally with a constant frequency
$\omega$. Such waves can be described by wavefunctions of the form
\begin{equation}
f(x,y,z,t) = \psi(x,y,z) \, e^{-i\omega t}.
\end{equation}
By writing a wavefunction in this form, we are performing a
\textbf{separation of variables}, which means specializing to
solutions consisting of two factors, one depending only on $\vec{r}$
and the other only on $t$. This is a common method for simplifying
PDEs, and is justified by the linearity of the wave equation. If we
can find harmonic solutions for each frequency $\omega$, then we can
linearly combine them to form more general solutions.

By direct substitution into the 3D wave equation \eqref{wave3d}, we
can show that $\psi(x)$ satisfies the differential equation
\begin{equation}
\left[\frac{\partial^2}{\partial x^2} + \frac{\partial^2}{\partial y^2} + \frac{\partial^2}{\partial z^2} + \left(\frac{\omega}{v}\right)^2\right] \, \psi(x,y,z) = 0.
\end{equation}
This is related to the original time-dependent wave equation by the
replacement of each $\partial/\partial t$ with $-i\omega$. Thus, it
contains $\omega$ as a numerical parameter.

    \hypertarget{waves-in-complex-media}{%
\subsubsection{Waves in complex media}\label{waves-in-complex-media}}

So far, our discussion has been limited to waves propagating in a
uniform, energy-conserving medium that has a fixed wave speed $v$.
There are two important generalizations of this scenario: (i)
non-uniform media, in which the wave speed varies with position, and
(ii) energy non-conserving media, in which the waves lose or gain energy
as they propagate. To capture these phenomena, we replace the constant
$v$ by
\begin{equation}
v = \frac{c}{n},
\end{equation}
where $n$ is called the \textbf{refractive index}, and the constant
$c$ is the wave speed in the limit $n = 1$. In the case of
electromagnetic waves, $c$ is called the ``speed of light in a
vacuum''.

If the refractive index is now allowed to vary with position, the wave
equation becomes
\begin{equation}
\left[\frac{\partial^2}{\partial x^2} + \frac{\partial^2}{\partial y^2} + \frac{\partial^2}{\partial z^2} + n^2(x,y,z)\, \left(\frac{\omega}{c}\right)^2\right] \, \psi(x,y,z) = 0.
\end{equation}
This PDE describes harmonic waves of fixed frequency $\omega$
propagating within the medium described by $n(x,y,z)$.

\subsubsection{Wave amplification and attenuation}
\label{wave-amplification-and-attenuation}

If the refractive index $n$ is allowed to be \emph{complex}, the wave
equation can describe \textbf{wave amplification} (gain) and
\textbf{wave attenuation} (loss). In physics, amplified waves are the
underlying basis for lasers, while attenuated waves describe the
absorption of light into black objects, or the dissipation of sound
into ambient heat. To study these phenomena, let us go back to
one-dimensional space and assume that the refractive index is
position-independent:
\begin{equation}
\left[\frac{d^2}{d x^2} + n^2\, \left(\frac{\omega}{c}\right)^2\right] \, \psi(x) = 0, \quad n \in \mathbb{C}.
\end{equation}
The solutions to this ODE have the form
\begin{equation}
\psi(x) = A e^{\pm ikx},\;\;\;\mathrm{where}\;\; A \in \mathbb{C}, \;k = \frac{n\omega}{c}.
\label{traveling}
\end{equation}
As before, we let $\omega, c \in \mathbb{R}^+$. If $n$ is a complex
number with a non-zero imaginary part, then $k$ is likewise complex.
Specifically, if we write
\begin{equation}
n = n' + i n'',
\end{equation}
where $n'$ and $n''$ are the real and imaginary parts of the
refractive index, then
\begin{equation}
\psi(x) = A e^{\pm in'(\omega/c)x}\, e^{\mp n''(\omega/c)x}.
\end{equation}
The first exponential factor describes the oscillation of the
wavefunction, with the $\pm$ sign determining whether the harmonic
wave is moving to the right (positive wave-number) or to the left
(negative wave-number). The second exponential describes the
amplification or attenuation of the wave. If $n'' \ne 0$, the
amplitude varies exponentially with $x$. An increase in the direction
of propagation describes an amplified wave; a decrease in the direction
of propagation describes an attenuated wave.

\subsection{Exercises}
\label{exercises}

\begin{enumerate}
\item 
Consider the 1D wave equation in a enclosed box of length $L$ and
uniform refractive index $n\in\mathbb{R}$. Let the walls of the box be
at $x = -L/2$ and $x = L/2$, and let the wavefunction go to zero at
these points: $\psi(\pm L/2) = 0$. For this boundary conditions, show
that $\psi(x) = 0$ for all $x$, \emph{except} for certain discrete
values of the frequency $\omega$. Find these frequencies, and the
corresponding non-zero solutions $\psi(x)$.

\item
  What determines whether the travelline-wave solution
  \eqref{traveling} is amplified, rather than attenuated? Can a wave
  equation with given complex refractive index $n$ possess both
  amplified wave solutions and attenuated wave solutions?

\item
  When the refractive index is complex, can the real part of the
  complex wavefunction be regarded as the solution to the same wave
  equation? If not, derive a real differential equation whose solution
  is the real part of Eq.~\eqref{traveling}.
\end{enumerate}
    
\end{document}
