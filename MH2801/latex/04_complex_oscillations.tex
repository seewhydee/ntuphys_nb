
% Default to the notebook output style

    


% Inherit from the specified cell style.




    
\documentclass[11pt]{article}

    
    
    \usepackage[T1]{fontenc}
    % Nicer default font than Computer Modern for most use cases
    \usepackage{palatino}

    % Basic figure setup, for now with no caption control since it's done
    % automatically by Pandoc (which extracts ![](path) syntax from Markdown).
    \usepackage{graphicx}
    % We will generate all images so they have a width \maxwidth. This means
    % that they will get their normal width if they fit onto the page, but
    % are scaled down if they would overflow the margins.
    \makeatletter
    \def\maxwidth{\ifdim\Gin@nat@width>\linewidth\linewidth
    \else\Gin@nat@width\fi}
    \makeatother
    \let\Oldincludegraphics\includegraphics
    % Set max figure width to be 80% of text width, for now hardcoded.
    \renewcommand{\includegraphics}[1]{\Oldincludegraphics[width=.8\maxwidth]{#1}}
    % Ensure that by default, figures have no caption (until we provide a
    % proper Figure object with a Caption API and a way to capture that
    % in the conversion process - todo).
    \usepackage{caption}
    \DeclareCaptionLabelFormat{nolabel}{}
    \captionsetup{labelformat=nolabel}

    \usepackage{adjustbox} % Used to constrain images to a maximum size 
    \usepackage{xcolor} % Allow colors to be defined
    \usepackage{enumerate} % Needed for markdown enumerations to work
    \usepackage{geometry} % Used to adjust the document margins
    \usepackage{amsmath} % Equations
    \usepackage{amssymb} % Equations
    \usepackage{textcomp} % defines textquotesingle
    % Hack from http://tex.stackexchange.com/a/47451/13684:
    \AtBeginDocument{%
        \def\PYZsq{\textquotesingle}% Upright quotes in Pygmentized code
    }
    \usepackage{upquote} % Upright quotes for verbatim code
    \usepackage{eurosym} % defines \euro
    \usepackage[mathletters]{ucs} % Extended unicode (utf-8) support
    \usepackage[utf8x]{inputenc} % Allow utf-8 characters in the tex document
    \usepackage{fancyvrb} % verbatim replacement that allows latex
    \usepackage{grffile} % extends the file name processing of package graphics 
                         % to support a larger range 
    % The hyperref package gives us a pdf with properly built
    % internal navigation ('pdf bookmarks' for the table of contents,
    % internal cross-reference links, web links for URLs, etc.)
    \usepackage{hyperref}
    \usepackage{longtable} % longtable support required by pandoc >1.10
    \usepackage{booktabs}  % table support for pandoc > 1.12.2
    \usepackage[normalem]{ulem} % ulem is needed to support strikethroughs (\sout)
                                % normalem makes italics be italics, not underlines
    

    
    
    % Colors for the hyperref package
    \definecolor{urlcolor}{rgb}{0,.145,.698}
    \definecolor{linkcolor}{rgb}{.71,0.21,0.01}
    \definecolor{citecolor}{rgb}{.12,.54,.11}

    % ANSI colors
    \definecolor{ansi-black}{HTML}{3E424D}
    \definecolor{ansi-black-intense}{HTML}{282C36}
    \definecolor{ansi-red}{HTML}{E75C58}
    \definecolor{ansi-red-intense}{HTML}{B22B31}
    \definecolor{ansi-green}{HTML}{00A250}
    \definecolor{ansi-green-intense}{HTML}{007427}
    \definecolor{ansi-yellow}{HTML}{DDB62B}
    \definecolor{ansi-yellow-intense}{HTML}{B27D12}
    \definecolor{ansi-blue}{HTML}{208FFB}
    \definecolor{ansi-blue-intense}{HTML}{0065CA}
    \definecolor{ansi-magenta}{HTML}{D160C4}
    \definecolor{ansi-magenta-intense}{HTML}{A03196}
    \definecolor{ansi-cyan}{HTML}{60C6C8}
    \definecolor{ansi-cyan-intense}{HTML}{258F8F}
    \definecolor{ansi-white}{HTML}{C5C1B4}
    \definecolor{ansi-white-intense}{HTML}{A1A6B2}

    % commands and environments needed by pandoc snippets
    % extracted from the output of `pandoc -s`
    \providecommand{\tightlist}{%
      \setlength{\itemsep}{0pt}\setlength{\parskip}{0pt}}
    \DefineVerbatimEnvironment{Highlighting}{Verbatim}{commandchars=\\\{\}}
    % Add ',fontsize=\small' for more characters per line
    \newenvironment{Shaded}{}{}
    \newcommand{\KeywordTok}[1]{\textcolor[rgb]{0.00,0.44,0.13}{\textbf{{#1}}}}
    \newcommand{\DataTypeTok}[1]{\textcolor[rgb]{0.56,0.13,0.00}{{#1}}}
    \newcommand{\DecValTok}[1]{\textcolor[rgb]{0.25,0.63,0.44}{{#1}}}
    \newcommand{\BaseNTok}[1]{\textcolor[rgb]{0.25,0.63,0.44}{{#1}}}
    \newcommand{\FloatTok}[1]{\textcolor[rgb]{0.25,0.63,0.44}{{#1}}}
    \newcommand{\CharTok}[1]{\textcolor[rgb]{0.25,0.44,0.63}{{#1}}}
    \newcommand{\StringTok}[1]{\textcolor[rgb]{0.25,0.44,0.63}{{#1}}}
    \newcommand{\CommentTok}[1]{\textcolor[rgb]{0.38,0.63,0.69}{\textit{{#1}}}}
    \newcommand{\OtherTok}[1]{\textcolor[rgb]{0.00,0.44,0.13}{{#1}}}
    \newcommand{\AlertTok}[1]{\textcolor[rgb]{1.00,0.00,0.00}{\textbf{{#1}}}}
    \newcommand{\FunctionTok}[1]{\textcolor[rgb]{0.02,0.16,0.49}{{#1}}}
    \newcommand{\RegionMarkerTok}[1]{{#1}}
    \newcommand{\ErrorTok}[1]{\textcolor[rgb]{1.00,0.00,0.00}{\textbf{{#1}}}}
    \newcommand{\NormalTok}[1]{{#1}}
    
    % Additional commands for more recent versions of Pandoc
    \newcommand{\ConstantTok}[1]{\textcolor[rgb]{0.53,0.00,0.00}{{#1}}}
    \newcommand{\SpecialCharTok}[1]{\textcolor[rgb]{0.25,0.44,0.63}{{#1}}}
    \newcommand{\VerbatimStringTok}[1]{\textcolor[rgb]{0.25,0.44,0.63}{{#1}}}
    \newcommand{\SpecialStringTok}[1]{\textcolor[rgb]{0.73,0.40,0.53}{{#1}}}
    \newcommand{\ImportTok}[1]{{#1}}
    \newcommand{\DocumentationTok}[1]{\textcolor[rgb]{0.73,0.13,0.13}{\textit{{#1}}}}
    \newcommand{\AnnotationTok}[1]{\textcolor[rgb]{0.38,0.63,0.69}{\textbf{\textit{{#1}}}}}
    \newcommand{\CommentVarTok}[1]{\textcolor[rgb]{0.38,0.63,0.69}{\textbf{\textit{{#1}}}}}
    \newcommand{\VariableTok}[1]{\textcolor[rgb]{0.10,0.09,0.49}{{#1}}}
    \newcommand{\ControlFlowTok}[1]{\textcolor[rgb]{0.00,0.44,0.13}{\textbf{{#1}}}}
    \newcommand{\OperatorTok}[1]{\textcolor[rgb]{0.40,0.40,0.40}{{#1}}}
    \newcommand{\BuiltInTok}[1]{{#1}}
    \newcommand{\ExtensionTok}[1]{{#1}}
    \newcommand{\PreprocessorTok}[1]{\textcolor[rgb]{0.74,0.48,0.00}{{#1}}}
    \newcommand{\AttributeTok}[1]{\textcolor[rgb]{0.49,0.56,0.16}{{#1}}}
    \newcommand{\InformationTok}[1]{\textcolor[rgb]{0.38,0.63,0.69}{\textbf{\textit{{#1}}}}}
    \newcommand{\WarningTok}[1]{\textcolor[rgb]{0.38,0.63,0.69}{\textbf{\textit{{#1}}}}}
    
    
    % Define a nice break command that doesn't care if a line doesn't already
    % exist.
    \def\br{\hspace*{\fill} \\* }
    % Math Jax compatability definitions
    \def\gt{>}
    \def\lt{<}
    % Document parameters
    \title{04\_complex\_oscillations}
    
    
    

    % Pygments definitions
    
\makeatletter
\def\PY@reset{\let\PY@it=\relax \let\PY@bf=\relax%
    \let\PY@ul=\relax \let\PY@tc=\relax%
    \let\PY@bc=\relax \let\PY@ff=\relax}
\def\PY@tok#1{\csname PY@tok@#1\endcsname}
\def\PY@toks#1+{\ifx\relax#1\empty\else%
    \PY@tok{#1}\expandafter\PY@toks\fi}
\def\PY@do#1{\PY@bc{\PY@tc{\PY@ul{%
    \PY@it{\PY@bf{\PY@ff{#1}}}}}}}
\def\PY#1#2{\PY@reset\PY@toks#1+\relax+\PY@do{#2}}

\expandafter\def\csname PY@tok@nd\endcsname{\def\PY@tc##1{\textcolor[rgb]{0.67,0.13,1.00}{##1}}}
\expandafter\def\csname PY@tok@nl\endcsname{\def\PY@tc##1{\textcolor[rgb]{0.63,0.63,0.00}{##1}}}
\expandafter\def\csname PY@tok@se\endcsname{\let\PY@bf=\textbf\def\PY@tc##1{\textcolor[rgb]{0.73,0.40,0.13}{##1}}}
\expandafter\def\csname PY@tok@err\endcsname{\def\PY@bc##1{\setlength{\fboxsep}{0pt}\fcolorbox[rgb]{1.00,0.00,0.00}{1,1,1}{\strut ##1}}}
\expandafter\def\csname PY@tok@sr\endcsname{\def\PY@tc##1{\textcolor[rgb]{0.73,0.40,0.53}{##1}}}
\expandafter\def\csname PY@tok@kt\endcsname{\def\PY@tc##1{\textcolor[rgb]{0.69,0.00,0.25}{##1}}}
\expandafter\def\csname PY@tok@sx\endcsname{\def\PY@tc##1{\textcolor[rgb]{0.00,0.50,0.00}{##1}}}
\expandafter\def\csname PY@tok@ne\endcsname{\let\PY@bf=\textbf\def\PY@tc##1{\textcolor[rgb]{0.82,0.25,0.23}{##1}}}
\expandafter\def\csname PY@tok@mi\endcsname{\def\PY@tc##1{\textcolor[rgb]{0.40,0.40,0.40}{##1}}}
\expandafter\def\csname PY@tok@go\endcsname{\def\PY@tc##1{\textcolor[rgb]{0.53,0.53,0.53}{##1}}}
\expandafter\def\csname PY@tok@no\endcsname{\def\PY@tc##1{\textcolor[rgb]{0.53,0.00,0.00}{##1}}}
\expandafter\def\csname PY@tok@gr\endcsname{\def\PY@tc##1{\textcolor[rgb]{1.00,0.00,0.00}{##1}}}
\expandafter\def\csname PY@tok@ow\endcsname{\let\PY@bf=\textbf\def\PY@tc##1{\textcolor[rgb]{0.67,0.13,1.00}{##1}}}
\expandafter\def\csname PY@tok@kc\endcsname{\let\PY@bf=\textbf\def\PY@tc##1{\textcolor[rgb]{0.00,0.50,0.00}{##1}}}
\expandafter\def\csname PY@tok@sb\endcsname{\def\PY@tc##1{\textcolor[rgb]{0.73,0.13,0.13}{##1}}}
\expandafter\def\csname PY@tok@ge\endcsname{\let\PY@it=\textit}
\expandafter\def\csname PY@tok@na\endcsname{\def\PY@tc##1{\textcolor[rgb]{0.49,0.56,0.16}{##1}}}
\expandafter\def\csname PY@tok@w\endcsname{\def\PY@tc##1{\textcolor[rgb]{0.73,0.73,0.73}{##1}}}
\expandafter\def\csname PY@tok@bp\endcsname{\def\PY@tc##1{\textcolor[rgb]{0.00,0.50,0.00}{##1}}}
\expandafter\def\csname PY@tok@kd\endcsname{\let\PY@bf=\textbf\def\PY@tc##1{\textcolor[rgb]{0.00,0.50,0.00}{##1}}}
\expandafter\def\csname PY@tok@gi\endcsname{\def\PY@tc##1{\textcolor[rgb]{0.00,0.63,0.00}{##1}}}
\expandafter\def\csname PY@tok@cp\endcsname{\def\PY@tc##1{\textcolor[rgb]{0.74,0.48,0.00}{##1}}}
\expandafter\def\csname PY@tok@mh\endcsname{\def\PY@tc##1{\textcolor[rgb]{0.40,0.40,0.40}{##1}}}
\expandafter\def\csname PY@tok@gt\endcsname{\def\PY@tc##1{\textcolor[rgb]{0.00,0.27,0.87}{##1}}}
\expandafter\def\csname PY@tok@si\endcsname{\let\PY@bf=\textbf\def\PY@tc##1{\textcolor[rgb]{0.73,0.40,0.53}{##1}}}
\expandafter\def\csname PY@tok@nv\endcsname{\def\PY@tc##1{\textcolor[rgb]{0.10,0.09,0.49}{##1}}}
\expandafter\def\csname PY@tok@c\endcsname{\let\PY@it=\textit\def\PY@tc##1{\textcolor[rgb]{0.25,0.50,0.50}{##1}}}
\expandafter\def\csname PY@tok@c1\endcsname{\let\PY@it=\textit\def\PY@tc##1{\textcolor[rgb]{0.25,0.50,0.50}{##1}}}
\expandafter\def\csname PY@tok@mf\endcsname{\def\PY@tc##1{\textcolor[rgb]{0.40,0.40,0.40}{##1}}}
\expandafter\def\csname PY@tok@cm\endcsname{\let\PY@it=\textit\def\PY@tc##1{\textcolor[rgb]{0.25,0.50,0.50}{##1}}}
\expandafter\def\csname PY@tok@s1\endcsname{\def\PY@tc##1{\textcolor[rgb]{0.73,0.13,0.13}{##1}}}
\expandafter\def\csname PY@tok@o\endcsname{\def\PY@tc##1{\textcolor[rgb]{0.40,0.40,0.40}{##1}}}
\expandafter\def\csname PY@tok@kr\endcsname{\let\PY@bf=\textbf\def\PY@tc##1{\textcolor[rgb]{0.00,0.50,0.00}{##1}}}
\expandafter\def\csname PY@tok@cs\endcsname{\let\PY@it=\textit\def\PY@tc##1{\textcolor[rgb]{0.25,0.50,0.50}{##1}}}
\expandafter\def\csname PY@tok@nn\endcsname{\let\PY@bf=\textbf\def\PY@tc##1{\textcolor[rgb]{0.00,0.00,1.00}{##1}}}
\expandafter\def\csname PY@tok@kn\endcsname{\let\PY@bf=\textbf\def\PY@tc##1{\textcolor[rgb]{0.00,0.50,0.00}{##1}}}
\expandafter\def\csname PY@tok@gs\endcsname{\let\PY@bf=\textbf}
\expandafter\def\csname PY@tok@k\endcsname{\let\PY@bf=\textbf\def\PY@tc##1{\textcolor[rgb]{0.00,0.50,0.00}{##1}}}
\expandafter\def\csname PY@tok@sc\endcsname{\def\PY@tc##1{\textcolor[rgb]{0.73,0.13,0.13}{##1}}}
\expandafter\def\csname PY@tok@vc\endcsname{\def\PY@tc##1{\textcolor[rgb]{0.10,0.09,0.49}{##1}}}
\expandafter\def\csname PY@tok@sh\endcsname{\def\PY@tc##1{\textcolor[rgb]{0.73,0.13,0.13}{##1}}}
\expandafter\def\csname PY@tok@ni\endcsname{\let\PY@bf=\textbf\def\PY@tc##1{\textcolor[rgb]{0.60,0.60,0.60}{##1}}}
\expandafter\def\csname PY@tok@nf\endcsname{\def\PY@tc##1{\textcolor[rgb]{0.00,0.00,1.00}{##1}}}
\expandafter\def\csname PY@tok@gd\endcsname{\def\PY@tc##1{\textcolor[rgb]{0.63,0.00,0.00}{##1}}}
\expandafter\def\csname PY@tok@ss\endcsname{\def\PY@tc##1{\textcolor[rgb]{0.10,0.09,0.49}{##1}}}
\expandafter\def\csname PY@tok@s2\endcsname{\def\PY@tc##1{\textcolor[rgb]{0.73,0.13,0.13}{##1}}}
\expandafter\def\csname PY@tok@nc\endcsname{\let\PY@bf=\textbf\def\PY@tc##1{\textcolor[rgb]{0.00,0.00,1.00}{##1}}}
\expandafter\def\csname PY@tok@il\endcsname{\def\PY@tc##1{\textcolor[rgb]{0.40,0.40,0.40}{##1}}}
\expandafter\def\csname PY@tok@gh\endcsname{\let\PY@bf=\textbf\def\PY@tc##1{\textcolor[rgb]{0.00,0.00,0.50}{##1}}}
\expandafter\def\csname PY@tok@kp\endcsname{\def\PY@tc##1{\textcolor[rgb]{0.00,0.50,0.00}{##1}}}
\expandafter\def\csname PY@tok@vi\endcsname{\def\PY@tc##1{\textcolor[rgb]{0.10,0.09,0.49}{##1}}}
\expandafter\def\csname PY@tok@vg\endcsname{\def\PY@tc##1{\textcolor[rgb]{0.10,0.09,0.49}{##1}}}
\expandafter\def\csname PY@tok@s\endcsname{\def\PY@tc##1{\textcolor[rgb]{0.73,0.13,0.13}{##1}}}
\expandafter\def\csname PY@tok@cpf\endcsname{\let\PY@it=\textit\def\PY@tc##1{\textcolor[rgb]{0.25,0.50,0.50}{##1}}}
\expandafter\def\csname PY@tok@mo\endcsname{\def\PY@tc##1{\textcolor[rgb]{0.40,0.40,0.40}{##1}}}
\expandafter\def\csname PY@tok@nt\endcsname{\let\PY@bf=\textbf\def\PY@tc##1{\textcolor[rgb]{0.00,0.50,0.00}{##1}}}
\expandafter\def\csname PY@tok@gp\endcsname{\let\PY@bf=\textbf\def\PY@tc##1{\textcolor[rgb]{0.00,0.00,0.50}{##1}}}
\expandafter\def\csname PY@tok@m\endcsname{\def\PY@tc##1{\textcolor[rgb]{0.40,0.40,0.40}{##1}}}
\expandafter\def\csname PY@tok@nb\endcsname{\def\PY@tc##1{\textcolor[rgb]{0.00,0.50,0.00}{##1}}}
\expandafter\def\csname PY@tok@gu\endcsname{\let\PY@bf=\textbf\def\PY@tc##1{\textcolor[rgb]{0.50,0.00,0.50}{##1}}}
\expandafter\def\csname PY@tok@mb\endcsname{\def\PY@tc##1{\textcolor[rgb]{0.40,0.40,0.40}{##1}}}
\expandafter\def\csname PY@tok@sd\endcsname{\let\PY@it=\textit\def\PY@tc##1{\textcolor[rgb]{0.73,0.13,0.13}{##1}}}
\expandafter\def\csname PY@tok@ch\endcsname{\let\PY@it=\textit\def\PY@tc##1{\textcolor[rgb]{0.25,0.50,0.50}{##1}}}

\def\PYZbs{\char`\\}
\def\PYZus{\char`\_}
\def\PYZob{\char`\{}
\def\PYZcb{\char`\}}
\def\PYZca{\char`\^}
\def\PYZam{\char`\&}
\def\PYZlt{\char`\<}
\def\PYZgt{\char`\>}
\def\PYZsh{\char`\#}
\def\PYZpc{\char`\%}
\def\PYZdl{\char`\$}
\def\PYZhy{\char`\-}
\def\PYZsq{\char`\'}
\def\PYZdq{\char`\"}
\def\PYZti{\char`\~}
% for compatibility with earlier versions
\def\PYZat{@}
\def\PYZlb{[}
\def\PYZrb{]}
\makeatother


    % Exact colors from NB
    \definecolor{incolor}{rgb}{0.0, 0.0, 0.5}
    \definecolor{outcolor}{rgb}{0.545, 0.0, 0.0}



    
    % Prevent overflowing lines due to hard-to-break entities
    \sloppy 
    % Setup hyperref package
    \hypersetup{
      breaklinks=true,  % so long urls are correctly broken across lines
      colorlinks=true,
      urlcolor=urlcolor,
      linkcolor=linkcolor,
      citecolor=citecolor,
      }
    % Slightly bigger margins than the latex defaults
    
    \geometry{verbose,tmargin=1in,bmargin=1in,lmargin=1in,rmargin=1in}
    
    

    \begin{document}
    
    
    \maketitle
    
    

    
    \section{4. Complex oscillations}\label{complex-oscillations}

The most common use of complex numbers in physics is for analyzing
oscillations and waves. We will illustrate this with a simple but
crucially important model, the \textbf{damped harmonic oscillator}.

\subsection{The harmonic oscillator
equation}\label{the-harmonic-oscillator-equation}

The damped harmonic oscillator describes a mechanical system consisting
of a particle of mass \(m\), subject to a spring force and a damping
force. The particle can move along one dimension, and we let \(x(t)\)
denote its displacement from the origin. The damping coefficient is
\(2m \gamma\), and the spring constant is \(k = m\omega_0^2\). The
parameters \(m\), \(\gamma\), and \(\omega_0\) are all positive real
numbers. (The quantity \(\omega_0\) is called the ``natural frequency of
oscillation'', because in the absence of the damping force this system
would act as a simple harmonic oscillator with frequency \(\omega_0\).)

The motion of the particle is described by Newton's second law:

\[m \frac{d^2 x}{dt^2} = F(x,t) = - 2m\gamma \frac{dx}{dt} - m\omega_0^2 x(t).\]

Dividing by the common factor of \(m\), and bringing everything to one
side, gives

\[\frac{d^2 x}{dt^2} + 2\gamma \frac{dx}{dt} + \omega_0^2 x(t) = 0.\]

We call this ordinary differential equation the ``damped harmonic
oscillator equation''. Since it's a second-order
\href{01_derivatives.ipynb\#ODE}{ordinary differential equation} (ODE),
the general solution must contain two independent parameters. If we
state the initial displacement and velocity, \(x(0)\) and
\(\dot{x}(0)\), there is a unique specific solution.

Note

Sometimes, we write the damped harmonic oscillator equation a bit
differently:\[\left[\frac{d^2}{dt^2} + 2\gamma \frac{d}{dt} + \omega_0^2 \right]\, x(t) = 0.\]The
quantity in the square brackets is regarded as an operator acting on
\(x(t)\). This operator consists of the sum of three terms: a
second-derivative operator, a constant times a first derivative, and
multiplication by a constant.

We are interested in solving for \(x(t)\). Because of the damping force,
the oscillator loses energy as it moves, and for
\(t \rightarrow \infty\) it approaches the equilibrium state \(x = 0\).
A typical solution is shown below:

    \begin{Verbatim}[commandchars=\\\{\}]
{\color{incolor}In [{\color{incolor}1}]:} \PY{o}{\PYZpc{}}\PY{k}{matplotlib} inline
        \PY{k+kn}{from} \PY{n+nn}{numpy} \PY{k}{import} \PY{n}{linspace}\PY{p}{,} \PY{n}{exp}\PY{p}{,} \PY{n}{cos}\PY{p}{,} \PY{n}{sqrt}
        \PY{k+kn}{import} \PY{n+nn}{matplotlib}\PY{n+nn}{.}\PY{n+nn}{pyplot} \PY{k}{as} \PY{n+nn}{plt}
        \PY{k+kn}{from} \PY{n+nn}{matplotlib} \PY{k}{import} \PY{n}{animation}\PY{p}{,} \PY{n}{rc}
        \PY{k+kn}{from} \PY{n+nn}{IPython}\PY{n+nn}{.}\PY{n+nn}{display} \PY{k}{import} \PY{n}{HTML}
        
        \PY{n}{rc}\PY{p}{(}\PY{l+s+s1}{\PYZsq{}}\PY{l+s+s1}{animation}\PY{l+s+s1}{\PYZsq{}}\PY{p}{,} \PY{n}{html}\PY{o}{=}\PY{l+s+s1}{\PYZsq{}}\PY{l+s+s1}{html5}\PY{l+s+s1}{\PYZsq{}}\PY{p}{)}
        
        \PY{k}{def} \PY{n+nf}{oscillation\PYZus{}animation}\PY{p}{(}\PY{p}{)}\PY{p}{:}
            \PY{n}{amplitude}\PY{p}{,} \PY{n}{gamma}\PY{p}{,} \PY{n}{omega0} \PY{o}{=} \PY{l+m+mf}{1.0}\PY{p}{,} \PY{l+m+mf}{0.1}\PY{p}{,} \PY{l+m+mf}{1.0}
            \PY{n}{tmin}\PY{p}{,} \PY{n}{tmax}\PY{p}{,} \PY{n}{nt} \PY{o}{=} \PY{l+m+mf}{0.}\PY{p}{,} \PY{l+m+mf}{50.}\PY{p}{,} \PY{l+m+mi}{200}
            \PY{n}{nframes}\PY{p}{,} \PY{n}{frame\PYZus{}dt} \PY{o}{=} \PY{l+m+mi}{100}\PY{p}{,} \PY{l+m+mi}{50}
            \PY{n}{tmin\PYZus{}plt}\PY{p}{,} \PY{n}{xlim} \PY{o}{=} \PY{o}{\PYZhy{}}\PY{l+m+mi}{5}\PY{p}{,} \PY{l+m+mf}{1.2} \PY{c+c1}{\PYZsh{} Axis limits}
            \PY{n}{circ\PYZus{}pos} \PY{o}{=} \PY{o}{\PYZhy{}}\PY{l+m+mi}{2}
        
            \PY{n}{fig} \PY{o}{=} \PY{n}{plt}\PY{o}{.}\PY{n}{figure}\PY{p}{(}\PY{n}{figsize}\PY{o}{=}\PY{p}{(}\PY{l+m+mi}{10}\PY{p}{,}\PY{l+m+mi}{5}\PY{p}{)}\PY{p}{)}
            \PY{n}{ax}  \PY{o}{=} \PY{n}{plt}\PY{o}{.}\PY{n}{subplot}\PY{p}{(}\PY{l+m+mi}{1}\PY{p}{,}\PY{l+m+mi}{1}\PY{p}{,}\PY{l+m+mi}{1}\PY{p}{)}
            \PY{n}{ax}\PY{o}{.}\PY{n}{set\PYZus{}xlim}\PY{p}{(}\PY{n}{tmin\PYZus{}plt}\PY{p}{,} \PY{n}{tmax}\PY{p}{)}
            \PY{n}{ax}\PY{o}{.}\PY{n}{set\PYZus{}ylim}\PY{p}{(}\PY{o}{\PYZhy{}}\PY{n}{xlim}\PY{p}{,} \PY{n}{xlim}\PY{p}{)}
        
            \PY{n}{t} \PY{o}{=} \PY{n}{linspace}\PY{p}{(}\PY{n}{tmin}\PY{p}{,} \PY{n}{tmax}\PY{p}{,} \PY{n}{nt}\PY{p}{)}
            \PY{n}{x} \PY{o}{=} \PY{n}{amplitude} \PY{o}{*} \PY{n}{exp}\PY{p}{(}\PY{o}{\PYZhy{}}\PY{n}{gamma}\PY{o}{*}\PY{n}{t}\PY{p}{)} \PY{o}{*} \PY{n}{cos}\PY{p}{(}\PY{n}{sqrt}\PY{p}{(}\PY{n}{omega0}\PY{o}{*}\PY{o}{*}\PY{l+m+mi}{2} \PY{o}{\PYZhy{}} \PY{n}{gamma}\PY{o}{*}\PY{o}{*}\PY{l+m+mi}{2}\PY{p}{)}\PY{o}{*}\PY{n}{t}\PY{p}{)}
            \PY{n}{plt}\PY{o}{.}\PY{n}{xlabel}\PY{p}{(}\PY{l+s+s1}{\PYZsq{}}\PY{l+s+s1}{t}\PY{l+s+s1}{\PYZsq{}}\PY{p}{)}
            \PY{n}{plt}\PY{o}{.}\PY{n}{ylabel}\PY{p}{(}\PY{l+s+s1}{\PYZsq{}}\PY{l+s+s1}{x}\PY{l+s+s1}{\PYZsq{}}\PY{p}{)}
            \PY{n}{plt}\PY{o}{.}\PY{n}{title}\PY{p}{(}\PY{l+s+s1}{\PYZsq{}}\PY{l+s+s1}{Motion of a damped harmonic oscillator.}\PY{l+s+s1}{\PYZsq{}}\PY{p}{)}
            \PY{n}{ax}\PY{o}{.}\PY{n}{plot}\PY{p}{(}\PY{n}{t}\PY{p}{,} \PY{n}{x}\PY{p}{,} \PY{n}{color}\PY{o}{=}\PY{l+s+s1}{\PYZsq{}}\PY{l+s+s1}{blue}\PY{l+s+s1}{\PYZsq{}}\PY{p}{,} \PY{n}{linewidth}\PY{o}{=}\PY{l+m+mi}{2}\PY{p}{)}
            \PY{n}{line}\PY{p}{,} \PY{o}{=} \PY{n}{ax}\PY{o}{.}\PY{n}{plot}\PY{p}{(}\PY{p}{[}\PY{p}{]}\PY{p}{,} \PY{p}{[}\PY{p}{]}\PY{p}{,} \PY{n}{color}\PY{o}{=}\PY{l+s+s1}{\PYZsq{}}\PY{l+s+s1}{grey}\PY{l+s+s1}{\PYZsq{}}\PY{p}{,} \PY{n}{linewidth}\PY{o}{=}\PY{l+m+mi}{2}\PY{p}{)}
            \PY{n}{circ}\PY{p}{,} \PY{o}{=} \PY{n}{ax}\PY{o}{.}\PY{n}{plot}\PY{p}{(}\PY{p}{[}\PY{p}{]}\PY{p}{,} \PY{p}{[}\PY{p}{]}\PY{p}{,} \PY{l+s+s1}{\PYZsq{}}\PY{l+s+s1}{o}\PY{l+s+s1}{\PYZsq{}}\PY{p}{,} \PY{n}{color}\PY{o}{=}\PY{l+s+s1}{\PYZsq{}}\PY{l+s+s1}{red}\PY{l+s+s1}{\PYZsq{}}\PY{p}{,} \PY{n}{markersize}\PY{o}{=}\PY{l+m+mi}{15}\PY{p}{)}
            \PY{n}{dash}\PY{p}{,} \PY{o}{=} \PY{n}{ax}\PY{o}{.}\PY{n}{plot}\PY{p}{(}\PY{p}{[}\PY{p}{]}\PY{p}{,} \PY{p}{[}\PY{p}{]}\PY{p}{,} \PY{l+s+s1}{\PYZsq{}}\PY{l+s+s1}{\PYZhy{}\PYZhy{}}\PY{l+s+s1}{\PYZsq{}}\PY{p}{,} \PY{n}{color}\PY{o}{=}\PY{l+s+s1}{\PYZsq{}}\PY{l+s+s1}{grey}\PY{l+s+s1}{\PYZsq{}}\PY{p}{,} \PY{n}{markersize}\PY{o}{=}\PY{l+m+mi}{15}\PY{p}{)}
            \PY{n}{plt}\PY{o}{.}\PY{n}{close}\PY{p}{(}\PY{p}{)}\PY{p}{;}
        
            \PY{c+c1}{\PYZsh{} initialization function: plot the background of each frame}
            \PY{k}{def} \PY{n+nf}{init}\PY{p}{(}\PY{p}{)}\PY{p}{:}
                \PY{n}{line}\PY{o}{.}\PY{n}{set\PYZus{}data}\PY{p}{(}\PY{p}{[}\PY{p}{]}\PY{p}{,} \PY{p}{[}\PY{p}{]}\PY{p}{)}
                \PY{n}{circ}\PY{o}{.}\PY{n}{set\PYZus{}data}\PY{p}{(}\PY{p}{[}\PY{p}{]}\PY{p}{,} \PY{p}{[}\PY{p}{]}\PY{p}{)}
                \PY{n}{dash}\PY{o}{.}\PY{n}{set\PYZus{}data}\PY{p}{(}\PY{p}{[}\PY{p}{]}\PY{p}{,} \PY{p}{[}\PY{p}{]}\PY{p}{)}
                \PY{k}{return} \PY{n}{line}\PY{p}{,} \PY{n}{circ}\PY{p}{,} \PY{n}{dash}
        
            \PY{c+c1}{\PYZsh{} animation function. This is called sequentially}
            \PY{k}{def} \PY{n+nf}{animate}\PY{p}{(}\PY{n}{n}\PY{p}{)}\PY{p}{:}
                \PY{n}{t} \PY{o}{=} \PY{n}{tmin} \PY{o}{+} \PY{p}{(}\PY{n}{tmax}\PY{o}{\PYZhy{}}\PY{n}{tmin}\PY{p}{)}\PY{o}{*}\PY{n}{n}\PY{o}{/}\PY{n}{nframes}
                \PY{n}{line}\PY{o}{.}\PY{n}{set\PYZus{}data}\PY{p}{(}\PY{p}{[}\PY{n}{t}\PY{p}{,} \PY{n}{t}\PY{p}{]}\PY{p}{,} \PY{p}{[}\PY{o}{\PYZhy{}}\PY{n}{xlim}\PY{p}{,} \PY{n}{xlim}\PY{p}{]}\PY{p}{)}        
                \PY{n}{xc} \PY{o}{=} \PY{n}{amplitude} \PY{o}{*} \PY{n}{exp}\PY{p}{(}\PY{o}{\PYZhy{}}\PY{n}{gamma}\PY{o}{*}\PY{n}{t}\PY{p}{)} \PY{o}{*} \PY{n}{cos}\PY{p}{(}\PY{n}{sqrt}\PY{p}{(}\PY{n}{omega0}\PY{o}{*}\PY{o}{*}\PY{l+m+mi}{2} \PY{o}{\PYZhy{}} \PY{n}{gamma}\PY{o}{*}\PY{o}{*}\PY{l+m+mi}{2}\PY{p}{)}\PY{o}{*}\PY{n}{t}\PY{p}{)}
                \PY{n}{circ}\PY{o}{.}\PY{n}{set\PYZus{}data}\PY{p}{(}\PY{n}{circ\PYZus{}pos}\PY{p}{,} \PY{n}{xc}\PY{p}{)}
                \PY{n}{dash}\PY{o}{.}\PY{n}{set\PYZus{}data}\PY{p}{(}\PY{p}{[}\PY{n}{circ\PYZus{}pos}\PY{p}{,} \PY{n}{t}\PY{p}{]}\PY{p}{,} \PY{p}{[}\PY{n}{xc}\PY{p}{,} \PY{n}{xc}\PY{p}{]}\PY{p}{)}
                \PY{k}{return} \PY{n}{line}\PY{p}{,} \PY{n}{circ}\PY{p}{,} \PY{n}{dash}
        
            \PY{c+c1}{\PYZsh{} call the animator. blit=True means only re\PYZhy{}draw the parts that have changed.}
            \PY{n}{animator} \PY{o}{=} \PY{n}{animation}\PY{o}{.}\PY{n}{FuncAnimation}\PY{p}{(}\PY{n}{fig}\PY{p}{,} \PY{n}{animate}\PY{p}{,} \PY{n}{init\PYZus{}func}\PY{o}{=}\PY{n}{init}\PY{p}{,}
                                               \PY{n}{frames}\PY{o}{=}\PY{n}{nframes}\PY{p}{,} \PY{n}{interval}\PY{o}{=}\PY{n}{frame\PYZus{}dt}\PY{p}{,} \PY{n}{blit}\PY{o}{=}\PY{k+kc}{True}\PY{p}{)}
            \PY{k}{return} \PY{n}{animator}
        
        \PY{n}{oscillation\PYZus{}animation}\PY{p}{(}\PY{p}{)}
\end{Verbatim}

            \begin{Verbatim}[commandchars=\\\{\}]
{\color{outcolor}Out[{\color{outcolor}1}]:} <matplotlib.animation.FuncAnimation at 0x7fa74a490c88>
\end{Verbatim}
        
    \subsection{Complex solution}\label{complex-solution}

The variable \(x(t)\) stands for the displacement of a particle, which
is a real quantity. But in order to solve the damped harmonic oscillator
equation, it's useful if we generalize \(x(t)\) to complex values. In
other words, let's treat the harmonic oscillator equation as a
`'complex'' ODE:

\[\frac{d^2 z}{dt^2} + 2\gamma \frac{dz}{dt} + \omega_0^2 z(t) = 0, \quad z(t) \in \mathbb{C}.\]

The same parameter-counting rule applies to complex ODEs just the same
as real ODEs, except that we use complex parameters in place of real
parameters. Since the complex damped harmonic oscillator equation is
second-order, its general solution should contain two independent
`'complex'' parameters---which is equivalent to four real parameters.

Once we have that general solution, we can do one of two things: (i)
plug in a complete set of (real) boundary conditions, which will give a
`'real'`specific solution, or (ii) take the real part of the complex
general solution, which will give the general solution to the'`real''
differential equation. We will discuss these two approaches later; for
the moment, let's focus on finding the solution to the complex ODE.

To find the complex solution, first note that the equation is
`'linear''. If we have two solutions \(z_1(t)\) and \(z_2(t)\), then any
combination

\[\alpha \, z_1(t) + \beta \,z_2(t),\quad \mathrm{where}\;\; \alpha, \beta \in \mathbb{C}\]

is also a solution. Therefore, a good strategy is to find several
specific solutions, and then combine them linearly to form a more
general solution. We simply make a guess (or an \textbf{ansatz}) for a
specific solution:

\[z(t) = e^{-i\omega t},\]

where \(\omega\) is a constant to be determined (which could be
complex). The first and second derivatives are:

\[\begin{align}\frac{dz}{dt} &= -i\omega\, e^{-i\omega t} \\ \frac{d^2z}{dt^2} &= -\omega^2\, e^{-i\omega t}\end{align}\]

Substituting these into the differential equation gives:

\[\left[-\omega^2 - 2i\gamma \omega + \omega_0^2 \right] e^{-i\omega t} = 0.\]

This equation can be satisfied for all \(t\) if the complex second-order
polynomial on the left-hand side is zero:

\[-\omega^2 - 2i\gamma \omega + \omega_0^2 = 0.\]

In other words, we need values of \(\omega\) which solve this quadratic
equation. The solutions can be obtained from the quadratic formula:

\[\omega = -i\gamma \pm \sqrt{\omega_0^2 - \gamma^2}.\]

Hence, we arrive at solutions which are oscillations with `'complex''
frequencies:

\[z(t) = \exp\left(-i\omega_\pm t\right), \;\;\mathrm{where}\;\; \omega_\pm = -i\gamma \pm \sqrt{\omega_0^2 - \gamma^2}.\]

For each value of \(\gamma\) and \(\omega_0\), there are two possible
frequencies, \(\omega_+\) and \(\omega_-\). For either choice of complex
frequency, the above expression for \(x(t)\) gives a valid specific
solution for the complex damped harmonic oscillator equation.

    \subsection{Complex frequencies}\label{complex-frequencies}

What does it mean to have an oscillation with a complex frequency? If we
write the real and imaginary parts of the frequency as
\(\omega = \omega_R + i \omega_I\), then

\[z(t) = e^{-i\omega t} = e^{\omega_I t} \; e^{-i\omega_R t}.\]

If both \(\omega_R\) and \(\omega_I\) are non-zero, this describes
\href{03_complex_numbers.ipynb\#complex_spiral}{a spiral trajectory in
the complex plane}, whose magnitude is either increasing or decreasing
with time, depending on the sign of \(\omega_O\). This is because we can
write

\[z(t) = e^{\omega_I t} \; e^{-i\omega_R t} = R(t)\, e^{i\theta(t)}, \;\;\mathrm{where}\;\;R(t) = e^{\omega_I t}, \; \theta(t) = -\omega_R t.\]

We therefore conclude that the real part of \(\omega\) determines the
(angular) frequency of oscillation, whereas the imaginary part
determines whether the oscillation amplitude is either growing with time
(amplification) or shrinking with time (damping). A positive imaginary
part implies amplification, and a negative imaginary part implies
damping, while zero imaginary part (i.e., a real frequency) implies
constant-amplitude oscillation.

Now let's look at the damped harmonic oscillator's complex frequencies,
\(\omega_\pm\), which we derived in the
\protect\hyperlink{ansatz}{previous section}:

\[\omega_\pm = -i\gamma \pm \sqrt{\omega_0^2 - \gamma^2}.\]

These depend on two real parameters: \(\gamma\) and \(\omega_0\). In the
plot below, you can see how the position of \(\omega_\pm\) in the
complex plane depends on the values of these parameters.

    \begin{Verbatim}[commandchars=\\\{\}]
{\color{incolor}In [{\color{incolor}2}]:} \PY{o}{\PYZpc{}}\PY{k}{matplotlib} inline
        \PY{k+kn}{from} \PY{n+nn}{ipywidgets} \PY{k}{import} \PY{n}{interact}\PY{p}{,} \PY{n}{FloatSlider}
        \PY{k+kn}{from} \PY{n+nn}{numpy} \PY{k}{import} \PY{n}{linspace}\PY{p}{,} \PY{n}{sin}\PY{p}{,} \PY{n}{cos}\PY{p}{,} \PY{n}{pi}\PY{p}{,} \PY{n}{sqrt}
        \PY{k+kn}{import} \PY{n+nn}{matplotlib}\PY{n+nn}{.}\PY{n+nn}{pyplot} \PY{k}{as} \PY{n+nn}{plt}
        
        \PY{k}{def} \PY{n+nf}{plot\PYZus{}frequencies}\PY{p}{(}\PY{n}{omega0}\PY{p}{,} \PY{n}{gamma}\PY{p}{)}\PY{p}{:}
            \PY{c+c1}{\PYZsh{}\PYZsh{} (Plot the circular arc that the points lie on.)}
            \PY{n}{phi} \PY{o}{=} \PY{n}{linspace}\PY{p}{(}\PY{o}{\PYZhy{}}\PY{n}{pi}\PY{p}{,} \PY{l+m+mi}{0}\PY{p}{,} \PY{l+m+mi}{50}\PY{p}{)}
            \PY{n}{plt}\PY{o}{.}\PY{n}{figure}\PY{p}{(}\PY{n}{figsize}\PY{o}{=}\PY{p}{(}\PY{l+m+mi}{10}\PY{p}{,}\PY{l+m+mi}{5}\PY{p}{)}\PY{p}{)}
            \PY{n}{plt}\PY{o}{.}\PY{n}{plot}\PY{p}{(}\PY{n}{omega0}\PY{o}{*}\PY{n}{cos}\PY{p}{(}\PY{n}{phi}\PY{p}{)}\PY{p}{,} \PY{n}{omega0}\PY{o}{*}\PY{n}{sin}\PY{p}{(}\PY{n}{phi}\PY{p}{)}\PY{p}{,} \PY{n}{color}\PY{o}{=}\PY{l+s+s2}{\PYZdq{}}\PY{l+s+s2}{grey}\PY{l+s+s2}{\PYZdq{}}\PY{p}{)}
            \PY{n}{rmax} \PY{o}{=} \PY{l+m+mf}{1.7}
        
            \PY{c+c1}{\PYZsh{}\PYZsh{} Compute omega\PYZus{}+ and omega\PYZus{}\PYZhy{} and plot in the complex plane.}
            \PY{n}{u} \PY{o}{=} \PY{n}{sqrt}\PY{p}{(}\PY{n+nb}{abs}\PY{p}{(}\PY{n}{omega0}\PY{o}{*}\PY{o}{*}\PY{l+m+mi}{2} \PY{o}{\PYZhy{}} \PY{n}{gamma}\PY{o}{*}\PY{o}{*}\PY{l+m+mi}{2}\PY{p}{)}\PY{p}{)}
            \PY{k}{if} \PY{n}{omega0} \PY{o}{\PYZgt{}}\PY{o}{=} \PY{n}{gamma}\PY{p}{:}
                \PY{n}{w1}\PY{p}{,} \PY{n}{w2} \PY{o}{=} \PY{o}{\PYZhy{}}\PY{l+m+mi}{1}\PY{n}{j} \PY{o}{*} \PY{n}{gamma} \PY{o}{+} \PY{n}{u}\PY{p}{,} \PY{o}{\PYZhy{}}\PY{l+m+mi}{1}\PY{n}{j} \PY{o}{*} \PY{n}{gamma} \PY{o}{\PYZhy{}} \PY{n}{u}
            \PY{k}{else}\PY{p}{:}
                \PY{n}{w1}\PY{p}{,} \PY{n}{w2} \PY{o}{=} \PY{o}{\PYZhy{}}\PY{l+m+mi}{1}\PY{n}{j} \PY{o}{*} \PY{p}{(}\PY{n}{gamma} \PY{o}{\PYZhy{}} \PY{n}{u}\PY{p}{)}\PY{p}{,} \PY{o}{\PYZhy{}}\PY{l+m+mi}{1}\PY{n}{j} \PY{o}{*} \PY{p}{(}\PY{n}{gamma} \PY{o}{+} \PY{n}{u}\PY{p}{)}
            
            \PY{n}{ax} \PY{o}{=} \PY{n}{plt}\PY{o}{.}\PY{n}{subplot}\PY{p}{(}\PY{l+m+mi}{1}\PY{p}{,}\PY{l+m+mi}{1}\PY{p}{,}\PY{l+m+mi}{1}\PY{p}{)}
            \PY{n}{plt}\PY{o}{.}\PY{n}{plot}\PY{p}{(}\PY{n}{w1}\PY{o}{.}\PY{n}{real}\PY{p}{,} \PY{n}{w1}\PY{o}{.}\PY{n}{imag}\PY{p}{,} \PY{l+s+s1}{\PYZsq{}}\PY{l+s+s1}{o}\PY{l+s+s1}{\PYZsq{}}\PY{p}{,} \PY{n}{color}\PY{o}{=}\PY{l+s+s2}{\PYZdq{}}\PY{l+s+s2}{red}\PY{l+s+s2}{\PYZdq{}}\PY{p}{,} \PY{n}{linewidth}\PY{o}{=}\PY{l+m+mi}{4}\PY{p}{,} \PY{n}{label}\PY{o}{=}\PY{l+s+s2}{\PYZdq{}}\PY{l+s+s2}{omega\PYZus{}+}\PY{l+s+s2}{\PYZdq{}}\PY{p}{)}
            \PY{n}{plt}\PY{o}{.}\PY{n}{plot}\PY{p}{(}\PY{n}{w2}\PY{o}{.}\PY{n}{real}\PY{p}{,} \PY{n}{w2}\PY{o}{.}\PY{n}{imag}\PY{p}{,} \PY{l+s+s1}{\PYZsq{}}\PY{l+s+s1}{o}\PY{l+s+s1}{\PYZsq{}}\PY{p}{,} \PY{n}{color}\PY{o}{=}\PY{l+s+s2}{\PYZdq{}}\PY{l+s+s2}{blue}\PY{l+s+s2}{\PYZdq{}}\PY{p}{,} \PY{n}{linewidth}\PY{o}{=}\PY{l+m+mi}{4}\PY{p}{,} \PY{n}{label}\PY{o}{=}\PY{l+s+s2}{\PYZdq{}}\PY{l+s+s2}{omega\PYZus{}\PYZhy{}}\PY{l+s+s2}{\PYZdq{}}\PY{p}{)}
            \PY{n}{plt}\PY{o}{.}\PY{n}{title}\PY{p}{(}\PY{l+s+s1}{\PYZsq{}}\PY{l+s+s1}{Complex plane positions of omega\PYZus{}+ and omega\PYZus{}\PYZhy{}}\PY{l+s+s1}{\PYZsq{}}\PY{p}{)}
            \PY{n}{plt}\PY{o}{.}\PY{n}{xlabel}\PY{p}{(}\PY{l+s+s1}{\PYZsq{}}\PY{l+s+s1}{Re(omega)}\PY{l+s+s1}{\PYZsq{}}\PY{p}{)}
            \PY{n}{plt}\PY{o}{.}\PY{n}{ylabel}\PY{p}{(}\PY{l+s+s1}{\PYZsq{}}\PY{l+s+s1}{Im(omega)}\PY{l+s+s1}{\PYZsq{}}\PY{p}{)}
            \PY{n}{plt}\PY{o}{.}\PY{n}{xlim}\PY{p}{(}\PY{o}{\PYZhy{}}\PY{n}{rmax}\PY{p}{,} \PY{n}{rmax}\PY{p}{)}
            \PY{n}{plt}\PY{o}{.}\PY{n}{ylim}\PY{p}{(}\PY{n+nb}{min}\PY{p}{(}\PY{o}{\PYZhy{}}\PY{n}{rmax}\PY{p}{,} \PY{l+m+mf}{1.1}\PY{o}{*}\PY{n}{w2}\PY{o}{.}\PY{n}{imag}\PY{p}{)}\PY{p}{,} \PY{l+m+mi}{0}\PY{p}{)}
            \PY{n}{plt}\PY{o}{.}\PY{n}{axes}\PY{p}{(}\PY{p}{)}\PY{o}{.}\PY{n}{set\PYZus{}aspect}\PY{p}{(}\PY{l+s+s1}{\PYZsq{}}\PY{l+s+s1}{equal}\PY{l+s+s1}{\PYZsq{}}\PY{p}{)}
            \PY{n}{ax}\PY{o}{.}\PY{n}{legend}\PY{p}{(}\PY{n}{numpoints}\PY{o}{=}\PY{l+m+mi}{1}\PY{p}{,} \PY{n}{loc}\PY{o}{=}\PY{l+s+s1}{\PYZsq{}}\PY{l+s+s1}{lower right}\PY{l+s+s1}{\PYZsq{}}\PY{p}{)}
        
        \PY{n}{interact}\PY{p}{(}\PY{n}{plot\PYZus{}frequencies}\PY{p}{,}
                 \PY{n}{omega0} \PY{o}{=} \PY{n}{FloatSlider}\PY{p}{(}\PY{n+nb}{min}\PY{o}{=}\PY{l+m+mf}{0.5}\PY{p}{,} \PY{n+nb}{max}\PY{o}{=}\PY{l+m+mf}{1.5}\PY{p}{,} \PY{n}{step}\PY{o}{=}\PY{l+m+mf}{0.05}\PY{p}{,} \PY{n}{value}\PY{o}{=}\PY{l+m+mf}{1.0}\PY{p}{)}\PY{p}{,}
                 \PY{n}{gamma}  \PY{o}{=} \PY{n}{FloatSlider}\PY{p}{(}\PY{n+nb}{min}\PY{o}{=}\PY{l+m+mf}{0.0}\PY{p}{,} \PY{n+nb}{max}\PY{o}{=}\PY{l+m+mf}{2.0}\PY{p}{,} \PY{n}{step}\PY{o}{=}\PY{l+m+mf}{0.05}\PY{p}{,} \PY{n}{value}\PY{o}{=}\PY{l+m+mf}{0.1}\PY{p}{)}\PY{p}{)}\PY{p}{;}
\end{Verbatim}

    \begin{center}
    \adjustimage{max size={0.9\linewidth}{0.9\paperheight}}{04_complex_oscillations_files/04_complex_oscillations_4_0.png}
    \end{center}
    { \hspace*{\fill} \\}
    
    In particular, note the following features:

\begin{itemize}
\tightlist
\item
  For \(\gamma = 0\) (zero damping), the two frequencies are both real,
  and take the values \(\pm \omega_0\). This corresponds to undamped (or
  ``simple'') harmonic oscillation at the oscillator's natural
  frequency.
\item
  If we increase \(\gamma\) from zero with \(\omega_0\) fixed, both
  \(\omega_+\) and \(\omega_-\) move downwards in the complex plane,
  along a circular arc. Because the imaginary part of the frequencies
  are negative, this implies damped oscillation.
\item
  At \(\gamma = \omega_0\), the frequencies meet along the imaginary
  axis. We will have more to say about this case
  \protect\hyperlink{critical_damping}{later}.
\item
  For \(\gamma > \omega_0\), the two frequencies move apart along the
  imaginary axis. Purely imaginary frequencies correspond to a
  trajectory that simply decays without oscillating. We'll say more
  about this kind of motion \protect\hyperlink{overdamped}{later}.
\end{itemize}

    \subsection{General solution for the damped harmonic
oscillator}\label{general-solution-for-the-damped-harmonic-oscillator}

First, suppose \(\omega_0 \ne \gamma\). In the previous section, we
found two classes of specific solutions, with complex frequencies
\(\omega_+\) and \(\omega_-\):

\[z_+(t) = e^{-i\omega_+ t} \;\;\mathrm{and}\;\; z_-(t) = e^{-i\omega_- t}, \;\;\mathrm{where}\;\;\; \omega_\pm = -i\gamma \pm \sqrt{\omega_0^2 - \gamma^2}.\]

We can write down a more general solution consisting of a linear
superposition of these specific solutions:

\[\begin{aligned}z(t) &= \psi_+ e^{-i\omega_+ t} + \psi_- e^{-i\omega_- t} \\ &= \psi_+ \, \exp\left[\left(-\gamma  - i \sqrt{\omega_0^2 - \gamma^2}\right)t\right] \; +\; \psi_- \, \exp\left[\left(-\gamma +i\sqrt{\omega_0^2 - \gamma^2}\right)t\right].\end{aligned}\]

This contains two undetermined complex parameters, \(\psi_+\) and
\(\psi_-\). These are \emph{independent} parameters since they are the
coefficients that multiply different functions (the functions are
different because \(\omega_0 \ne \gamma\) implies that
\(\omega_+ \ne \omega_-\)). Hence, the above equation for \(z(t)\) is a
general solution for the complex damped harmonic oscillator equation.

To obtain the general solution to the `'real'' damped harmonic
oscillator equation, we have to take the real part of the complex
solution. The result can be further simplified depending on whether the
term \(\omega_0^2 - \gamma^2\) in the formula is positive or negative.
This leads to what are called \textbf{under-damped solutions} and
\textbf{over-damped solutions}, to be discussed in the following
subsections.

What about if \(\omega_0 = \gamma\)? In this instance,
\(\omega_+ = \omega_-\), which means that \(\psi_+\) and \(\psi_-\)
aren't independent parameters. Therefore, the above equation for
\(z(t)\) isn't a valid general solution in this particular case!
Instead, the general solution is something called a
\textbf{critically-damped solution}, which we will
\protect\hyperlink{critical_damping}{discuss later}.

    \subsubsection{Under-damped motion}\label{under-damped-motion}

For \(\omega_0 > \gamma\), let us define, for convenience,

\[\Omega = \sqrt{\omega_0^2 - \gamma^2}.\]

Then we can simplify the real solution as follows:

\[\begin{align}x(t) &= \mathrm{Re}\left[z(t)\right] \\ &= e^{-\gamma t} \; \mathrm{Re}\left[\psi_+ \, e^{-i \Omega t} \,+\, \psi_- \, e^{i\Omega t}\right] \\ &= e^{-\gamma t} \left[ A\cos\left(\Omega t\right) + B \sin\left(\Omega t\right)\right], \;\;\mathrm{where}\;\; A, B \in \mathbb{R}\end{align}\]

With a bit of algebra, we can show that

\[A = \mathrm{Re}\left[\psi_+ + \psi_-\right], \quad B = \mathrm{Im}\left[\psi_+ - \psi_-\right].\]

The coefficients \(A\) and \(B\) act as two independent `'real''
parameters, so this is a valid general solution for the real damped
harmonic oscillator equation. Using the trigonometric formulas, the
solution can be equivalently written as

\[x(t) = C e^{-\gamma t} \cos\left[\Omega t + \Phi\right],\]

with the parameters \(C = \sqrt{A^2 + B^2}\) and
\(\Phi = - \tan^{-1}\left[B/A\right]\).

Either way, this is called an \textbf{under-damped solution}. As shown
below, the trajectory is an oscillation whose amplitude decreases with
time. The decrease in the amplitude can be visualized using a smooth
``envelope'' given by \(\pm C e^{-\gamma t}\), which is drawn with
dashes in the figure. Inside this envelope, the trajectory oscillates
with frequency \(\Omega = \sqrt{\omega_0^2 - \gamma^2}\), which is
slightly less than the natural frequency of oscillation \(\omega_0\).

    \begin{Verbatim}[commandchars=\\\{\}]
{\color{incolor}In [{\color{incolor}3}]:} \PY{o}{\PYZpc{}}\PY{k}{matplotlib} inline
        \PY{k+kn}{from} \PY{n+nn}{ipywidgets} \PY{k}{import} \PY{n}{interact}\PY{p}{,} \PY{n}{FloatSlider}
        \PY{k+kn}{from} \PY{n+nn}{numpy} \PY{k}{import} \PY{n}{linspace}\PY{p}{,} \PY{n}{cos}\PY{p}{,} \PY{n}{sqrt}\PY{p}{,} \PY{n}{exp}
        \PY{k+kn}{import} \PY{n+nn}{matplotlib}\PY{n+nn}{.}\PY{n+nn}{pyplot} \PY{k}{as} \PY{n+nn}{plt}
        
        \PY{k}{def} \PY{n+nf}{plot\PYZus{}underdamped}\PY{p}{(}\PY{n}{omega0}\PY{p}{,} \PY{n}{gamma}\PY{p}{,} \PY{n}{C}\PY{p}{,} \PY{n}{phi}\PY{p}{)}\PY{p}{:}
            \PY{n}{x0}\PY{p}{,} \PY{n}{t} \PY{o}{=} \PY{l+m+mf}{1.0}\PY{p}{,} \PY{n}{linspace}\PY{p}{(}\PY{l+m+mi}{0}\PY{p}{,} \PY{l+m+mi}{20}\PY{p}{,} \PY{l+m+mi}{200}\PY{p}{)}
            \PY{n}{xe} \PY{o}{=} \PY{n}{C} \PY{o}{*} \PY{n}{exp}\PY{p}{(}\PY{o}{\PYZhy{}}\PY{n}{gamma}\PY{o}{*}\PY{n}{t}\PY{p}{)}
            \PY{n}{x}  \PY{o}{=} \PY{n}{xe} \PY{o}{*} \PY{n}{cos}\PY{p}{(}\PY{n}{sqrt}\PY{p}{(}\PY{n}{omega0}\PY{o}{*}\PY{o}{*}\PY{l+m+mi}{2} \PY{o}{\PYZhy{}} \PY{n}{gamma}\PY{o}{*}\PY{o}{*}\PY{l+m+mi}{2}\PY{p}{)}\PY{o}{*}\PY{n}{t}\PY{o}{+}\PY{n}{phi}\PY{p}{)}
            
            \PY{n}{plt}\PY{o}{.}\PY{n}{figure}\PY{p}{(}\PY{n}{figsize}\PY{o}{=}\PY{p}{(}\PY{l+m+mi}{10}\PY{p}{,}\PY{l+m+mi}{5}\PY{p}{)}\PY{p}{)}
            \PY{n}{plt}\PY{o}{.}\PY{n}{title}\PY{p}{(}\PY{l+s+s1}{\PYZsq{}}\PY{l+s+s1}{Motion of an underdamped oscillator}\PY{l+s+s1}{\PYZsq{}}\PY{p}{)}
            \PY{n}{plt}\PY{o}{.}\PY{n}{plot}\PY{p}{(}\PY{n}{t}\PY{p}{,} \PY{n}{x}\PY{p}{,} \PY{l+s+s1}{\PYZsq{}}\PY{l+s+s1}{b}\PY{l+s+s1}{\PYZsq{}}\PY{p}{,} \PY{n}{label}\PY{o}{=}\PY{l+s+s1}{\PYZsq{}}\PY{l+s+s1}{x(t)}\PY{l+s+s1}{\PYZsq{}}\PY{p}{)}
            \PY{c+c1}{\PYZsh{}\PYZsh{} Plot envelope}
            \PY{n}{ax} \PY{o}{=} \PY{n}{plt}\PY{o}{.}\PY{n}{subplot}\PY{p}{(}\PY{l+m+mi}{1}\PY{p}{,}\PY{l+m+mi}{1}\PY{p}{,}\PY{l+m+mi}{1}\PY{p}{)}
            \PY{n}{plt}\PY{o}{.}\PY{n}{plot}\PY{p}{(}\PY{n}{t}\PY{p}{,} \PY{n}{xe}\PY{p}{,} \PY{l+s+s1}{\PYZsq{}}\PY{l+s+s1}{r\PYZhy{}\PYZhy{}}\PY{l+s+s1}{\PYZsq{}}\PY{p}{,} \PY{n}{label}\PY{o}{=}\PY{l+s+s1}{\PYZsq{}}\PY{l+s+s1}{Envelope}\PY{l+s+s1}{\PYZsq{}}\PY{p}{)}
            \PY{n}{plt}\PY{o}{.}\PY{n}{plot}\PY{p}{(}\PY{n}{t}\PY{p}{,} \PY{o}{\PYZhy{}}\PY{n}{xe}\PY{p}{,} \PY{l+s+s1}{\PYZsq{}}\PY{l+s+s1}{r\PYZhy{}\PYZhy{}}\PY{l+s+s1}{\PYZsq{}}\PY{p}{)}
            \PY{n}{plt}\PY{o}{.}\PY{n}{xlabel}\PY{p}{(}\PY{l+s+s1}{\PYZsq{}}\PY{l+s+s1}{t}\PY{l+s+s1}{\PYZsq{}}\PY{p}{)}
            \PY{n}{plt}\PY{o}{.}\PY{n}{ylabel}\PY{p}{(}\PY{l+s+s1}{\PYZsq{}}\PY{l+s+s1}{x}\PY{l+s+s1}{\PYZsq{}}\PY{p}{)}
            \PY{n}{plt}\PY{o}{.}\PY{n}{xlim}\PY{p}{(}\PY{n}{t}\PY{p}{[}\PY{l+m+mi}{0}\PY{p}{]}\PY{p}{,} \PY{n}{t}\PY{p}{[}\PY{o}{\PYZhy{}}\PY{l+m+mi}{1}\PY{p}{]}\PY{p}{)}
            \PY{n}{plt}\PY{o}{.}\PY{n}{ylim}\PY{p}{(}\PY{o}{\PYZhy{}}\PY{n}{x0}\PY{p}{,} \PY{n}{x0}\PY{p}{)}
            \PY{n}{ax}\PY{o}{.}\PY{n}{legend}\PY{p}{(}\PY{n}{numpoints}\PY{o}{=}\PY{l+m+mi}{1}\PY{p}{,} \PY{n}{loc}\PY{o}{=}\PY{l+s+s1}{\PYZsq{}}\PY{l+s+s1}{upper right}\PY{l+s+s1}{\PYZsq{}}\PY{p}{)}
            \PY{n}{plt}\PY{o}{.}\PY{n}{show}\PY{p}{(}\PY{p}{)}
        
        \PY{n}{interact}\PY{p}{(}\PY{n}{plot\PYZus{}underdamped}\PY{p}{,}
                 \PY{n}{omega0} \PY{o}{=} \PY{n}{FloatSlider}\PY{p}{(}\PY{n+nb}{min}\PY{o}{=}\PY{l+m+mf}{0.5}\PY{p}{,} \PY{n+nb}{max}\PY{o}{=}\PY{l+m+mf}{1.5}\PY{p}{,} \PY{n}{step}\PY{o}{=}\PY{l+m+mf}{0.02}\PY{p}{,} \PY{n}{value}\PY{o}{=}\PY{l+m+mf}{1.0}\PY{p}{)}\PY{p}{,}
                 \PY{n}{gamma}  \PY{o}{=} \PY{n}{FloatSlider}\PY{p}{(}\PY{n+nb}{min}\PY{o}{=}\PY{l+m+mf}{0.0}\PY{p}{,} \PY{n+nb}{max}\PY{o}{=}\PY{l+m+mf}{0.5}\PY{p}{,} \PY{n}{step}\PY{o}{=}\PY{l+m+mf}{0.02}\PY{p}{,} \PY{n}{value}\PY{o}{=}\PY{l+m+mf}{0.1}\PY{p}{)}\PY{p}{,}
                 \PY{n}{C}      \PY{o}{=} \PY{n}{FloatSlider}\PY{p}{(}\PY{n+nb}{min}\PY{o}{=}\PY{l+m+mf}{0.0}\PY{p}{,} \PY{n+nb}{max}\PY{o}{=}\PY{l+m+mf}{1.0}\PY{p}{,} \PY{n}{step}\PY{o}{=}\PY{l+m+mf}{0.05}\PY{p}{,} \PY{n}{value}\PY{o}{=}\PY{l+m+mf}{1.0}\PY{p}{)}\PY{p}{,}
                 \PY{n}{phi}    \PY{o}{=} \PY{n}{FloatSlider}\PY{p}{(}\PY{n+nb}{min}\PY{o}{=}\PY{o}{\PYZhy{}}\PY{l+m+mf}{3.14}\PY{p}{,} \PY{n+nb}{max}\PY{o}{=}\PY{l+m+mf}{3.14}\PY{p}{,} \PY{n}{step}\PY{o}{=}\PY{l+m+mf}{0.002}\PY{p}{,} \PY{n}{value}\PY{o}{=}\PY{l+m+mf}{0.0}\PY{p}{)}\PY{p}{)}\PY{p}{;}
\end{Verbatim}

    \begin{center}
    \adjustimage{max size={0.9\linewidth}{0.9\paperheight}}{04_complex_oscillations_files/04_complex_oscillations_8_0.png}
    \end{center}
    { \hspace*{\fill} \\}
    
    \subsubsection{Over-damped motion}\label{over-damped-motion}

For \(\omega_0 < \gamma\), the square root term becomes imaginary. It is
convenient to define

\[\Gamma = \sqrt{\gamma^2 - \omega_0^2} \quad \Rightarrow \quad \sqrt{\omega_0^2 - \gamma^2} = i \Gamma.\]

Then the real solution simplifies in a different way:

\[\begin{align} x(t) &= \mathrm{Re}\left[z(t)\right] \\&= \mathrm{Re}\left[\psi_+ e^{\left(-\gamma  + \Gamma\right)t} + \psi_- e^{\left(-\gamma - \Gamma\right)t} \right] \\ &= C_+ e^{-(\gamma - \Gamma) t} + C_- e^{-(\gamma + \Gamma) t},\end{align}\]

where

\[C_\pm = \mathrm{Re}[\psi_\pm].\]

This is called an \textbf{over-damped} solution. The solution consists
of two terms, both exponentially decaying in time, with
\((\gamma-\Gamma)\) and \((\gamma + \Gamma)\) serving as the decay
rates. Note that both decay rates are positive real numbers, because
\(\Gamma < \gamma\) from the definition of \(\Gamma\). Also, note that
the first decay rate \((\gamma - \Gamma)\) is a \emph{decreasing}
function of \(\gamma\), whereas the second decay rate
\((\gamma + \Gamma)\) is an \emph{increasing} function of \(\gamma\).

    \begin{Verbatim}[commandchars=\\\{\}]
{\color{incolor}In [{\color{incolor}4}]:} \PY{o}{\PYZpc{}}\PY{k}{matplotlib} inline
        \PY{k+kn}{from} \PY{n+nn}{ipywidgets} \PY{k}{import} \PY{n}{interact}\PY{p}{,} \PY{n}{FloatSlider}
        \PY{k+kn}{from} \PY{n+nn}{numpy} \PY{k}{import} \PY{n}{linspace}\PY{p}{,} \PY{n}{sqrt}\PY{p}{,} \PY{n}{exp}
        \PY{k+kn}{import} \PY{n+nn}{matplotlib}\PY{n+nn}{.}\PY{n+nn}{pyplot} \PY{k}{as} \PY{n+nn}{plt}
        
        \PY{k}{def} \PY{n+nf}{plot\PYZus{}overdamped}\PY{p}{(}\PY{n}{omega0}\PY{p}{,} \PY{n}{gamma}\PY{p}{,} \PY{n}{C1}\PY{p}{,} \PY{n}{C2}\PY{p}{)}\PY{p}{:}
            \PY{n}{x0}\PY{p}{,} \PY{n}{t} \PY{o}{=} \PY{l+m+mf}{2.0}\PY{p}{,} \PY{n}{linspace}\PY{p}{(}\PY{l+m+mi}{0}\PY{p}{,} \PY{l+m+mi}{10}\PY{p}{,} \PY{l+m+mi}{200}\PY{p}{)}
        
            \PY{n}{G} \PY{o}{=} \PY{n}{sqrt}\PY{p}{(}\PY{n}{gamma}\PY{o}{*}\PY{o}{*}\PY{l+m+mi}{2} \PY{o}{\PYZhy{}} \PY{n}{omega0}\PY{o}{*}\PY{o}{*}\PY{l+m+mi}{2}\PY{p}{)}
            \PY{n}{xe} \PY{o}{=} \PY{n}{C1} \PY{o}{*} \PY{n}{exp}\PY{p}{(}\PY{o}{\PYZhy{}}\PY{p}{(}\PY{n}{gamma}\PY{o}{\PYZhy{}}\PY{n}{G}\PY{p}{)}\PY{o}{*}\PY{n}{t}\PY{p}{)}
            \PY{n}{x}  \PY{o}{=} \PY{n}{xe} \PY{o}{+} \PY{n}{C2} \PY{o}{*} \PY{n}{exp}\PY{p}{(}\PY{o}{\PYZhy{}}\PY{p}{(}\PY{n}{gamma}\PY{o}{+}\PY{n}{G}\PY{p}{)}\PY{o}{*}\PY{n}{t}\PY{p}{)}
            
            \PY{n}{plt}\PY{o}{.}\PY{n}{figure}\PY{p}{(}\PY{n}{figsize}\PY{o}{=}\PY{p}{(}\PY{l+m+mi}{10}\PY{p}{,}\PY{l+m+mi}{5}\PY{p}{)}\PY{p}{)}
            \PY{n}{plt}\PY{o}{.}\PY{n}{title}\PY{p}{(}\PY{l+s+s1}{\PYZsq{}}\PY{l+s+s1}{Motion of an overdamped oscillator}\PY{l+s+s1}{\PYZsq{}}\PY{p}{)}
            \PY{n}{ax} \PY{o}{=} \PY{n}{plt}\PY{o}{.}\PY{n}{subplot}\PY{p}{(}\PY{l+m+mi}{1}\PY{p}{,}\PY{l+m+mi}{1}\PY{p}{,}\PY{l+m+mi}{1}\PY{p}{)}
            \PY{n}{plt}\PY{o}{.}\PY{n}{plot}\PY{p}{(}\PY{n}{t}\PY{p}{,} \PY{n}{x}\PY{p}{,} \PY{l+s+s1}{\PYZsq{}}\PY{l+s+s1}{b}\PY{l+s+s1}{\PYZsq{}}\PY{p}{,} \PY{n}{label}\PY{o}{=}\PY{l+s+s1}{\PYZsq{}}\PY{l+s+s1}{x(t)}\PY{l+s+s1}{\PYZsq{}}\PY{p}{)}
            \PY{n}{plt}\PY{o}{.}\PY{n}{plot}\PY{p}{(}\PY{n}{t}\PY{p}{,} \PY{n}{xe}\PY{p}{,} \PY{l+s+s1}{\PYZsq{}}\PY{l+s+s1}{r\PYZhy{}\PYZhy{}}\PY{l+s+s1}{\PYZsq{}}\PY{p}{,} \PY{n}{label}\PY{o}{=}\PY{l+s+s1}{\PYZsq{}}\PY{l+s+s1}{Limit}\PY{l+s+s1}{\PYZsq{}}\PY{p}{)}
            \PY{n}{plt}\PY{o}{.}\PY{n}{xlabel}\PY{p}{(}\PY{l+s+s1}{\PYZsq{}}\PY{l+s+s1}{t}\PY{l+s+s1}{\PYZsq{}}\PY{p}{)}
            \PY{n}{plt}\PY{o}{.}\PY{n}{ylabel}\PY{p}{(}\PY{l+s+s1}{\PYZsq{}}\PY{l+s+s1}{x}\PY{l+s+s1}{\PYZsq{}}\PY{p}{)}
            \PY{n}{plt}\PY{o}{.}\PY{n}{xlim}\PY{p}{(}\PY{n}{t}\PY{p}{[}\PY{l+m+mi}{0}\PY{p}{]}\PY{p}{,} \PY{n}{t}\PY{p}{[}\PY{o}{\PYZhy{}}\PY{l+m+mi}{1}\PY{p}{]}\PY{p}{)}
            \PY{n}{plt}\PY{o}{.}\PY{n}{ylim}\PY{p}{(}\PY{o}{\PYZhy{}}\PY{l+m+mf}{0.1}\PY{o}{*}\PY{n}{x0}\PY{p}{,} \PY{n}{x0}\PY{p}{)}
            \PY{n}{ax}\PY{o}{.}\PY{n}{legend}\PY{p}{(}\PY{n}{numpoints}\PY{o}{=}\PY{l+m+mi}{1}\PY{p}{,} \PY{n}{loc}\PY{o}{=}\PY{l+s+s1}{\PYZsq{}}\PY{l+s+s1}{upper right}\PY{l+s+s1}{\PYZsq{}}\PY{p}{)}
        
        \PY{n}{interact}\PY{p}{(}\PY{n}{plot\PYZus{}overdamped}\PY{p}{,}
                 \PY{n}{omega0}\PY{o}{=} \PY{n}{FloatSlider}\PY{p}{(}\PY{n+nb}{min}\PY{o}{=}\PY{l+m+mf}{0.1}\PY{p}{,} \PY{n+nb}{max}\PY{o}{=}\PY{l+m+mf}{1.0}\PY{p}{,} \PY{n}{step}\PY{o}{=}\PY{l+m+mf}{0.05}\PY{p}{,} \PY{n}{value}\PY{o}{=}\PY{l+m+mf}{1.0}\PY{p}{)}\PY{p}{,}
                 \PY{n}{gamma} \PY{o}{=} \PY{n}{FloatSlider}\PY{p}{(}\PY{n+nb}{min}\PY{o}{=}\PY{l+m+mf}{1.0}\PY{p}{,} \PY{n+nb}{max}\PY{o}{=}\PY{l+m+mf}{5.0}\PY{p}{,} \PY{n}{step}\PY{o}{=}\PY{l+m+mf}{0.05}\PY{p}{,} \PY{n}{value}\PY{o}{=}\PY{l+m+mf}{0.1}\PY{p}{)}\PY{p}{,}
                 \PY{n}{C1}    \PY{o}{=} \PY{n}{FloatSlider}\PY{p}{(}\PY{n+nb}{min}\PY{o}{=}\PY{l+m+mf}{0.0}\PY{p}{,} \PY{n+nb}{max}\PY{o}{=}\PY{l+m+mf}{1.0}\PY{p}{,} \PY{n}{step}\PY{o}{=}\PY{l+m+mf}{0.05}\PY{p}{,} \PY{n}{value}\PY{o}{=}\PY{l+m+mf}{1.0}\PY{p}{)}\PY{p}{,}
                 \PY{n}{C2}    \PY{o}{=} \PY{n}{FloatSlider}\PY{p}{(}\PY{n+nb}{min}\PY{o}{=}\PY{l+m+mf}{0.0}\PY{p}{,} \PY{n+nb}{max}\PY{o}{=}\PY{l+m+mf}{1.0}\PY{p}{,} \PY{n}{step}\PY{o}{=}\PY{l+m+mf}{0.05}\PY{p}{,} \PY{n}{value}\PY{o}{=}\PY{l+m+mf}{1.0}\PY{p}{)}\PY{p}{)}\PY{p}{;}
\end{Verbatim}

    \begin{center}
    \adjustimage{max size={0.9\linewidth}{0.9\paperheight}}{04_complex_oscillations_files/04_complex_oscillations_10_0.png}
    \end{center}
    { \hspace*{\fill} \\}
    
    The larger decay rate, \((\gamma + \Gamma)\), is associated with a
faster-decaying exponential. Therefore, at long times the second term
becomes negligible compared to the first term. Then the solution
approaches the limit

\[x(t) \approx C_+ e^{-(\gamma - \Gamma) t} \qquad (\mathrm{large}\;\;t).\]

This limiting curve is shown as a red dash in the above figure.

This has an interesting implication: \emph{the stronger the damping, the
slower the effective decay rate at long times}. Why does this happen? In
the over-damped regime, the motion of the oscillator is dominated by the
damping force rather than the spring force; as the oscillator tries to
return to its equilibrium position \(x = 0\), the damping acts against
this motion. Hence, the stronger the damping, the slower the decay to
equilibrium.

This contrasts sharply with the ``under-damped'' regime discussed
\protect\hyperlink{underdamped}{above}, in which the spring force
dominates the damping force. In that case, stronger damping speeds up
the decay to equilibrium, by causing the kinetic energy of the
oscillation to be dissipated more rapidly.

    \subsubsection{Critical damping}\label{critical-damping}

\textbf{Critical damping} occurs when \(\omega_0 = \gamma\). Under this
special condition, \protect\hyperlink{ansatz}{the solution that we
previously derived} reduces to

\[z(t) = \left(\psi_+ + \psi_-\right) e^{-\gamma t}.\]

This has only \emph{one} independent complex parameter, i.e.~the
parameter \((\psi_+ + \psi_-)\). Therefore, it cannot be a general
solution for the complex damped harmonic oscillator equation, which is
still a second-order ODE.

We will not go into detail here regarding the procedure for finding the
general solution for the critically-damped oscillator, leaving it as an
\protect\hyperlink{exercises}{exercise} for the interested reader.
Basically, we can Taylor expand the solution on either side of the
critical point, and then show that there is a solution of the form

\[z(t) = \left(A + B t\right)\, e^{-\gamma t},\]

which contains the desired two independent parameters.

The critically-damped solution contains an exponential decay constant of
\(\gamma\), which is the same as the decay constant for the
\protect\hyperlink{underdamped}{envelope function in the under-damped
regime}, and \emph{smaller} than the (long-time) decay constants in the
\protect\hyperlink{overdamped}{over-damped regime}. Hence, we can regard
the critically-damped solution as the \emph{fastest-decaying
non-oscillatory solution}.

This feature of critical damping is employed in many engineering
contexts, the most familiar being automatic door closers. If the damping
is too weak or the spring force is too strong (under-damped), the door
will tend to slam shut, whereas if the damping is too strong or the
spring force is too weak (under-damping), the door will take
unnecessarily long to close. Hence, door closers need to be tuned to a
``sweet spot'' that corresponds to the critical damping point.

    \subsection{Stating the solution in terms of initial
conditions}\label{stating-the-solution-in-terms-of-initial-conditions}

The general solution for the complex damped harmonic oscillator
equation, which we \protect\hyperlink{ansatz}{obtained above}, contains
two undetermined parameters which are the complex amplitudes of the
``clockwise'' and ``counterclockwise'' complex oscillations:

\[z(t) = \psi_+ e^{-i\omega_+ t} + \psi_- e^{-i\omega_- t}, \quad\mathrm{where} \;\; \omega_\pm =  -i\gamma  \pm \sqrt{\omega_0^2 - \gamma^2}.\]

However, mechanics problems are often expressed in terms of an
\textbf{initial-value problem}, which expresses the state of the system
at some initial time \(t = 0\). Suppose we are given \(z(0) \equiv x_0\)
and \(\dot{z}(0) \equiv v_0\); then what is \(z(t)\) in terms of \(x_0\)
and \(v_0\)?

We can solve the initial-value problem by finding \(z(0)\) and
\(\dot{z}(0)\) in terms of the above general solution for \(z(t)\). The
results are

\[\begin{aligned} z(0) &= \quad \psi_+ + \psi_- &= x_0& \\ \dot{z}(0) &= -i\omega_+ \psi_+ - i \omega_- \psi_- &= v_0&.\end{aligned}\]

These two equations can be combined into a 2x2 matrix equation:

\[\begin{bmatrix}1 & 1 \\ -i\omega_+ & -i\omega_-\end{bmatrix} \begin{bmatrix}\psi_+ \\ \psi_-\end{bmatrix} = \begin{bmatrix}x_0 \\ v_0\end{bmatrix}.\]

So long as the system is not at the critical point (i.e.,
\(\omega_+ \ne \omega_-\)), the matrix is non-singular, and we can
invert it to obtain \(\psi_\pm\):

\[\begin{bmatrix}\psi_+ \\ \psi_-\end{bmatrix} = \frac{1}{i(\omega_+-\omega_-)}\begin{bmatrix}-i\omega_-x_0 - v_0 \\ i\omega_+x_0 + v_0 \end{bmatrix}.\]

We can plug these coefficients back into the general solution. After
some algebra, the result simplifies to

\[z(t) = e^{-\gamma t} \left[x_0 \cos(\Omega t) + \frac{\gamma x_0 + v_0}{\Omega} \, \sin(\Omega t)\right], \;\; \mathrm{where}\;\; \Omega \equiv \sqrt{\omega_0^2 - \gamma^2}.\]

For the under-damped case, \(\Omega\) is real, and this solution is
consistent with the one \protect\hyperlink{underdamped}{we previously
derived}, except that it is now explicitly expressed in terms our
initial conditions \(x_0\) and \(v_0\). As for the
\protect\hyperlink{overdamped}{over-damped case}, we can perform the
replacement

\[\Omega \rightarrow i \Gamma = i \sqrt{\gamma^2 - \omega_0^2}.\]

Then, using the \href{00_mathfunctions.ipynb\#hyperbolic}{relationships
between trigonometric and hyperbolic functions}, the solution can be
re-written as

\[\begin{aligned}z(t) &= e^{-\gamma t} \left[x_0 \cosh(\Gamma t) + \frac{\gamma x_0 + v_0}{i\Gamma} \, i \sinh(\Gamma t)\right] \\ &= \left(\frac{x_0}{2} + \frac{\gamma x_0 + v_0}{2\Gamma}\right) e^{-(\gamma - \Gamma) t} + \left(\frac{x_0}{2} - \frac{\gamma x_0 + v_0}{2\Gamma}\right) e^{-(\gamma+\Gamma)t},\end{aligned}\]

which is again consistent with our
\protect\hyperlink{overdamped}{previous result}.

In either case, so long as we plug in real values for \(x_0\) and
\(v_0\), the solution is guaranteed to be real for all \(t\). That's to
be expected, since the real solution is also one of the specific
solutions for the complex harmonic oscillator equation. The resulting
solution is plotted below.

    \begin{Verbatim}[commandchars=\\\{\}]
{\color{incolor}In [{\color{incolor}5}]:} \PY{o}{\PYZpc{}}\PY{k}{matplotlib} inline
        \PY{k+kn}{from} \PY{n+nn}{ipywidgets} \PY{k}{import} \PY{n}{interact}\PY{p}{,} \PY{n}{FloatSlider}
        \PY{k+kn}{from} \PY{n+nn}{numpy} \PY{k}{import} \PY{n}{linspace}\PY{p}{,} \PY{n}{sin}\PY{p}{,} \PY{n}{cos}\PY{p}{,} \PY{n}{sqrt}\PY{p}{,} \PY{n}{exp}
        \PY{k+kn}{import} \PY{n+nn}{matplotlib}\PY{n+nn}{.}\PY{n+nn}{pyplot} \PY{k}{as} \PY{n+nn}{plt}
        \PY{k+kn}{from} \PY{n+nn}{math} \PY{k}{import} \PY{n}{ceil}
        
        \PY{k}{def} \PY{n+nf}{plot\PYZus{}oscillator}\PY{p}{(}\PY{n}{omega0}\PY{p}{,} \PY{n}{gamma}\PY{p}{,} \PY{n}{x0}\PY{p}{,} \PY{n}{v0}\PY{p}{)}\PY{p}{:}
            \PY{n}{t} \PY{o}{=} \PY{n}{linspace}\PY{p}{(}\PY{l+m+mi}{0}\PY{p}{,} \PY{l+m+mi}{20}\PY{p}{,} \PY{l+m+mi}{200}\PY{p}{)}
        
            \PY{k}{if} \PY{n}{omega0} \PY{o}{\PYZgt{}} \PY{n}{gamma}\PY{p}{:}
                \PY{n}{W} \PY{o}{=} \PY{n}{sqrt}\PY{p}{(}\PY{n}{omega0}\PY{o}{*}\PY{o}{*}\PY{l+m+mi}{2} \PY{o}{\PYZhy{}} \PY{n}{gamma}\PY{o}{*}\PY{o}{*}\PY{l+m+mi}{2}\PY{p}{)}
                \PY{n}{A}\PY{p}{,} \PY{n}{B} \PY{o}{=} \PY{n}{x0}\PY{p}{,} \PY{p}{(}\PY{n}{gamma}\PY{o}{*}\PY{n}{x0}\PY{o}{+}\PY{n}{v0}\PY{p}{)}\PY{o}{/}\PY{n}{W}
                \PY{n}{x} \PY{o}{=} \PY{n}{exp}\PY{p}{(}\PY{o}{\PYZhy{}}\PY{n}{gamma}\PY{o}{*}\PY{n}{t}\PY{p}{)} \PY{o}{*} \PY{p}{(}\PY{n}{A} \PY{o}{*} \PY{n}{cos}\PY{p}{(}\PY{n}{W}\PY{o}{*}\PY{n}{t}\PY{p}{)} \PY{o}{+} \PY{n}{B} \PY{o}{*} \PY{n}{sin}\PY{p}{(}\PY{n}{W}\PY{o}{*}\PY{n}{t}\PY{p}{)}\PY{p}{)}
            \PY{k}{elif} \PY{n}{omega0} \PY{o}{\PYZlt{}} \PY{n}{gamma}\PY{p}{:}
                \PY{n}{G} \PY{o}{=} \PY{n}{sqrt}\PY{p}{(}\PY{n}{gamma}\PY{o}{*}\PY{o}{*}\PY{l+m+mi}{2} \PY{o}{\PYZhy{}} \PY{n}{omega0}\PY{o}{*}\PY{o}{*}\PY{l+m+mi}{2}\PY{p}{)}
                \PY{n}{A} \PY{o}{=} \PY{l+m+mf}{0.5}\PY{o}{*}\PY{p}{(}\PY{n}{x0} \PY{o}{+} \PY{p}{(}\PY{n}{gamma}\PY{o}{*}\PY{n}{x0} \PY{o}{+} \PY{n}{v0}\PY{p}{)}\PY{o}{/}\PY{n}{G}\PY{p}{)}
                \PY{n}{B} \PY{o}{=} \PY{l+m+mf}{0.5}\PY{o}{*}\PY{p}{(}\PY{n}{x0} \PY{o}{\PYZhy{}} \PY{p}{(}\PY{n}{gamma}\PY{o}{*}\PY{n}{x0} \PY{o}{+} \PY{n}{v0}\PY{p}{)}\PY{o}{/}\PY{n}{G}\PY{p}{)}
                \PY{n}{x} \PY{o}{=} \PY{n}{A}\PY{o}{*}\PY{n}{exp}\PY{p}{(}\PY{o}{\PYZhy{}}\PY{p}{(}\PY{n}{gamma}\PY{o}{\PYZhy{}}\PY{n}{G}\PY{p}{)}\PY{o}{*}\PY{n}{t}\PY{p}{)} \PY{o}{+} \PY{n}{B} \PY{o}{*} \PY{n}{exp}\PY{p}{(}\PY{o}{\PYZhy{}}\PY{p}{(}\PY{n}{gamma}\PY{o}{+}\PY{n}{G}\PY{p}{)}\PY{o}{*}\PY{n}{t}\PY{p}{)}
            \PY{k}{else}\PY{p}{:}
                \PY{n}{x} \PY{o}{=} \PY{p}{(}\PY{n}{x0} \PY{o}{+} \PY{p}{(}\PY{n}{v0}\PY{o}{+}\PY{n}{gamma}\PY{o}{*}\PY{n}{x0}\PY{p}{)}\PY{o}{*}\PY{n}{t}\PY{p}{)} \PY{o}{*} \PY{n}{exp}\PY{p}{(}\PY{o}{\PYZhy{}}\PY{n}{gamma}\PY{o}{*}\PY{n}{t}\PY{p}{)}
        
            \PY{n}{plt}\PY{o}{.}\PY{n}{figure}\PY{p}{(}\PY{n}{figsize}\PY{o}{=}\PY{p}{(}\PY{l+m+mi}{10}\PY{p}{,}\PY{l+m+mi}{5}\PY{p}{)}\PY{p}{)}
            \PY{n}{plt}\PY{o}{.}\PY{n}{title}\PY{p}{(}\PY{l+s+s1}{\PYZsq{}}\PY{l+s+s1}{Motion of a damped harmonic oscillator}\PY{l+s+s1}{\PYZsq{}}\PY{p}{)}
            \PY{n}{plt}\PY{o}{.}\PY{n}{plot}\PY{p}{(}\PY{n}{t}\PY{p}{,} \PY{n}{x}\PY{p}{,} \PY{l+s+s1}{\PYZsq{}}\PY{l+s+s1}{b}\PY{l+s+s1}{\PYZsq{}}\PY{p}{,} \PY{n}{label}\PY{o}{=}\PY{l+s+s1}{\PYZsq{}}\PY{l+s+s1}{x(t)}\PY{l+s+s1}{\PYZsq{}}\PY{p}{)}
            \PY{n}{ax} \PY{o}{=} \PY{n}{plt}\PY{o}{.}\PY{n}{subplot}\PY{p}{(}\PY{l+m+mi}{1}\PY{p}{,}\PY{l+m+mi}{1}\PY{p}{,}\PY{l+m+mi}{1}\PY{p}{)}
            \PY{n}{plt}\PY{o}{.}\PY{n}{xlabel}\PY{p}{(}\PY{l+s+s1}{\PYZsq{}}\PY{l+s+s1}{t}\PY{l+s+s1}{\PYZsq{}}\PY{p}{)}
            \PY{n}{plt}\PY{o}{.}\PY{n}{ylabel}\PY{p}{(}\PY{l+s+s1}{\PYZsq{}}\PY{l+s+s1}{x}\PY{l+s+s1}{\PYZsq{}}\PY{p}{)}
            \PY{n}{lim} \PY{o}{=} \PY{n+nb}{max}\PY{p}{(}\PY{n}{ceil}\PY{p}{(}\PY{l+m+mf}{1.1}\PY{o}{*}\PY{n+nb}{max}\PY{p}{(}\PY{n+nb}{abs}\PY{p}{(}\PY{n}{x}\PY{p}{)}\PY{p}{)}\PY{p}{)}\PY{p}{,} \PY{l+m+mf}{1.0}\PY{p}{)}
            \PY{n}{plt}\PY{o}{.}\PY{n}{ylim}\PY{p}{(}\PY{o}{\PYZhy{}}\PY{n}{lim}\PY{p}{,} \PY{n}{lim}\PY{p}{)}
            \PY{n}{ax}\PY{o}{.}\PY{n}{legend}\PY{p}{(}\PY{n}{numpoints}\PY{o}{=}\PY{l+m+mi}{1}\PY{p}{,} \PY{n}{loc}\PY{o}{=}\PY{l+s+s1}{\PYZsq{}}\PY{l+s+s1}{upper right}\PY{l+s+s1}{\PYZsq{}}\PY{p}{)}
        
        \PY{n}{interact}\PY{p}{(}\PY{n}{plot\PYZus{}oscillator}\PY{p}{,}
                 \PY{n}{omega0} \PY{o}{=} \PY{n}{FloatSlider}\PY{p}{(}\PY{n+nb}{min}\PY{o}{=}\PY{l+m+mf}{0.1}\PY{p}{,} \PY{n+nb}{max}\PY{o}{=}\PY{l+m+mf}{1.9}\PY{p}{,} \PY{n}{step}\PY{o}{=}\PY{l+m+mf}{0.05}\PY{p}{,} \PY{n}{value}\PY{o}{=}\PY{l+m+mf}{1.0}\PY{p}{)}\PY{p}{,}
                 \PY{n}{gamma}  \PY{o}{=} \PY{n}{FloatSlider}\PY{p}{(}\PY{n+nb}{min}\PY{o}{=}\PY{l+m+mf}{0.0}\PY{p}{,} \PY{n+nb}{max}\PY{o}{=}\PY{l+m+mf}{2.0}\PY{p}{,} \PY{n}{step}\PY{o}{=}\PY{l+m+mf}{0.05}\PY{p}{,} \PY{n}{value}\PY{o}{=}\PY{l+m+mf}{1.0}\PY{p}{)}\PY{p}{,}
                 \PY{n}{x0}     \PY{o}{=} \PY{n}{FloatSlider}\PY{p}{(}\PY{n+nb}{min}\PY{o}{=}\PY{o}{\PYZhy{}}\PY{l+m+mf}{2.}\PY{p}{,} \PY{n+nb}{max}\PY{o}{=}\PY{l+m+mf}{2.0}\PY{p}{,} \PY{n}{step}\PY{o}{=}\PY{l+m+mf}{0.05}\PY{p}{,} \PY{n}{value}\PY{o}{=}\PY{l+m+mf}{1.0}\PY{p}{)}\PY{p}{,}
                 \PY{n}{v0}     \PY{o}{=} \PY{n}{FloatSlider}\PY{p}{(}\PY{n+nb}{min}\PY{o}{=}\PY{o}{\PYZhy{}}\PY{l+m+mf}{5.}\PY{p}{,} \PY{n+nb}{max}\PY{o}{=}\PY{l+m+mf}{5.0}\PY{p}{,} \PY{n}{step}\PY{o}{=}\PY{l+m+mf}{0.05}\PY{p}{,} \PY{n}{value}\PY{o}{=}\PY{l+m+mf}{5.0}\PY{p}{)}\PY{p}{)}\PY{p}{;}
\end{Verbatim}

    \begin{center}
    \adjustimage{max size={0.9\linewidth}{0.9\paperheight}}{04_complex_oscillations_files/04_complex_oscillations_14_0.png}
    \end{center}
    { \hspace*{\fill} \\}
    
    \subsection{Exercises}\label{exercises}

In the \protect\hyperlink{ansatz}{general solution for the complex
damped harmonic oscillator equation}, we encountered the complex
frequencies\[\omega_\pm = -i\gamma \pm \sqrt{\omega_0^2 - \gamma^2}.\]For
fixed \(\omega_0\) and \(\omega_0 > \gamma\) (under-damping), prove that
\(\omega_\pm\) lie along a circular arc in the complex plane.

Derive the \protect\hyperlink{critical_damping}{general solution for the
critically-damped oscillator}, by following these steps:

Consider the complex ODE, in the under-damped regime
\(\omega_0 > \gamma\). We \protect\hyperlink{ansatz}{have shown} that
the general solution has the
form\[z(t) = \psi_+ \, \exp\left[\left(-\gamma  - i \sqrt{\omega_0^2 - \gamma^2}\right)t\right] \; +\; \psi_- \, \exp\left[\left(-\gamma +i\sqrt{\omega_0^2 - \gamma^2}\right)t\right]\]for
some complex parameters \(\psi_+\) and \(\psi_-\). Let us define the
positive parameter \(\varepsilon = \sqrt{\omega_0^2 - \gamma^2}\).
Re-write \(z(t)\) in terms of \(\gamma\) and \(\varepsilon\) (i.e.,
eliminating \(\omega_0\)).

Taylor expand \(z(t)\) in the parameter \(\varepsilon\).

The expression for \(z(t)\) is presently parameterized by \(\psi_+\),
\(\psi_-\), \(\varepsilon\), and \(\gamma\). We are free to re-define
the parameters, e.g.~by
taking\[\begin{aligned}\alpha &= \psi_+ + \psi_- \\ \beta &= -i\varepsilon(\psi_+ - \psi_-).\end{aligned}\]Show
that\[z(t) = e^{-\gamma t}\left[\alpha + \beta t + \cdots\right],\]where
the omitted terms are of order \(\varepsilon^2\) and higher.

Show that in the limit \(\varepsilon \rightarrow 0\), the
re-parameterized form for \(z(t)\) reduces to the
\protect\hyperlink{critical_damping}{critically-coupled general
solution}.

Repeat the above derivation for the critically-damped solution, but
starting from the over-damped regime \(\gamma > \omega_0\).

A \textbf{parametric oscillator} is an oscillator whose spring
``constant'' varies with time, as described by the ordinary differential
equation

\[\left[\frac{d^2}{dt^2} + 2\gamma\frac{d}{dt} + \Omega(t)^2\right]x(t) = 0, \quad\mathrm{where}\;\;\Omega(t) = \omega_0\left[1 + \alpha \cos(2\omega_1 t)\right].\]

The term ``parametric'' refers to the fact that the parameter
\(\Omega\), which is normally a constant, has been turned into a
time-dependent quantity.

Suppose the ``modulation frequency'', \(\omega_1\), is much smaller than
the natural frequency \(\omega_0\). Let's make \(x(t)\) complex, and
look for a solution of the form

\[x(t) = \psi(t) \, e^{-i\omega_0 t},\]

where \(\psi(t)\) is a complex ``envelope function'' which varies much
more slowly than the \(e^{-i\omega_0 t}\) factor. Mathematically, the
slowness of the variation is represented by the condition

\[\left|\frac{d^2\psi}{dt^2}\right| \ll \omega_0 \left|\frac{d\psi}{dt}\right|.\]

In such a situation, the second time derivative can be neglected; this
is called the ``slowly-varying envelope approximation''. By making this
approximation, show that the parametric oscillator equation reduces to
the form

\[\frac{d\left[\ln(\psi)\right]}{dt} = f(t),\]

and find \(f(t)\). Hence, solve for \(\psi(t)\) and show that the
oscillation ampitude \(|\psi(t)|\) consists of an exponential decay
overlaid on a sinusoidal modulation.


    % Add a bibliography block to the postdoc
    
    
    
    \end{document}
