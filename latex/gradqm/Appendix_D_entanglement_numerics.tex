\documentclass[pra,12pt]{revtex4}
\usepackage{amsmath}
\usepackage{amssymb}
\usepackage{graphicx}
\usepackage{color}
\usepackage{mathrsfs}
\usepackage{framed}
\usepackage[pdfborder={0 0 0},colorlinks=true,linkcolor=blue,urlcolor=blue]{hyperref}

\def\ket#1{\left|#1\right\rangle}
\def\bra#1{\left\langle#1\right|}
\def\braket#1{\left\langle#1\right\rangle}

\usepackage{fancyhdr}
\fancyhf{}
\lhead{\tiny Y.~D.~Chong}
\rhead{\scriptsize PH4401: Quantum Mechanics III}
\lfoot{}
\rfoot{\thepage}
\pagestyle{fancy}

\setlength{\parindent}{14pt}
\renewcommand{\theequation}{D.\arabic{equation}}
\renewcommand{\thesection}{D\arabic{section}}

\renewcommand{\baselinestretch}{1.0}
\setlength{\parskip}{0.04in}

\begin{document}

\begin{center}
{\large \textbf{Appendix D: Numerical Tensor Products}}
\end{center}

\noindent
This appendix discusses how tensor products are handled in numerical
linear algebra software.  We will focus on Python with the
\href{https://numpy.org/}{Numeric Python (numpy)} module.  The
discussion is also applicable, with minor modifications, to GNU Octave
or Matlab.  We assume the reader is familiar with the
\href{https://docs.scipy.org/doc/numpy/user/quickstart.html}{basics of
  Python/Numpy}, e.g.~how vectors can be represented by 1D arrays,
linear operators (matrices) can be represented by 2D arrays, etc.

Tensor products are implemented by the
\href{https://docs.scipy.org/doc/numpy/reference/generated/numpy.kron.html}{\texttt{numpy.kron}}
function, which performs an operation called a \textbf{Kronecker
  product}.  The function takes two inputs, which can be 1D arrays, 2D
arrays, or even higher-dimensional arrays (which we won't discuss).
It returns a new array representing the tensor product of the inputs,
whose dimensionality depends on that of the inputs.  The function can
be used to compute products of vectors ($|a\rangle\otimes|b\rangle$),
products of operators ($\hat{O}_A\otimes \hat{O}_B$), etc.  It can
even compute ``mixed'' products like $|a\rangle\otimes\hat{O}_B$,
which is useful for calculating partial projections and partial
traces.

In the next few sections, we will prove that the various tensor
products of bras, kets, and operators can be represented using the
following Numpy expressions involving
\href{https://docs.scipy.org/doc/numpy/reference/generated/numpy.kron.html}{\texttt{numpy.kron}}:
\begin{framed}
  \begin{align*}
    |a\rangle\otimes|b\rangle &\;\;\leftrightarrow\;
    \texttt{kron(a, b)} &
    \langle a | \otimes \langle b| &\;\;\leftrightarrow\;
    \texttt{kron(a.conj(), b.conj())} \\
    \hat{A}\otimes\hat{B} &\;\;\leftrightarrow\;
    \texttt{kron(A, B)} \\
    |a\rangle \otimes \hat{B} &\;\;\leftrightarrow\;
    \texttt{kron(a, B.T).T} &
    \langle a| \otimes \hat{B} &\;\;\leftrightarrow\;
    \texttt{kron(a.conj(), B)} \\
    \hat{A} \otimes | b\rangle &\;\;\leftrightarrow\;
    \texttt{kron(A.T, b).T} &
    \hat{A} \otimes \langle b| &\;\;\leftrightarrow\;
    \texttt{kron(A, b.conj())}
  \end{align*}
\end{framed}

\section{Products of vectors}

Suppose $a$ and $b$ are both 1D arrays, of length $M$ and $N$
respectively; let their components be $(a_0, a_1, \dots, a_{M-1})$ and
$(b_0, b_1, \dots, b_{N-1})$.  Following Numpy conventions, we do not
explicitly distinguish between ``row vectors'' and ``column vectors'',
and component indices start from 0.  The Kronecker product between $a$
and $b$ generates the following 1D array:
\begin{equation}
  \texttt{kron}(a, b) = \Big(a_0b_0,\,\dots,\, a_0 b_{N-1},\, a_1 b_0,\, \dots,\, a_1 b_{N-1},\, \dots,\, a_{M-1}b_{N-1}\Big).\label{kronabdef1}
\end{equation}
We can think of this as \textit{taking each component of $a$, and
  multiplying it by the entire $b$ array}:
\begin{equation}
  \texttt{kron}(a, b) = \Big(a_0b,\, a_1 b,\, \dots,\, a_{M-1}b\Big).
  \label{kronabdef2}
\end{equation}
As we shall see, this description of the Kronecker product extends to
higher-dimensional arrays as well.  In the present case, $a$ and $b$
are both 1D, and the result is a 1D array of $MN$ components, which
can be described compactly in index notation by
\begin{equation}
  \big[\, \texttt{kron}(a, b) \,\big]_{\mu} = a_m \, b_n \;\;\;\mathrm{where}\;\;\;\mu = mN+n.
  \label{kronab}
\end{equation}
The index $\mu$ is defined so that as we sweep through $m =
0,\dots,M-1$ and $n = 0,\dots,N-1$, $\mu$ runs through the values
$0,1,\dots,MN-1$ without duplication.  Note, by the way, that the
order of inputs into $\texttt{kron}$ is important:
$\texttt{kron}(a,b)$ is not the same as $\texttt{kron}(b,a)$!  The
asymmetry between $a$ and $b$ is apparent in the definitions
\eqref{kronabdef1} and \eqref{kronabdef2}.

In terms of abstract linear algebra (as used in quantum theory), let
$\mathscr{H}_A$ be an $M$-dimensional space with basis
$\{|m\rangle\}$, and $\mathscr{H}_B$ be an $N$-dimensional space with
basis $\{|n\rangle\}$.  Any two vectors $|a\rangle \in \mathscr{H}_A$
and $|b\rangle \in \mathscr{H}_B$ can be written as
\begin{equation}
  |a\rangle = \sum_{m=0}^{M-1} a_m |m\rangle, \quad |b\rangle = \sum_{n=0}^{N-1} b_n |n\rangle.
\end{equation}
A natural basis for the product space $\mathscr{H}_A\otimes
\mathscr{H}_B$ is
\begin{equation}
  \Big\{|\mu\rangle \equiv |m\rangle |n\rangle\Big\}
  \;\;\;\mathrm{where} \;
  \begin{cases}
    \mu\!\!\!\! &= mN+n \\
    m \!\!\!\!&= 0,1,\dots,M-1 \\
    n \!\!\!\!&= 0,1, \dots, N-1.
  \end{cases}
  \label{mubasis}
\end{equation}
Using Eq.~\eqref{kronab}, we can show that
\begin{equation}
  |a\rangle\otimes|b\rangle
  = \sum_{mn} a_m b_n |m\rangle |n\rangle
  = \sum_{\mu=0}^{MN-1} \big[\texttt{kron}(a,b)\big]_\mu \; |\mu\rangle.
\end{equation}
Therefore, we need only remember that the tensor product of two
kets is represented by
\begin{equation}
  |a\rangle\otimes|b\rangle \;\;\leftrightarrow\;\;
  \texttt{kron}(a,b).
  \label{result1}
\end{equation}
Likewise, for bras,
\begin{equation}
  \;\;\,\langle a| \otimes \langle b| \;\;\leftrightarrow\;\;
  \texttt{kron}(a^*,b^*).
  \label{result1a}
\end{equation}

\section{Products of matrices}

Let $A$ and $B$ be 2D arrays of size $M\times M$ and $N\times N$
respectively:
\begin{equation}
  A = \begin{bmatrix}A_{00} & \cdots & A_{0,M-1} \\ \vdots & \ddots & \vdots \\ A_{M-1,0} & \cdots & A_{M-1,M-1} \end{bmatrix}, \;\;
  B = \begin{bmatrix}B_{00} & \cdots & B_{0,N-1} \\ \vdots & \ddots & \vdots \\  B_{N-1,0} & \cdots & B_{N-1,N-1} \end{bmatrix}.
\end{equation}
Then the Kronecker product of $A$ and $B$ is an $MN\times MN$ array of
the form
\begin{equation}
  \texttt{kron}(A,B) = \begin{bmatrix} A_{00}B & \cdots & A_{0,M-1}B \\ \vdots & \ddots & \vdots \\ A_{M-1,0}B & \cdots & A_{M-1,M-1}B\end{bmatrix}.
    \label{kronAB_explicit}
\end{equation}
As before, this can be interpreted as taking each component of $A$,
and multiplying it by $B$.  The result can be written using index
notation as
\begin{equation}
  \big[\,\texttt{kron}(A,B)\,\big]_{\mu\mu'} = A_{mm'} B_{nn'}\;\;\;\mathrm{where}\;\;\;\mu = mN+n, \; \mu' = m'N+n'.
  \label{kronAB}
\end{equation}

In the language of abstract linear algebra, let $\mathscr{H}_A$ and
$\mathscr{H}_B$ again be spaces with bases $\{|m\rangle\}$ and
$\{|n\rangle\}$.  Consider two linear operators $\hat{A}$ and
$\hat{B}$ acting respectively on these spaces:
\begin{equation}
  \hat{A} = \sum_{m,m'=0}^{M-1}  |m\rangle A_{mm'} \langle m'|, \quad \hat{B} = \sum_{n,n'=0}^{N-1} |n\rangle B_{nn'}\langle n'|.
\end{equation}
Then we can show using Eq.~\eqref{kronAB} that
\begin{align}
  \hat{A}\otimes\hat{B} &=
  \sum_{mm'nn'} |m\rangle|n\rangle\,A_{mm'} B_{nn'}\,\langle m'| \langle n'| \\
  &= \sum_{\mu,\mu'} |\mu\rangle \;
  \big[\,\texttt{kron}(A,B)\,\big]_{\mu\mu'} \; \langle\mu'|,
\end{align}
where $\big\{|\mu\rangle\big\}$ is the basis for
$\mathscr{H}_A\otimes\mathscr{H}_B$ previously defined in
Eq.~\eqref{mubasis}.  Thus,
\begin{equation}
  \hat{A}\otimes\hat{B} \;\;\leftrightarrow\;\;
  \texttt{kron}(A,B).
  \label{result2}
\end{equation}
This result, like Eq.~\eqref{result1}, is nice because it means that
we can relegate the handling of tensor product components entirely to
the \texttt{kron} function.  So long as we make a particular basis
choice for the spaces $\mathscr{H}_A$ and $\mathscr{H}_B$, and keep to
that choice, \texttt{kron} will return the vector products and
operator products expressed using an appropriate and natural basis for
$\mathscr{H}_A\otimes\mathscr{H}_B$ [i.e., the basis defined in
  Eq.~\eqref{mubasis}].

\section{Mixed products}

For ``mixed'' products of operators with bras or kets, the
representation using \texttt{kron} is more complicated, but only
slightly.  First, consider the 1D array $a$ and 2D array $B$:
\begin{equation}
  a = (a_0, \dots, a_{M-1}), \;\;\;
  B = \begin{bmatrix}B_{00} & \cdots & B_{0,N-1} \\ \vdots & \ddots & \vdots \\  B_{N-1,0} & \cdots & B_{N-1,N-1} \end{bmatrix}.
\end{equation}
Then the Kronecker product between the two is
\begin{equation}
  \texttt{kron}(a, B) = (a_0 B, a_1 B, \dots, a_{M-1} B).
\end{equation}
Note that $a$ is explicitly treated as a row vector.  In component
terms,
\begin{equation}
  [\texttt{kron}(a, B)]_{n\mu'} = a_{m'} B_{nn'}, \;\;\;\mathrm{where}\;\;\;
  \mu' = m'N+n'.
  \label{kronaB}
\end{equation}
In linear algebraic terms, let
\begin{equation}
  |a\rangle = \sum_m a_m |m\rangle, \;\;\; \hat{B} = \sum_{nn'}
  |n\rangle B_{nn'} \langle n'|.
\end{equation}
Then
\begin{align}
  |a\rangle \otimes \hat{B} = \sum_{\mu n'} |\mu\rangle \,
  a_m B_{nn'} \, \langle n'|, \qquad \mu = mN + n.
  \label{aBprod}
\end{align}
This does not quite match Eq.~\eqref{kronaB}!  The basic problem is
that the Kronecker product treats $a$ a row vector.  However, we can
patch things up by massaging Eq.~\eqref{kronaB} a bit:
\begin{align}
  \begin{aligned}
    \left.[\right.\texttt{kron}(a, B^T)^T]_{\mu n'} &=
    [\texttt{kron}(a, B^T)]_{n' \mu} \\
    &= a_{m} (B^T)_{n'n} \quad\;\;\mathrm{where}\;\; \mu' = m'N + n' \\
    &= a_{m} B_{nn'}.
  \end{aligned}
\end{align}
This is an appropriate match for Eq.~\eqref{aBprod}, so we conclude
that
\begin{equation}
  |a\rangle \otimes \hat{B} \;\;\leftrightarrow\;\;
  \texttt{kron}(a, B^T)^T.
\end{equation}
To take the product using the bra $\langle a|$, we replace
Eq.~\eqref{aBprod} by
\begin{align}
  \langle a| \otimes \hat{B} = \sum_{n\mu'} |n\rangle \,
  a_{m'}^* B_{nn'} \, \langle \mu'|, \qquad \mu' = m'N + n'.
\end{align}
Comparing this to Eq.~\eqref{kronaB} yields
\begin{equation}
  \langle a| \otimes \hat{B} \;\;\leftrightarrow\;\;
  \texttt{kron}(a^*, B).
\end{equation}

Likewise, consider the 2D array $A$ and 1D array $b$:
\begin{equation}
  A = \begin{bmatrix}A_{00} & \cdots & A_{0,M-1} \\ \vdots & \ddots & \vdots \\ A_{M-1,0} & \cdots & A_{M-1,M-1} \end{bmatrix}, \;\;\;
  b = (b_0, \dots, b_{N-1}).
\end{equation}
Then the Kronecker product is
\begin{equation}
  \texttt{kron}(A, b) =
  \begin{bmatrix}
    A_{00} b & \cdots & A_{0,M-1} b \\
    \vdots & \ddots & \vdots \\
    A_{M-1,0} b & \cdots & A_{M-1,M-1} b
  \end{bmatrix}.
\end{equation}
Similar to before, $b$ is treated as a row vector.  In component
terms,
\begin{equation}
  [\texttt{kron}(A, b)]_{m\mu'} = A_{mm'} b_{n'}, \;\;\;\mathrm{where}\;\;\;
  \mu' = m'N+n'.
\end{equation}
Using the same procedure as before, we can straightforwardly show that
\begin{align}
  \hat{A} \otimes | b\rangle &=
  \sum_{\mu m'} |\mu\rangle \, [\texttt{kron}(A^T, b)^T]_{\mu m'}
  \, \langle m'| && \leftrightarrow \;\; \texttt{kron}(A^T, b)^T\\
  \hat{A} \otimes \langle b| &=
  \sum_{m\mu'} |m\rangle \, [\texttt{kron}(A, b^*)]_{m\mu'}
  \,  \langle \mu'| && \leftrightarrow \;\; \texttt{kron}(A, b^*).
\end{align}

\end{document}
