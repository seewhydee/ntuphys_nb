\documentclass[pra,12pt]{revtex4}
\usepackage{amsmath}
\usepackage{amssymb}
\usepackage{graphicx}
\usepackage{color}
\usepackage{mathrsfs}
\usepackage[pdfborder={0 0 0},colorlinks=true,linkcolor=blue,urlcolor=blue]{hyperref}

\def\ket#1{\left|#1\right\rangle}
\def\bra#1{\left\langle#1\right|}
\def\braket#1{\left\langle#1\right\rangle}

\usepackage{fancyhdr}
\fancyhf{}
\lhead{\tiny Y.~D.~Chong}
\rhead{\scriptsize PH4401: Quantum Mechanics III}
\lfoot{}
\rfoot{\thepage}
\pagestyle{fancy}

\setlength{\parindent}{14pt}
\renewcommand{\theequation}{E.\arabic{equation}}
\renewcommand{\thesection}{E\arabic{section}}

\renewcommand{\baselinestretch}{1.0}
\setlength{\parskip}{0.07in}

\begin{document}

\begin{center}
{\large \textbf{Appendix E: Coherent States}}
\end{center}

Coherent states are special states of bosonic systems (including the
quantum harmonic oscillator, whose excitation quanta can be regarded
as bosonic particles) whose dynamics are highly similar to classical
oscillator trajectories.  They provide an important link between
quantum and classical harmonic oscillators.

\section{Definition}

The Hamiltonian of a simple harmonic oscillator (with $\hbar = m =
\omega_0 = 1$ for simplicity) is
\begin{equation}
  \hat{H} = \frac{\hat{p}^2}{2} + \frac{\hat{x}^2}{2},
  \label{hsho}
\end{equation}
where $\hat{x}$ and $\hat{p}$ are the position and momentum operators.
The ladder operators are
\begin{align}
  \hat{a} &= \frac{1}{\sqrt{2}} \left(\hat{x} + i\hat{p}\right)
  \label{a} \\
  \hat{a}^\dagger &= \frac{1}{\sqrt{2}} \left(\hat{x} - i\hat{p}\right).
  \label{adagger}
\end{align}
These obey the commutation relation
\begin{equation}
  \left[\hat{a}, \hat{a}^\dagger\right] = 1.
  \label{commutator}
\end{equation}
As a result, we can also regard these as the creation and annihilation
operators for a bosonic particle that has only one single-particle
state.

The Hamiltonian for the harmonic oscillator, Eq.~\eqref{hsho}, can be
written as $\hat{H} = \hat{a}^\dagger\hat{a} + 1/2$.  The annihilation
operator $\hat{a}$ kills off the ground state $|\varnothing\rangle$:
\begin{equation}
  \hat{a} |\varnothing\rangle = 0.
  \label{annihilation}
\end{equation}
Thus, $|\varnothing\rangle$ is analogous to the ``vacuum state'' for a
bosonic particle.

Returning to the Hamiltonian \eqref{hsho}, suppose we add a term
proportional to $\hat{x}$:
\begin{equation}
  \hat{H}' = \frac{\hat{p}^2}{2} + \frac{\hat{x}^2}{2} - \sqrt{2}\alpha_1\hat{x}.
  \label{hshift}
\end{equation}
The coefficient of $-\sqrt{2}\alpha_1$, where $\alpha_1 \in
\mathbb{R}$, is for later convenience.  By completing the square, we
can show that this additional term corresponds to a
\textit{displacement} of the center of the potential, plus an energy
shift:
\begin{equation}
  \hat{H}' = \frac{\hat{p}^2}{2} + \frac{1}{2}\left(\hat{x} - \sqrt{2}\alpha_1\right)^2 - \alpha_1^2.
\end{equation}
Let $|\alpha_1\rangle$ denote the ground state for the displaced and
shifted harmonic oscillator.  By analogy with how we solved the
original harmonic oscillator problem, let us define a new annihilation
operator with displaced $x$:
\begin{align}
  \hat{a}' = \frac{1}{\sqrt{2}}\left(\hat{x} - \sqrt{2}\alpha_1 + i \hat{p}\right).
\end{align}
This is related to the original annihilation operator by
\begin{equation}
  \hat{a}' = \hat{a} - \alpha_1.
  \label{aprime}
\end{equation}
We can easily show that $[\hat{a}',\hat{a}'^\dagger] = 1$, and that
$\hat{H}' = \hat{a}'^\dagger \hat{a}' + 1/2 - \alpha_1^2$.  Hence,
\begin{equation}
  \hat{a}' \, |\alpha_1 \rangle = 0.
\end{equation}
But Eq.~\eqref{aprime} implies that in terms of the \textit{original}
annihilation operator,
\begin{equation}
  \hat{a}\, |\alpha_1\rangle = \alpha_1 \,|\alpha_1\rangle.
  \label{aeigenv}
\end{equation}
In other words, $|\alpha_1\rangle$ is an eigenstate of the original
harmonic oscillator's annihilation operator, with the displacement
parameter $\alpha_1$ as the corresponding eigenvalue!  For reasons
that will become clear later, we call $|\alpha_1\rangle$ a
\textbf{coherent state} of the harmonic oscillator $\hat{H}$.

\section{Explicit expression for the coherent state}

Let us derive an explicit expression for the coherent state in terms
of $\hat{a}$ and $\hat{a}^\dagger$, the creation and annihilation
operators of the original harmonic oscillator.  Consider the
translation operator
\begin{equation}
  \hat{T}(\Delta x) = \exp(-i\hat{p}\Delta x).
\end{equation}
Since $|\alpha_1\rangle$ is the ground state of a displaced harmonic
oscillator, it can be generated by performing a displacement of the
original oscillator's ground state $|\varnothing\rangle$.  The
displacement is $\Delta x = \sqrt{2}\alpha_1$:
\begin{align}
  |\alpha_1\rangle &= \hat{T}\big(\sqrt{2}\alpha_1\big)\; |\varnothing\rangle  \\
  &= \exp\left[\alpha_1\left(\hat{a}^\dagger - \hat{a}\right)\right] |\varnothing\rangle.
\end{align}
In deriving the second line, we have used
Eqs.~\eqref{a}--\eqref{adagger} to express $\hat{p}$ in terms of
$\hat{a}$ and $\hat{a}^\dagger$.  We can further simplify the result
by using the Baker-Campbell-Hausdorff formula for operator
exponentials:
\begin{equation}
  \mathrm{If} \;\; [[\hat{A},\hat{B}],\hat{A}] = [[\hat{A},\hat{B}],\hat{B}] = 0
  \;\;\Rightarrow \;\; e^{\hat{A}+\hat{B}}
  = e^{-[\hat{A},\hat{B}]/2}\, e^{\hat{A}} e^{\hat{B}}.
\end{equation}
The result is
\begin{equation}
  |\alpha_1\rangle = e^{-\alpha_1^2/2} \, e^{\alpha_1 \hat{a}^\dagger} |\varnothing\rangle.
\end{equation}
If we write the exponential in its series form,
\begin{equation}
  |\alpha_1\rangle = e^{-\alpha_1^2/2} \, \left(1 + \alpha_1 \hat{a}^\dagger
  + \frac{\alpha_1^2}{2} \left(\hat{a}^\dagger\right)^2 + \cdots\right)
 |\varnothing\rangle,
\end{equation}
then we see that from the point of view of the bosonic excitations of
the original Hamiltonian $\hat{H}$, the state $|\alpha_1\,\rangle$ has
an \textit{indeterminate number of bosons}.  It is a superposition of
the zero-boson (vacuum) state, a one-boson state, a two-boson state,
etc.

We can generalize the coherent state by performing a shift not just in
space, but also in momentum.  Instead of Eq.~\eqref{hshift}, let us
define
\begin{align}
  \hat{H}' &= \frac{1}{2}\left(\hat{p} - \sqrt{2}\alpha_2\right)^2
  + \frac{1}{2}\left(\hat{x} - \sqrt{2}\alpha_1\right)^2 \\
  &= \hat{H} - \left(\alpha \hat{a}^\dagger + \alpha^*\hat{a}\right)
  + \textrm{constant},
\end{align}
where
\begin{equation}
  \alpha \equiv \alpha_1 + i \alpha_2 \;\in \;\mathbb{C}.
\end{equation}
It can then be shown that the ground state of $\hat{H}'$, which we
denote by $|\alpha\rangle$, satisfies
\begin{equation}
  \hat{a} \, |\alpha\rangle = \alpha \,|\alpha\rangle.
  \label{aaction}
\end{equation}
(Note that $\hat{a}$ is not Hermitian, so its eigenvalue $\alpha$ need
not be real.)  In explicit terms,
\begin{equation}
  |\alpha\rangle
  \,=\, \exp\left[\alpha\hat{a}^\dagger - \alpha^*\hat{a}\right] |\varnothing\rangle
  \,=\, e^{-|\alpha|^2/2} e^{\alpha\hat{a}^\dagger} |\varnothing\rangle.
  \label{alphaexp}
\end{equation}

\section{Basic properties}

There is one coherent state $|\alpha\rangle$ for each complex number
$\alpha \in \mathbb{C}$.  They have the following properties:
\begin{enumerate}
\item They are normalized:
  \begin{equation}
    \langle\alpha|\alpha\rangle = 1.
    \label{normalization}
  \end{equation}
  This follows from the fact that they are ground states of displaced
  harmonic oscillators.

\item They form a complete set, meaning that the identity operator can
  be resolved as
  \begin{equation}
    \hat{I} = C \int d^2\alpha |\alpha\rangle\langle\alpha|,
  \end{equation}
  where $C$ is some numerical constant and $\int d^2\alpha$ denotes an
  integral over the complex plane.  However, the coherent states do
  \textit{not} form an orthonormal set, as they are over-complete:
  $\langle\alpha|\alpha'\rangle \ne 0$ for $\alpha \ne \alpha'$.

\item The expected number of particles in a coherent state is
  \begin{equation}
    \langle\alpha| \hat{a}^\dagger\hat{a} | \alpha\rangle
    = |\alpha|^2.
  \end{equation}

\item The probability distribution of the number of particles follows
  a \textit{Poisson distribution}:
  \begin{equation}
    P(n) = |\langle n | \alpha\rangle|^2 = e^{-|\alpha|^2} \frac{|\alpha|^{2n}}{n!}.
  \end{equation}
  The mean and variance of this distribution are both $|\alpha|^2$.

\item The mean position and momentum are
  \begin{align}
    \langle \alpha | \hat{x}|\alpha\rangle &= \sqrt{2} \, \mathrm{Re}(\alpha) 
    \label{x} \\
    \langle \alpha | \hat{p}|\alpha\rangle &= \sqrt{2} \, \mathrm{Im}(\alpha).
    \label{p}
  \end{align}
\end{enumerate}

\section{Dynamical properties}

Take the harmonic oscillator Hamiltonian with zero-point energy
omitted for convenience:
\begin{equation}
  \hat{H} = \hat{a}^\dagger \hat{a}.
\end{equation}
Suppose we initialize the system in a coherent state
$|\alpha_0\rangle$ for some $\alpha_0 \in \mathbb{C}$.  This is not an
energy eigenstate of $\hat{H}$, so how will it subsequently evolve?

It turns out that the dynamical state has the form
\begin{equation}
  |\psi(t)\rangle = |\alpha(t)\rangle, \;\;\;\mathrm{where}\;\;\alpha(0) = \alpha_0.
\end{equation}
In other words, the system is always in a coherent state, but the
complex parameter $\alpha(t)$ varies with time.  To find $\alpha(t)$,
plug the ansatz into the time-dependent Schr\"odinger equation:
\begin{align}
  i \frac{d}{dt} |\alpha(t)\rangle &= \hat{a}^\dagger \hat{a} |\alpha(t)\rangle\\
  i \,\Big\langle\alpha(t)\Big| \frac{d}{dt} \Big|\alpha(t)\Big\rangle &=
  |\alpha(t)|^2.
\end{align}
We can calculate the left-hand side using Eqs.~\eqref{aaction},
\eqref{alphaexp}, and \eqref{normalization}:
\begin{align}
  i \, \Big\langle\alpha(t)\Big| \frac{d}{dt} \Big|\alpha(t)\Big\rangle
  &= i\, \langle \alpha(t) | \; \frac{d}{dt}
  \exp\left[ e^{-|\alpha|^2/2} e^{\alpha\hat{a}^\dagger} \right]
  |\varnothing\rangle \\
  &= i\, \langle \alpha(t) | \left(-\frac{1}{2} \frac{d}{dt}(\alpha\alpha^*)
  + \frac{d\alpha}{dt} \hat{a}^\dagger \right) |\alpha(t)\rangle \\
  &= i \left(-\frac{1}{2} \dot{\alpha}\alpha^*
  - \frac{1}{2}\alpha\dot{\alpha}^* + \dot{\alpha}\alpha^*\right).
\end{align}
Hence,
\begin{equation}
  \frac{i}{2}\left(\dot{\alpha}\alpha^* - \alpha\dot{\alpha}^*\right) = |\alpha|^2.
  \label{eom}
\end{equation}
This looks complicated, but it's not.  Observe that if
\begin{equation}
  \frac{d\alpha}{dt} = -i \alpha(t)
  \label{eom2}
\end{equation}
Then Eq.~\eqref{eom} is automatically satisfied.  The solution to
Eq.~\eqref{eom2} that satisfies the initial condition $\alpha(t) =
\alpha_0$ is simply
\begin{equation}
  \alpha(t) = \alpha_0 \; e^{-it}.
\end{equation}
Referring back to Eqs.~\eqref{x}--\eqref{p}, this implies that the
mean position and momentum have the following time-dependence:
\begin{align}
  \langle x\rangle &=\, \;\;\sqrt{2} |\alpha_0| \cos\left[t - \mathrm{arg}(\alpha_0)\right] \\
  \langle p\rangle &= -\sqrt{2} |\alpha_0| \sin\left[t - \mathrm{arg}(\alpha_0)\right].
\end{align}
The dynamics of a coherent state therefore reproduces the motion of a
classical harmonic oscillator with $m = \omega_0 = 1$.

\end{document}
