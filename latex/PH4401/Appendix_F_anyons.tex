\documentclass[pra,12pt]{revtex4}
\usepackage{amsmath}
\usepackage{amssymb}
\usepackage{graphicx}
\usepackage{color}
\usepackage{mathrsfs}
\usepackage{calligra}
\usepackage[pdfborder={0 0 0},colorlinks=true,linkcolor=blue,urlcolor=blue]{hyperref}

\def\ket#1{\left|#1\right\rangle}
\def\bra#1{\left\langle#1\right|}
\def\braket#1{\left\langle#1\right\rangle}

\DeclareMathAlphabet{\mathcalligra}{T1}{calligra}{m}{n}
\DeclareFontShape{T1}{calligra}{m}{n}{<->s*[2.4]callig15}{}

\usepackage{fancyhdr}
\fancyhf{}
\lhead{\tiny Y.~D.~Chong}
\rhead{\scriptsize PH4401: Quantum Mechanics III}
\lfoot{}
\rfoot{\thepage}
\pagestyle{fancy}

\setlength{\parindent}{14pt}
\renewcommand{\theequation}{E.\arabic{equation}}
\renewcommand{\thesection}{E\arabic{section}}

\renewcommand{\baselinestretch}{1.0}
\setlength{\parskip}{0.07in}

\begin{document}

\begin{center}
{\large \textbf{Appendix F: Anyons}}
\end{center}

One of the strangest consequences of magnetic vector potentials
(introduced in Chapter 4) is that they can influence the statistics of
identical particles.  In two spatial dimensions---and \textit{only} in
2D---vector potentials make possible a class of identical particles
known as \textbf{anyons}, which act like neither the fermions nor
bosons discussed in Chapter 3.  They have a form of particle exchange
symmetry that is intermediate between the fermionic and bosonic cases.

\section{Bound Flux Tubes}

The theory of anyons generated by vector potentials was developed by
Wilczek.  He considered a scenario in which identical particles move
in 2D (the $x$-$y$ plane), with each particle carrying a ``flux tube''
corresponding to an infinitely thin concentration of magnetic flux
$\Phi_B$ (pointing out of the plane, along $z$).

As we saw in Chapter 4, Section I.C, each flux tube can be described
by a singular vector potential.  If $\mathbf{r}_n$ is the center of
the flux tube, the vector potential has the form
\begin{equation}
  \mathbf{A}^{(n)}(\mathbf{r}) = \frac{\Phi_B}{2\pi
    |\mathbf{r}-\mathbf{r}_n|} \; \mathbf{e}_\phi^{(n)}(\mathbf{r}),
  \label{Asolenoid}
\end{equation}
where $\mathbf{e}_\phi^{(n)}(\mathbf{r})$ denotes the azimuthal unit
vector at position $\mathbf{r}$ relative to the origin $\mathbf{r}_n$.
The superscript $(n)$ is used to denote that the vector potential
field is centered at particle $n$.

Suppose the particles have charge $-e$.  Each particle $m$ is acted
upon by the vector potentials from all the other particles, which
appear in the Hamiltonian according to the prescription
\begin{equation}
  \hat{\mathbf{p}}_m \rightarrow \hat{\mathbf{p}}_m
  + \sum_{n \ne m} \mathbf{A}^{(n)}(\hat{\mathbf{r}}_m),
\end{equation}
where $\hat{\mathbf{p}}_m$ is the momentum operator for particle $m$.
Note that each particle's flux tube does not act upon itself (just as,
in the case of electrostatic forces, the electrostatic force of a
particle on itself is conventionally ignored).  Assuming there are no
other potentials present in the system, the non-relativistic
Hamiltonian for the system of particles is
\begin{equation}
  \hat{H} = \frac{1}{2m} \sum_m \left| \,\hat{\mathbf{p}}_m
  + \sum_{n \ne m} \mathbf{A}^{(n)}(\hat{\mathbf{r}}_m)\,\right|^2.
\end{equation}

Let us consider the case of two particles, and adopt the wavefunction
representation.  The Hamiltonian is
\begin{equation}
  \hat{H} = \frac{1}{2m} \left[ \left| \, -i\hbar \nabla_1
  + \mathbf{A}^{(2)}(\mathbf{r}_1)\,\right|^2
  + \left| \, -i\hbar \nabla_2
  + \mathbf{A}^{(1)}(\mathbf{r}_2)\,\right|^2\right],
\end{equation}
where $\nabla_m$ is the gradient operator constructed from partial
derivatives on $\mathbf{r}_m$, with the other particle position fixed.
This Hamiltonian acts upon two-particle wavefunctions of the form
$\psi(\mathbf{r}_1, \mathbf{r}_2)$, which obeys either fermionic or
bosonic exchange symmetries:
\begin{equation}
  \psi(\mathbf{r}_1, \mathbf{r}_2) = \sigma \psi(\mathbf{r}_2, \mathbf{r}_1),
  \label{exchange}
\end{equation}
where $\sigma$ is the eigenvalue of the exchange operator (i.e.,
$\sigma = 1$ for bosons, and $\sigma = -1$ for fermions).

\section{Gauge transformation}

The two particle coordinates can be written out in Cartesian form as
$(x_1, y_1, x_2, y_2)$.  It will be convenient to re-express these in
terms of center-of-mass and relative polar coordinates $(R, \theta,
\mathcalligra{r}, \phi)$, defined as follows:
\begin{align}
  \left\{\;
  \begin{aligned}
  R &= \sqrt{\left(\frac{x_1+x_2}{2}\right)^2 + \left(\frac{y_1+y_2}{2}\right)^2} \\
  \theta &= \tan^{-1}\left(\frac{y_1+y_2}{x_1+x_2}\right) \\
  \mathcalligra{r} &= \sqrt{\left(x_1-x_2\right)^2 + \left(y_1-y_2\right)^2} \\
  \phi &= \tan^{-1}\left(\frac{y_1-y_2}{x_1-x_2}\right)
  \end{aligned}
  \right.
  \;\;\leftrightarrow\;\;
  \left\{ \;
  \begin{aligned}
    x_1 &= R \cos\theta + \frac{\mathcalligra{r}}{2} \,\cos\phi \\
    y_1 &= R \sin\theta + \frac{\mathcalligra{r}}{2} \,\sin\phi \\
    x_2 &= R \cos\theta - \frac{\mathcalligra{r}}{2} \,\cos\phi \\
    y_2 &= R \sin\theta - \frac{\mathcalligra{r}}{2} \,\sin\phi.
  \end{aligned}
  \right.
  \label{reparm}
\end{align}
Note that this reparameterization is specific to 2D.

From the right-hand side of Eq.~\eqref{reparm}, we see that doing the
transformation $\phi \rightarrow \phi \pm \pi$, while keeping $(R,
\theta, \mathcalligra{r})$ constant, is equivalent to exchanging the
values of $(x_1,y_1)$ and $(x_2,y_2)$.  In other words, the two
particles can be exchanged by a rotation of $\pm \pi$ around their
fixed center of mass.  Thus, for two identical particles,
Eq.~\eqref{exchange} can be written as
\begin{equation}
  \psi(R, \theta, \mathcalligra{r}, \phi \pm \pi) \;=\;
  \sigma \, \psi(R, \theta, \mathcalligra{r}, \phi).
\end{equation}
where $\sigma = 1$ for bosons and $\sigma = -1$ for fermions.


\end{document}
